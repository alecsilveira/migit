\chapter{Mas}



\section{Salto de l'inea y de p'agina}

\subsection{P'arrafos justificados}

Normalmente los libros se suelen componer con todos los renglones del
mismo tama~no. \LaTeX{} inserta los saltos de l'inea y los espacios
entre las palabras optimizando el contenido de los p'arrafos enteros.
Si es necesario, tambi'en introduce guiones, dividiendo las palabras
que no encajen bien al final de los renglones. El modo de componer los
p'arrafos depende de la clase de documento. Normalmente se introduce
una sangr'ia horizontal en la primera l'inea de un p'arrafo y no se
introduce espacio adicional entre cada dos p'arrafos. Para m'as
informaci'on v'ease el apartado~\ref{parsp}.

En casos especiales se podr'ia ordenar a \LaTeX{} que introduzca un
salto de l'inea.

\begin{command}
\ci{\bs} o \ci{newline}
\end{command}
\noindent comienza una l'inea nueva sin comenzar un p'arrafo nuevo.

\begin{command}
\ci{\bs*}
\end{command}
\noindent adem'as proh'ibe que se produzca un salto de p'agina tras el
salto de l'inea.

\begin{command}
\ci{newpage}
\end{command}
\noindent comienza una p'agina nueva.
\pagebreak

\begin{command}
\ci{linebreak}\verb|[|\emph{n}\verb|]|,
\ci{nolinebreak}\verb|[|\emph{n}\verb|]|,
\ci{pagebreak}\verb|[|\emph{n}\verb|]| and
\ci{nopagebreak}\verb|[|\emph{n}\verb|]|
\end{command}
\noindent hacen lo que indican sus nombres: salto de l'inea,
ning'un salto de l'inea, salto de p'agina y ning'un salto de
p'agina. Adem'as le permite al autor el influir sobre sus acciones
a trav'es del argumento opcional \emph{n}. Se puede establecer a
un valor entre cero y cuatro. Al poner \emph{n} menor de 4 se le
deja a \LaTeX{} la posibilidad de ignorar la orden si el resultado
resulta muy malo.

\LaTeX{} siempre intenta realizar los saltos de l'inea lo mejor
posible. Si no puede encontrar ninguna posibilidad satisfactoria para
producir los bordes de los p'arrafos totalmente rectos, cumpliendo con
las reglas impuestas, entonces dejar'a un rengl'on demasiado largo.
En este caso \LaTeX{} producir'a el correspondiente mensaje de
advertencia (``\wni{overfull box}'') mientras procesa el fichero de
entrada. Esto sucede en especial si no se encuentra un lugar apropiado
para introducir un gui'on entre las s'ilabas. Si se introduce la orden
\ci{sloppy}, \LaTeX{} ser'a menos severo en sus exigencias y evita
tales renglones con longitudes mayores, aumentando la separaci'on
entre las palabras ---si bien el resultado final no es de lo mejor---.
En este caso se dan mensajes de advertencia (``\wni{underfull
  hbox}''). El resultado suele ser perfectamente aceptable la mayor'ia
de las veces. La orden \ci{fussy} act'ua en sentido contrario. Esto
podr'ia hacerlo en caso que desee ver a \LaTeX{} quejarse en todos los
sitios.


\subsection{Silabeo} \label{hyph}

\LaTeX{} silabea las palabras cuando resulta necesario. Si el
algoritmo de silabeo no produce los resultados correctos, entonces se
puede remediar esta situaci'on con 'ordenes como las que presentamos a
continuaci'on. Esto suele ser especialmente necesario en palabras
compuestas o de idiomas extranjeros.

La instrucci'on
\begin{command}
\ci{hyphenation}\verb|{|\emph{lista de palabras}\verb|}|
\end{command}
\noindent da lugar a que las palabras mencionadas en ella se puedan
dividir en cualquier momento en, y s'olo en, los lugares indicados con
``\verb|-|''\@. Esta orden deber'ia aparecer en el pre'ambulo del
fichero de entrada y deber'ia contener solamente palabras construidas
sin caracteres especiales.
%%% Alguna idea de la definici'on de ``letras normales''???
%%% Es lo que aparece en el documento en ingl'es...
No se hacen distinciones entre las letras may'usculas y min'usculas
de las palabras a las que se refiera esta orden. El ejemplo siguiente
permitir'a localizar las s'ilabas de ``fichero'' y ``Fichero'' del
mismo modo, e impedir'a que en las palabras ``FORTRAN'', ``Fortran'' y
``fortran'' se introduzcan guiones. No se permiten caracteres con
acentos o s'imbolos en el argumento.
%%% Pues podr'ian plante'arselo por lo menos... :-)

Ejemplo:
\begin{code}
\verb|\hyphenation{FORTRAN fi-che-ro}|
\end{code}

Dentro de una palabra, la instrucci'on \ci{-} establece un sitio donde
colocar un gui'on si fuese necesario. Adem'as, 'estos se convierten en
los 'unicos lugares donde se permite introducir los guiones en esta
palabra. Esta instrucci'on es especialmente 'util para las palabras
que contienten caracteres especiales (como, por ej., los caracteres
con acento ortogr'afico), ya que \LaTeX{} no silabea de modo
autom'atico las palabras que contienen estos caracteres.
%\footnote{A no ser que est'e usando los nuevos
%\wi{tipos DC}}.

\begin{example}
Me parece que esto es: su\-per\-%
ca\-li\-fra\-gi\-lis\-ti\-co\-%
ex\-pia\-li\-do\-so
\end{example}

Tambi'en se pueden se pueden mantener varias palabras en el mismo
rengl'on con la orden
\begin{command}
\ci{mbox}\verb|{|\emph{texto}\verb|}|
\end{command}
\noindent Hace que su argumento se mantenga siempre unido bajo cualquier
circunstancia, o sea, que no se puede dividir.

\begin{example}
Dentro de poco tendr'e otro tel'efono.
Ser'a el \mbox{(0203) 3783-225}.

El par'ametro \mbox{\emph{nombre
de fichero}} debe contener el nombre
del fichero.
\end{example}


\subsection{Guiones y rayas}

\LaTeX{} reconoce cuatro tipos de \wi{guiones}. Para tener acceso a
tres de 'estos se pone una cantidad diferente de guiones consecutivos.
El cuarto tipo es el signo matem'atico `menos':
\index{-}%
\index{--}%
\index{---}%
\index{-@$-$}%
\index{matem'atico!menos}%

\begin{example}
psico-terap'eutico \\
10--18~horas \\
Madrid -- Barcelona \\
?`S'i? ---dijo ella--- \\
0, 1 y $-1$
\end{example}

% En ingl'es, los nombres de estos guiones son:
% \texttt{-} \wi{hyphen}, \texttt{--} \wi{en-dash},
% \texttt{---} \wi{em-dash} y \verb|$-$| \wi{signo menos}.

\subsection{Puntos suspensivos (`\ldots')}

En una m'aquina de escribir, tanto para la \wi{coma} como para el
\wi{punto} se les da el mismo espaciado que a cualquier otro
car'acter. En la impresi'on de libros, estos caracteres s'olo ocupan
un peque~no espacio y se colocan muy pr'oximos al car'acter que les
precede. Por esto, los ``puntos suspensivos'' no se pueden introducir
con tres puntos normales, ya que no tendr'ian el espaciado correcto.
Para estos puntos existe una instrucci'on especial llamada
\begin{command}
\ci{ldots}
\end{command}
\index{...@\ldots}
\begin{example}
No as'i ... sino as'i:\\
New York, Tokyo, Budapest\ldots
\end{example}

\subsection{Ligaduras}

Algunas combinaciones de letras no se componen con las distintas
letras que la forman, sino que, de hecho, se usan s'imbolos
especiales.
\begin{code}
{\large ff fi fl ffi\ldots}\quad
en lugar de\quad {\large f{}f f{}i f{}l f{}f{}i \ldots}
\end{code}
Estas \wi{ligaduras} se pueden evitar intercalando \ci{mbox}\verb|{}|
entre el par letras en cuesti'on.





\section{Distancias entre palabras}

Para conseguir un margen derecho recto en la salida, \LaTeX{}
introduce cantidades variables de espacios entre las palabras. Al
final de una oraci'on, introduce unos espacios algo mayores que
favorecen la legibilidad del texto. \LaTeX{} presupone que las frases
acaban con puntos, signos de interrogaci'on y de admiraci'on. Si hay
un punto tras una letra may'uscula, entonces esto no se considera el
fin de una oraci'on ya que los puntos tras las letras may'usculas
normalmente se utilizan para abreviaturas.

El autor debe indicar cualquier excepci'on a estas reglas. Una
\emph{barra invertida} \verb|\| antes de un espacio en blanco produce
un espacio en blanco que no se ensanchar'a. Un car'acter de
tilde~`\verb|~|'\ genera un espacio que no se puede ensanchar y en el
que no se puede producir ning'un cambio de rengl'on. Si antes de un
punto aparece la instrucci'on \verb|\@|, significa que este punto
acaba una oraci'on, aunque se encuentre tras una letra may'uscula.
\cih{"@} \index{~@\verb.~.} \index{tilde@tilde (\verb.~.)} \index{.!
  espacio tras}

\begin{example}
En la fig.\ 1 del cap.\ 1\dots \\
El Dr.~L'opez se encuentra \\
con D~na.~P'erez. \\
\dots\ 5~m de ancho. \\
Necesito vitamina~C\@. ?`Y t'u?
\end{example}

Este tratamiento especial para los espacios al final de las oraciones
se puede evitar con la instrucci'on
\begin{command}
\ci{frenchspacing}
\end{command}
\noindent que le indica a \LaTeX{} que \emph{no} introduzca m'as
espacios tras un punto que tras cualquier otro car'acter. Esto es muy
com'un en diversos idiomas, como es el caso del espa~nol. En este caso
la instrucci'on \verb|\@| no es necesaria.




\section{Referencias cruzadas}

En los libros, informes y art'iculos existen, a menudo,
\wi{referencias cruzadas} a figuras, tablas y segmentos especiales de
texto que se hayan en otros lugares del documento. \LaTeX{}
proporciona las siguientes instrucciones para producir referencias
cruzadas:
\begin{command}
  \ci{label}\verb|{|\emph{marcador}\verb|}|,
  \ci{ref}\verb|{|\emph{marcador}\verb|}| y
  \ci{pageref}\verb|{|\emph{marcador}\verb|}|
\end{command}
\noindent donde \emph{marcador} es un identificador elegido por el
usuario. \LaTeX{} reemplaza \verb|\ref| por el n'umero del apartado,
subapartado, figura, tabla o teorema donde se introdujo la
instrucci'on \verb|\label| correspondiente. La orden \verb|\pageref|
imprime el n'umero de p'agina donde se produce la orden \verb|\label|
con igual argumento. Aqu'i tambi'en se utilizan los n'umeros del
procesamiento anterior.

\begin{example}
Una referencia a este subapartado
\label{sec:este} aparecer'ia como:

``vea el apartado~\ref{sec:este} en
la p'agina~\pageref{sec:este}.''
\end{example}

\section{Notas a pie de p'agina}

Con la instrucci'on
\begin{command}
\ci{footnote}\verb|{|\emph{texto de la nota al pie}\verb|}|
\end{command}
\noindent se imprimir'a una nota en el pie de la p'agina actual.

\begin{example}
Las notas a pie de p'agina%
\footnote{Esta es una nota a pie
de p'agina} son utilizadas con
frecuencia por la gente que usa
\LaTeX.
\end{example}

Tambi'en existe una variante de esta instrucci'on, que es
\begin{command}
\ci{footnote}\verb|[|\emph{n'umero}\verb|]{|\emph{texto de la nota al
    pie}\verb|}|
\end{command}
De esta forma para la nota al pie correspondiente se emplear'a para el
marcador el \emph{n'umero} que se ha indicado en vez del valor del
contador de notas al pie. Esta variante \emph{s'olo} se puede emplear
dentro de los p'arrafos.

\section{Palabras resaltadas}

En los escritos a m'aquina, para resaltar determinados segmentos de
texto 'estos se $\underline{\mathrm{subrayan}}$. En los libros
impresos estas palabras se \emph{resaltan} o se \emph{destacan}. La
orden con la que se cambia a un tipo de letra \emph{resaltado} es
\begin{command}
\ci{emph}\verb|{|\emph{texto}\verb|}|
\end{command}
\noindent Su argumento es el texto que se debe \wi{resaltar}.

\begin{example}
\emph{Si est'a empleando
\emph{resalte} en un texto
ya resaltado, entonces \LaTeX{}
utiliza \emph{redonda} para volver
a resaltar texto.}
\end{example}


\section{Entornos} \label{env}

Para componer textos con un prop'osito especial \LaTeX{} define muchos
tipos de \wi{entornos} para toda clase de dise~nos:
\begin{command}
\ci{begin}\verb|{|\emph{nombre}\verb|}|\quad
   \emph{texto}\quad
\ci{end}\verb|{|\emph{nombre}\verb|}|
\end{command}
\noindent donde \emph{nombre} es el nombre del entorno. Los entornos
son ``grupos'' o ``agrupaciones''. Tambi'en se puede cambiar a un
nuevo entorno dentro de otro, en cuyo caso debe tenerse cuidado con la
secuencia:
\begin{code}
\verb|\begin{aaa}...\begin{bbb}...\end{bbb}...\end{aaa}|
\end{code}

En los apartados siguientes se explican todos los entornos importantes.

\subsection{Listas y descripciones (\texttt{itemize},
  \texttt{enumerate}, \texttt{description})}

El entorno \ei{itemize} es adecuado para las listas sencillas, el
entorno \ei{enumerate} para relaciones numeradas y el entorno
\ei{description} para descripciones.\cih{item}

\begin{example}
\begin{enumerate}
\item Puede mezclar los entornos
de listas a su gusto:
\begin{itemize}
\item Pero podr'ia comenzar a
perecer inc'omodo.
\item Si abusa de ellas.
\end{itemize}
\item Por lo tanto, recuerde:
\begin{description}
\item[Lo innecesario] no va a
resultar adecuado porque
lo coloque en una lista.
\item[Lo adecuado,] sin embargo,
se puede presentar agradablemente
en una lista.
\end{description}
\end{enumerate}
\end{example}

\subsection{Justificaciones y centrado (\texttt{flushleft},
            \texttt{flushright}, \texttt{center})}

Los entornos \ei{flushleft} y \ei{flushright} producen p'arrafos
justificados a la izquierda y a la derecha (sin nivelaci'on de
bordes).%
\index{justificado a la izquierda}\index{justificado a la derecha} El
entorno \ei{center} genera texto centrado. Si no se introduce \ci{\bs}
para dividir los renglones, entonces \LaTeX{} lo har'a
autom'aticamente.

\begin{example}
\begin{flushleft}
Este texto est'a\\ justificado a
la izquierda. \LaTeX{} no intenta
forzar que todas las l'ineas
tengan longitud.
\end{flushleft}
\end{example}

\begin{example}
\begin{flushright}
Este texto est'a\\ justificado a
la derecha. \LaTeX{} no intenta
forzar que todas las l'ineas
tengan igual longitud.
\end{flushright}
\end{example}

\begin{example}
\begin{center}
En el centro\\de la tierra
\end{center}
\end{example}

\subsection{Citas (\texttt{quote}, \texttt{quotation}, \texttt{verse})}

El entorno \ei{quote} sirve para citas peque~nas, ejemplos y para
resaltar oraciones.
\begin{example}
Una regla de oro en tipograf'ia
para el largo de los renglones
dice:
\begin{quote}
Ning'un rengl'on debe contener
m'as de 66~letras.
\end{quote}
Por esto se suelen utilizar varias
columnas en los peri'odicos.
\end{example}
% Curioso lo que hace el ``66~letras'' en el primer rengl'on. Hay un
% modo de indicar preferencias entre la divisi'on de ``debe'' y
% ``66~letras''?


Hay dos entornos muy parecidos: el entorno \ei{quotation} y el entorno
\ei{verse}. El entorno \texttt{quotation} es adecuado para citas
mayores que consten de varios p'arrafos. El entorno \texttt{verse} es
apropiado para poemas en los que la separaci'on de los renglones es
esencial. Los versos (los renglones) se dividen con \ci{\bs} y las
estrofas con renglones en blanco.

\begin{example}
\begin{flushleft}
\begin{verse}
Soberano gofio en polvo,\\
sustento de mi barriga,\\
el d'ia que no te como\\
para m'i no hay alegr'ia.
\end{verse}
\end{flushleft}
\end{example}

\subsection{Edici'on directa (\texttt{verbatim}, \texttt{verb})}

El texto que se encuentre entre \verb|\begin{|\ei{verbatim}\verb|}| y
  \verb|\end{verbatim}| aparecer'a tal como se ha introducido, como si
se hubiese escrito con una m'aquina de escribir, con todos los espacios
en blanco y cambios de l'inea y sin interpretaci'on de las
instrucciones de \LaTeX.

Dentro de un p'arrafo se puede lograr el mismo efecto con
\begin{command}
\ci{verb}\verb|+|\emph{text}\verb|+|
\end{command}

\noindent El \verb|+| s'olo es un ejemplo de car'acter delimitador. Se
puede usar cualquier car'acter excepto las letras, \verb|*| o
caracteres en blanco.

\begin{example}
La instrucci'on \verb|\ldots|%
\ldots

\begin{verbatim}
10 PRINT "HELLO WORLD ";
20 GOTO 10
\end{verbatim}
\end{example}

\begin{example}
\begin{verbatim*}
La version con estrella del
entorno          verbatim
destaca los espacios     en
el  texto
\end{verbatim*}
\end{example}

La instrucci'on \ci{verb} se puede usar, del mismo modo, con un
asterisco:
\begin{example}
\verb*|de esta   manera :-) |
\end{example}

El entorno \texttt{verbatim} y la instrucci'on \verb|\verb| no pueden
utilizarse como par'ametros de otras instrucciones.


\subsection{Estadillos (\texttt{tabular})}

El entorno \ei{tabular} sirve para crear \wi{estadillos}, con l'ineas
horizontales y verticales seg'un se desee. \LaTeX{} determina el ancho
de las columnas de modo autom'atico.

El argumento \emph{especificaciones del estadillo} de la instrucci'on
\begin{command}
\verb|\begin{tabular}{|\emph{especificaciones del estadillo}\verb|}|
\end{command}
\noindent define el dise~no del estadillo. Utilice \texttt{l} para una
columna con texto justificado a la izquierda, \texttt{r} para
justificar el texto a la derecha, \texttt{c} para texto centrado,
\verb|p{|\emph{ancho}\verb|}| para una columna que contenga texto con
saltos de l'inea, y \verb.|. para una l'inea vertical.


Dentro de un entorno \texttt{tabular}, \verb|&| salta a la pr'oxima
columna, \ci{\bs} separa los renglones y \ci{hline} introduce una
l'inea horizontal.
\index{"|@ \verb."|.}
\begin{example}
\begin{tabular}{|r|l|}
\hline
7C0 & hexadecimal \\
3700 & octal \\
11111000000 & binario \\
\hline \hline
1984 & decimal \\
\hline
\end{tabular}
\end{example}

\begin{example}
\begin{tabular}{|p{4.7cm}|}
\hline
Bienvenido al p'arrafo del Sr.\
Caj'on. Esperamos que disfrute
del espect'aculo.\\
\hline
\end{tabular}

\end{example}

Con la construcci'on \verb|@{...}| se puede especificar el separador
de columnas. Esta construcci'on elimina el espacio entre columnas y lo
reemplaza con lo que se haya introducido entre los par'entesis. Un uso
muy frecuente de esta construcci'on se explica m'as adelante con el
problema de la alineaci'on de la coma decimal. Otro uso posible es
para eliminar el espacio que antecede y precede a los renglones de una
tabla con \verb|@{}|.

\begin{example}
\begin{tabular}{@{} l @{}}
\hline
ning'un espacio a la izquierda
ni derecha\\\hline
\end{tabular}
\end{example}
\begin{example}
\begin{tabular}{l}
\hline
espacios a la izquierda
y a la derecha\\
\hline
\end{tabular}
\end{example}

\index{alineaci'on decimal} Ya que no hay ning'un mecanismo
incorporado para alinear columnas num'ericas sobre la coma decimal
\footnote{Si se halla instalado el conjunto `tools'\ en su sistema,
  eche un vistazo al paquete \pai{dcolumn}.}, podr'iamos ``imitarlo''
usando dos columnas: un entero alineado a la derecha y luego los
decimales a la izquierda. La instrucci'on \verb|@{,}| en el argumento
de \verb|\begin{tabular}| reemplaza el espacio normal entre columnas
  con una ``,'', dando la apariencia de una 'unica columna justificada
  por la coma decimal. !`No se olvide de reemplazar la coma decimal en
  sus n'umeros con un separador de columna (\verb|&|)! Se puede
  colocar una etiqueta sobre nuestra ``columna'' num'erica empleando
  la instrucci'on \ci{multicolumn}.

\begin{example}
\begin{tabular}{c r @{,} l}
Expresi'on en pi       &
\multicolumn{2}{c}{Valor} \\
\hline
$\pi$               & 3&1416  \\
$\pi^{\pi}$         & 36&46   \\
$(\pi^{\pi})^{\pi}$ & 80662&7 \\
\end{tabular}
\end{example}

\section{Elementos flotantes}

Hoy en d'ia, la mayor'ia de las publicaciones contienen muchas
ilustraciones y tablas. Estos elementos necesitan un tratamiento
especial porque no se pueden cortar entre p'aginas. Un m'etodo podr'ia
ser comenzando una p'agina nueva cada vez que una ilustraci'on o una
tabla sea demasiado larga para caber en la p'agina actual. Este
enfoque deja p'aginas parcialmente vac'ias, lo que resulta poco
est'etico.

La soluci'on a este problema es hacer que cualquier ilustraci'on o
tabla que no quepa en la p'agina actual `flote'\ hasta una p'agina
posterior mientras se rellena la p'agina actual con el texto del
documento.

\LaTeX{} ofrece dos entornos para los \wi{elementos flotantes}. Uno
 para las tablas y otro para las ilustraciones. Para aprovechar
 completamente estos dos entornos es importante entender
 aproximadamente c'omo maneja \LaTeX{} estos objetos flotantes
 internamente. Si no, los objetos flotantes se pueden convertir en una
 fuente de frustaciones porque \LaTeX{} nunca los pone donde Vd.\
 quiere que vayan.

\bigskip
Primeramente, echemos un vistazo a las instrucciones que \LaTeX{}
proporciona para objetos flotantes.

Cualquier cosa que se incluya en un entorno \ei{figure} o \ei{table}
ser'a tratado como materia flotante. Ambos entornos flotantes
proporcionan un par'ametro opcional
\begin{command}
\verb|\begin{figure}[|\emph{designador de colocado}\verb|]| o\\
\verb|\begin{table}[|\emph{designador de colocado}\verb|]|
\end{command}
\noindent llamado el \emph{designador de colocado}. Este par'ametro se
emplea para indicarle a \LaTeX{} los lugares donde se permite que vaya
colocado el objeto flotante. Un \emph{designador de colocado} se
construye con una cadena de \emph{permisos de colocaci'on flotante}.
V'ease la tabla~\ref{tab:permiss}.

\begin{table}[!bp]
\caption{Permisos de colocaci'on flotante}\label{tab:permiss}
\noindent \begin{minipage}{\textwidth}
\medskip
\begin{center}
\begin{tabular}{@{}cp{10cm}@{}}
  Designador&Permiso para colocar el objeto flotante\ldots\\ \hline
  \rule{0pt}{1.05em}
\texttt{h} & aqu'i (\emph{here}), muy pr'oximo al
  lugar en el texto donde se ha introducido. Es 'util, principalmente,
  para objetos flotantes peque~nos.\\[0.3ex]
\texttt{t} & en la parte superior de una p'agina (\emph{top}).\\[0.3ex]
\texttt{b} & en la parte inferior de una p'agina
  (\emph{bottom}).\\[0.3ex]
\texttt{p} & en una \emph{p'agina} especial que s'olo contenga
  elementos flotantes.\\[0.3ex]
\texttt{!} & sin considerar la mayor'ia de los par'ametros
  internos\footnote{Como el n'umero m'aximo de elementos flotantes un
  una p'agina.} que impedir'ian a este objeto flotante que se colocase.
\end{tabular}
\end{center}
\end{minipage}
\end{table}

\pagebreak[3]
Una tabla se podr'ia comenzar con, por ejemplo, la siguiente l'inea:
\begin{code}
\verb|\begin{table}[!hbp]|
\end{code}
\noindent El \wi{designador de colocado} \verb|[!hbp]| le permite a
\LaTeX{} colocar la tabla justamente aqu'i (\texttt{h}) o al final
 (\texttt{b}) de alguna p'agina o en alguna p'agina especial para
 elementos flotantes, y en cualquier parte si no queda tan bien
 (\texttt{!}). Si no se da ning'un designador de colocado, entonces
 las clases normalizadas sobreentienden \verb|[tbp]|.

 \LaTeX{} colocar'a todos los objetos flotantes que encuentra seg'un
 los de\-sig\-na\-do\-res de colocado que haya indicado el autor. Si
 un objeto flotante no se puede colocar en la p'agina actual entonces
 se aplaza su colocaci'on, para lo cual se introduce en una
 cola\footnote{Son de tipo \emph{fifo}: lo que se introdujo primero es
   lo primero en extraerse.} de \emph{tablas} o \emph{figuras}
 (ilustraciones).  Cuando se comienza una nueva p'agina, lo primero
 que hace \LaTeX{} es confirmar si se puede construir una p'agina
 especial con los objetos flotantes que se hayan en las colas. Si no
 es posible, entonces se trata el primer objeto que se encuentra en
 las colas como si lo acab'asemos de introducir. Entonces \LaTeX{}
 vuelve a intentar colocar el objeto seg'un sus designadores de
 colocado (eso s'i, sin tener en cuenta la opci'on `\verb|h|',\ que ya
 no es posible). Cualquier objeto flotante nuevo que aparezca en el
 texto se introduce en la cola correspondiente. \LaTeX{} mantiene
 estrictamente el orden original de apariciones de cada tipo de objeto
 flotante.

Esta es la raz'on por la que una ilustraci'on que no se puede colocar
desplaza al resto de las figuras al final del documento. Por lo tanto:


\begin{quote}
  Si \LaTeX{} no coloca los objetos flotantes como esperaba, suele
  deberse 'unicamente a un objeto flotante que est'a atascando una de
  las dos colas de objetos flotantes.
\end{quote}

\bigskip
\noindent Adem'as, existen algunas cosas m'as que se deben indicar
sobre los entornos \ei{table} y \ei{figure}. Con la instrucci'on
\begin{command}
\ci{caption}\verb|{|\emph{texto de t'itulo}\verb|}|
\end{command}
\noindent se puede definir un t'itulo para el objeto flotante. \LaTeX{}
le a~nadir'a la cadena ``Figura'' o ``Tabla'' y un n'umero de secuencia.


Las dos instrucciones
\begin{command}
\ci{listoffigures} y \ci{listoftables}
\end{command}
\noindent funcionan de modo an'alogo a la orden
\verb|\tableofcontents|, imprimiendo un 'indice de figuras o de tablas
respectivamente. En estas listas se repetir'an los t'itulos
completos. Si Vd.\ tiende a utilizar t'itulos largos, deber'ia tener
una versi'on de estos t'itulos m'as cortos para introducirlos en estos
'indices. Esto se consigue dando la versi'on corta entre corchetes
tras la orden \verb|\caption|.
\begin{code}
\verb|\caption[Corto]{LLLLLaaaaaaaaarrrrrrrrgggggooooooo}|
\end{code}

Con \verb|\label| y \verb|\ref| se pueden crear referencias a un
objeto flotante dentro del texto.

El siguiente ejemplo dibuja un cuadrado y lo inserta en el
documento. Podr'ia utilizar esto si desea reservar espacios para
im'agenes que vaya a pegar en el documento acabado.

\begin{example}
La ilustraci'on~\ref{blanco} es un ejemplo del Pop-Art.
\begin{figure}[!hbp]
\makebox[\textwidth]{\framebox[5cm]{\rule{0pt}{5cm}}}
\caption{$5\times 5$ cent'imetros} \label{blanco}
\end{figure}
\end{example}

\noindent En el ejemplo anterior\footnote{suponiendo que la cola de
  figuras est'e vac'ia.} \LaTeX{} intentar'a \emph{por todos los
  medios}~(\texttt{!}) colocar la ilustraci'on exactamente
\emph{aqu'i}~(\texttt{h}). Si no puede, intentar'a colocarla en la
\emph{parte inferior}~(\texttt{b}) de la p'agina. Si no consigue
colocar esta figura en la p'agina actual, determina si es posible
crear una p'agina (\verb|p|) con elementos flotantes exclusivamente
que contenga esta ilustraci'on y algunas tablas que pudieran haber en
la cola de tablas. Si no hay material suficiente para una p'agina
especial de objetos flotante, entonces \LaTeX{} comienza una p'agina
nueva y otra vez trata la figura como si acabase de aparecer en el
texto.

Bajo determinadas condiciones podr'ia ser necesario emplear la orden
\begin{command}
\ci{clearpage}
\end{command}
\noindent Le ordena a \LaTeX{} que coloque \emph{inmediatamente} todos
los objetos flotantes que se hallen en las colas y despu'es comenzar
una p'agina nueva.

M'as adelante veremos c'omo incluir im'agenes en formato PostScript en
sus documentos de \LaTeXe.


\section{A~nadiendo instrucciones y entornos nuevos}

En el primer cap'itulo se explic'o que \LaTeX{} necesita informaci'on
sobre la estructura l'ogica del texto para elegir el formato
adecuado. Este es un concepto muy bien cuidado. Pero en la pr'actica
solemos chocar con las limitaciones que esto nos impone, ya que
\LaTeX{} simplemente no tiene el entorno especializado o la orden que
deseamos para un prop'osito espec'ifico.

Una soluci'on es emplear varias 'ordenes de \LaTeX{} para producir el
dise~no que se tiene en mente. Si tiene que hacer esto una vez, no hay
ning'un problema. Pero si esto sucede repetidamente, entonces lleva
mucho tiempo. Si alguna vez desease cambiar el formato tendr'ia que
revisar el fichero de entrada entero y editar todos los elementos en
cuesti'on.

Para resolver este problema, \LaTeX{} le permite definir sus propias
instrucciones y entornos.

\subsection{Instrucciones nuevas}

Para a~nadir sus propias instrucciones utilice la orden
\begin{command}
\ci{newcommand}\verb|{|%
   \emph{nombre}\verb|}[|\emph{num}\verb|]{|\emph{definici'on}\verb|}|
\end{command}
\noindent B'asicamente, la instrucci'on necesita dos argumentos: el
\emph{nombre} de la instrucci'on que quiere crear y la
\emph{definici'on} de la instrucci'on. El argumento entre corchetes
\emph{num} es opcional. Puede usarlo para crear 'ordenes nuevas que
tomen hasta 9 argumentos.

Los dos ejemplos siguientes deber'ian ayudarle a captar la idea. El
primer ejemplo define una instrucci'on nueva llamada
\verb|\udl|. Esta es una forma abreviada de introducir ``Una
Descripci'on de \LaTeXe''. Una orden como 'esta ser'ia muy 'util si
tuviese que escribir el t'itulo de este documento una y otra vez.

\begin{example}
\newcommand{\udl}
    {Una Descripci'on de \LaTeXe}
% en el cuerpo del documento :
``\udl'' \ldots{} ``\udl''
\end{example}

El siguiente ejemplo ilustra c'omo usar el argumento \emph{num}. La
secuencia \verb|#1| encuentra un sustituto en el argumento que
especifique. Si quisiera m'as de un argumento, emplee \verb|#2| y as'i
sucesivamente.

\begin{example}
\newcommand{\txsit}[1]
    {Una Descripci'on \emph{#1}
     Peque~na de \LaTeXe}
% en el cuerpo del documento:
\begin{itemize}
\item \txsit{no tan}
\item \txsit{muy}
\end{itemize}
\end{example}

\LaTeX{} no le permitir'a crear una instrucci'on nueva con un nombre
que ya existe. Si quiere ignorar de modo expl'icito una instrucci'on
existente tiene que utilizar \ci{renewcommand}. Aparte de su nombre,
utiliza la misma sintaxis que la instrucci'on \verb|\newcommand|. En
determinados casos podr'ia querer utilizar la instrucci'on
\ci{providecommand}. Funciona como \ci{newcommand}, pero si ya hay una
instrucci'on definida con este nombre, entonces \LaTeXe{} simplemente
ignora esta otra definici'on que acaba de indicar.


\subsection{Entornos nuevos}
De modo an'alogo a la instrucci'on \verb|\newcommand| existe una orden
para crear sus propios entornos. Cuando est'abamos escribiendo esta
introducci'on, hemos creado entornos especiales para estructuras que
se empleaban repetidamente en toda la descripci'on: ``ejemplos'',
``segmentos de c'odigo'' y ``cajas de definici'on de instrucciones''.
La instrucci'on \ci{newenvironment} utiliza la siguiente sintaxis:

\begin{command}
\ci{newenvironment}\verb|{|%
       \emph{nombre}\verb|}[|\emph{num}\verb|]{|%
       \emph{antes}\verb|}{|\emph{despu'es}\verb|}|
\end{command}

Al igual que la instrucci'on \verb|\newcommand|, se puede usar
\ci{newenvironment} con o sin argumento opcional. Lo que se
especifique en el argumento \emph{antes} se procesa antes que el texto
dentro del entorno. Lo que se indica en el argumento \emph{despu'es}
se procesa cuando se encuentra la instrucci'on
\verb|\end{|\emph{nombre}\verb|}|.

El siguiente ejemplo ilustra el empleo de la instrucci'on
\ci{newenvironment}.

\begin{source}
\newenvironment{king}
    {\begin{quote}}{\end{quote}}
% use esto en el cuerpo
\begin{king}
Mis humildes vasallos\ldots
\end{king}
\end{source}

El argumento \emph{num} se utiliza igual que la instrucci'on
\verb|\newcommand|. \LaTeX{} se asegura de que no defina un entorno
que ya exist'ia. Si alguna vez desea cambiar una entorno existente,
entonces puede utilizar la instrucci'on \ci{renewenvironment}. Tiene
la misma sintaxis que la instrucci'on \ci{newenvironment}.


%\endinput
