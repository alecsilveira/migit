
 
\chapter{Composici'on de f'ormulas matem'aticas}

\begin{intro}
  !`Ahora estese preparado! En este cap'itulo abordaremos el punto
  fuerte de \TeX: la composici'on matem'atica. Pero le advertimos que
  este cap'itulo s'olo mira la superficie. Mientras lo que aqu'i
  explicamos es suficiente para mucha gente, no desespere si no puede
  encontrar una soluci'on a sus necesidades de composici'on. Es muy
  probable que su problema est'e abordado en AMS-\LaTeXe
  \footnote{\texttt{CTAN:/tex-archive/macros/latex/packages/amslatex}}
  o en alg'un otro paquete.
\end{intro}
  
\section{Generalidades}

\LaTeX{} posee un modo especial para componer \wi{matem'aticas}. En un
p'arrafo, el texto matem'atico se introduce entre \ci{(} y \ci{)},
\index{$@\texttt{\$}} entre \texttt{\$} y \texttt{\$} o entre
\verb|\begin{|\ei{math}\verb|}| y \verb|\end{math}|.
\index{formulas@f'ormulas}

\begin{example}
Siendo $a$ y $b$ los catetos
y $c$ la hip'otenusa
de un tri'angulo rect'angulo,
entonces $c^{2}=a^{2}+b^{2}$
(Teorema de Pit'agoras).
\end{example}

\begin{example}
\TeX{} se pronuncia como
 $\tau\epsilon\chi$.\\[6pt]
100~m$^{2}$ de 'area 'util \\[6pt]
De mi $\heartsuit$.
\end{example}

Las f'ormulas matem'aticas mayores o las ecuaciones quedan mejor en
renglones separados del texto. Para ello se ponen entre \ci{[} y
\ci{]} o entre \verb|\begin{|\ei{displaymath}\verb|}| y
  \verb|\end{displaymath}|. Esto produce f'ormulas sin n'umero de
ecuaci'on. Si desea que \LaTeX{} las enumere, puede emplear en entorno
\ei{equation}.

\begin{example}
Siendo $a$ y $b$ los catetos
y $c$ la hip'otenusa
de un tri'angulo rect'angulo,
entonces
\begin{displaymath}
c = \sqrt{  a^{2}+b^{2}  }
\end{displaymath}
(Teorema de Pit'agoras).
\end{example}

Con \ci{label} y \ci{ref} se puede hacer referencia a una
ecuaci'on del documento.

\begin{example}
\begin{equation} \label{eq:eps}
\epsilon > 0
\end{equation}
De (\ref{eq:eps}) se deduce\ldots
\end{example}

Observe que las expresiones se componen con un estilo diferente al
disponerlas en p'arrafos separados del texto:

\begin{example}
$\lim_{n \to \infty}
\sum_{k=1}^n \frac{1}{k^2}
= \frac{\pi^2}{6}$
\end{example}
\begin{example}
\begin{displaymath}
\lim_{n \to \infty}
\sum_{k=1}^n \frac{1}{k^2}
= \frac{\pi^2}{6}
\end{displaymath}
\end{example}


Existen diferencias entre el \emph{modo matem'atico} y el \emph{modo
  texto}. Por ejemplo, en el \emph{modo matem'atico}:

\begin{enumerate}

\item Los espacios en blanco y los cambios de l'inea no tienen ning'un
  significado. Todos los espacios se determinar'an a partir de la
  l'ogica de la expresi'on matem'atica o se deben indicar con
  instrucciones especiales como \ci{,}, \ci{quad}, \ci{qquad}, \ci{:},
  \ci{;}, \verb*|\ | y \verb|\!|\cih{"!}.%
\index{instrucciones!\quad@\verb*".\ ".}%
\index{\quad@\hspace*{-1.2ex}\verb*".\ ".}%

\begin{example}
\begin{equation}
\forall x \in \mathbf{R}:
\qquad x^{2} \geq 0
\end{equation}
\end{example}
 
\item Los renglones en blanco est'an prohibidos. S'olo puede haber un
  p'arrafo por f'ormula.
\item Cada letra en particular ser'a tenida en cuenta como el nombre
  de una variable y se pondr'a como tal (cursiva con espacios
  adicionales). Para introducir texto normal dentro de un texto
  matem'atico (con escritura en redondilla y con espacios entre
  palabras) debe incluirse dentro de la orden \verb|\textrm{...}|.

\begin{example}
\begin{equation}
x^{2} \geq 0\qquad
\textrm{para todo }x\in\mathbf{R}
\end{equation}
\end{example}
 
\end{enumerate}

%
% A~nada el paquete AMSSYB ... Blackboard bold .... R para n'umeros
% reales
%

Los matem'aticos pueden ser muy exigentes con los s'imbolos que se
emplean: aqu'i ser'ia m'as convencional emplear `\emph{blackboard
  bold}' \index{blackboard bold@\emph{blackboad bold}} \index{simbolos
  en negrita@s'imbolos en negrita} que se obtienen con \ci{mathbb} del
paquete \pai{amsfonts} o \pai{amssymb}.  \ifx\mathbb\undefined\else El
'ultimo ejemplo se convierte en
\begin{example}
\begin{displaymath}
x^{2} \geq 0\qquad
\textrm{para todo }x\in\mathbb{R}
\end{displaymath}
\end{example}
\fi

\section{Agrupaciones en modo matem'atico}

En modo matem'atico la mayor'ia de las instrucciones s'olo afecta al
car'acter siguiente. Si desea que una instrucci'on influya sobre
varios caracteres, entonces debe agruparlos empleando llaves
(\verb|{...}|).

\begin{example}
\begin{equation}
a^x+y \neq a^{x+y}
\end{equation}
\end{example}
 
\section{Elementos de las f'ormulas matem'aticas}

En este apartado se describen las instrucciones m'as importantes que
se utilizan en las f'ormulas matem'aticas. En el
apartado~\ref{symbols} de la p'agina~\pageref{symbols} podr'a
encontrar una lista de todos los s'imbolos disponibles.


\textbf{Las \wi{letras griegas} min'usculas} se introducen como
\verb|\alpha|, \verb|\beta|, \verb|\gamma|\ldots, y las
may'usculas\footnote{No hay definida ninguna Alfa may'uscula en
  \LaTeXe{} porque tiene el mismo aspecto que la redondilla A. Una vez
  que se haga la nueva codificaci'on matem'atica, esto cambiar'a.} se
introducen como \verb|\Gamma|, \verb|\Delta|\ldots

\begin{example}
$\lambda,\xi,\pi,\mu,\Phi,\Omega$
\end{example}

\index{exponente}\index{sub'indice}%
\textbf{Los exponentes y los sub'indices} se pueden indicar empleando
el car'acter \verb|^|\index{^@\verb"|^"|} y el car'acter
\verb|_|\index{_@\verb"|_"|}.

\begin{example}
$a_{1}$ \qquad $x^{2}$ \qquad
$e^{-\alpha t}$ \qquad
$a^{3}_{ij}$\\
$e^{x^2} \neq {e^x}^2$
\end{example}

El \textbf{\wi{signo de ra'iz cuadrada}} se introduce con \ci{sqrt}, y
la ra'iz \mbox{$n$-'esima} con \verb|\sqrt[|$n$\verb|]|. \LaTeX\ elige
autom'aticamente el tama'no del signo de ra'iz. Si s'olo necesita el
signo de la ra'iz emplee \verb|\surd|.

\begin{example}
$\sqrt{x}$ \qquad 
$\sqrt{ x^{2}+\sqrt{y} }$ 
\qquad $\sqrt[3]{2}$\\[3pt]
$\surd[x^2 + y^2]$
\end{example}

Las instrucciones \ci{overline} y \ci{underline} producen
\textbf{l'ineas horizontales} directamente encima o debajo de una
expresi'on.
\index{linea@l'inea!horizontal}
\begin{example}
$\overline{m+n}$
\end{example}

Las 'ordenes \ci{overbrace} y \ci{underbrace} crean \textbf{llaves
  horizontales} largas encima o bien debajo de una expresi'on.
\index{llave!horizontal}
\begin{example}
$\underbrace{ a+b+\cdots+z }_{26}$
\end{example}

\index{acentos!matem'aticos}Para poner acentos matem'aticos, como
peque~nas flechas o \wi{tilde}s a las variables, se pueden utilizar
las 'ordenes que aparecen en la tabla~\ref{mathacc}. Los 'angulos y
tildes que abarcan varios caracteres se obtienen con \ci{widetilde} y
\ci{widehat}. Con el s'imbolo \verb|'|\index{'@\verb"|'"|} se
introduce el signo de \wi{prima}.
% un gui'on es --

\begin{example}
\begin{displaymath}
y=x^{2}\qquad y'=2x\qquad y''=2
\end{displaymath}
\end{example}

Con frecuencia los \textbf{\wi{vectores}} se indican a~nadi'endoles
\wi{s'imbolos de flecha} peque~nos encima de la variable. Esto se
realiza con la orden \ci{vec}. Para designar al vector que va desde
$A$ hasta $B$ resultan adecuadas las instrucciones \ci{overrightarrow}
y \ci{overleftarrow}.

\begin{example}
\begin{displaymath}
\vec a\quad\overrightarrow{AB}
\end{displaymath}
\end{example}


Existen funciones matem'aticas (seno, coseno, tangente,
logaritmos\ldots) que se presentan con redondilla y \emph{nunca} en
it'alica. Para 'estas \LaTeX{} proporciona las siguientes
instrucciones: \index{funciones!matem'aticas}

\begin{verbatim}
\arccos   \cos    \csc   \exp   \ker     \limsup  \min   \sinh
\arcsin   \cosh   \deg   \gcd   \lg      \ln      \Pr    \sup
\arctan   \cot    \det   \hom   \lim     \log     \sec   \tan
\arg      \coth   \dim   \inf   \liminf  \max     \sin   \tanh
\end{verbatim}

\begin{example}
\[\lim_{n \rightarrow 0}
\frac{\sin x}{x}=1\]
\end{example}

Para la \wi{funci'on m'odulo} existen dos 'ordenes distintas:
\ci{bmod} para el operador binario, como en ``$a \bmod b$'', y
\ci{pmod} para expresiones como ``$x\equiv a \pmod{b}$''.

Un \textbf{\wi{quebrado}} o
\textbf{fracci'on}\index{fraccion@fracci'on} se pone con la orden
\ci{frac}\verb|{...}{...}|. Para los quebrados sencillos a veces
suele ser preferible utilizar el operador \verb|/|, %\index{/@\verb|/|}
como en $1/2$.
\begin{example}
$1\frac{1}{2}$~horas
\begin{displaymath}
\frac{ x^{2} }{ k+1 }\qquad
x^{ \frac{2}{k+1} }\qquad
x^{ 1/2 }
\end{displaymath}
\end{example}

Los \textbf{\wi{coeficientes de los binomios}} y estructuras similares
se pueden componer con la instrucci'on \verb|{... |\ci{choose}%
\verb| ...}| o \verb|{... |\ci{atop}\verb| ...}|. Con la segunda orden se
consigue lo mismo pero sin par'entesis.

\begin{example}
\begin{displaymath}
{n \choose k}\qquad {x \atop y+2}
\end{displaymath}
\end{example}
 
\medskip

El \textbf{\wi{signo de integral}} se obtiene con \ci{int} y el
\textbf{\wi{signo de sumatorio}} con \ci{sum}. Los l'imites superior e
inferior se indican con~\verb|^| y~\verb|_|, como se hace para los
super'indices y sub'indices.

\begin{example}
\begin{displaymath}
\sum_{i=1}^{n} \qquad
\int_{0}^{\frac{\pi}{2}} \qquad
\end{displaymath}
\end{example}

Para las \textbf{\wi{llaves}} y otros \wi{delimitadores} tenemos todos
los tipos de s'imbolos de \TeX{}
(p.~ej.~$[\;\langle\;\|\;\updownarrow$).  Los par'entesis y los
corchetes se introducen con las teclas correspondientes, las llaves
con \verb|\{| y \verb|\}|, y el resto con instrucciones especiales
(p.~ej.~\verb|\updownarrow|). En la tabla~\ref{tab:delimiters} de la
p'ag.~\pageref{tab:delimiters} podr'a encontrar una lista de los
delimitadores disponibles.

\begin{example}
\begin{displaymath}
{a,b,c}\neq\{a,b,c\}
\end{displaymath}
\end{example}

Para que \LaTeX\ elija de modo autom'atico el tama'no apropiado se
pone la orden \ci{left} delante del delimitador de apertura y
\ci{right} delante del que cierra. Observe que debe cerrar cada
\ci{left} con el \ci{right} correspondiente. Si no desea nada en la
derecha, entonces emplee `\ci{right.}'.


\begin{example}
\begin{displaymath}
1 + \left( \frac{1}{ 1-x^{2} }
    \right) ^3
\end{displaymath}
\end{example}


En algunos casos es necesario fijar de modo expl'icito el tama~no
correcto del delimitador matem'atico\index{delimitador!matem'atico}.
Para esto se pueden utilizar las instrucciones \ci{big}, \ci{Big},
\ci{bigg} y \ci{Bigg} como prefijos de la mayor'ia de las 'ordenes de
delimitadores\footnote{Estas instrucciones pueden no funcionar del
  modo deseado si se ha utilizado una instrucci'on de cambio del
  tama~no del tipo, o si se ha especificado la opci'on \texttt{11pt} o
  \texttt{12pt}. Empl'eense los paquetes \pai{exscale} o \pai{amstex}
  para corregir esta anomal'ia.}.

\begin{example}
$\Big( (x+1) (x-1) \Big) ^{2}$\\
$\big(\Big(\bigg(\Bigg($\quad
$\big\}\Big\}\bigg\}\Bigg\}$\quad
$\big\|\Big\|\bigg\|\Bigg\|$
\end{example}

Para poner los \textbf{\wi{puntos suspensivos}} en una ecuaci'on existen
varias 'ordenes. \ci{ldots} coloca los puntos en la l'inea base y
\ci{cdots} los pone en la zona media del rengl'on. Ademas de 'estos,
tambi'en est'an las instrucciones \ci{vdots} para puntos verticales y
\ci{ddots} para puntos en diagonal.
\index{puntos suspensivos!verticales}%
\index{puntos suspensivos!en diagonal}%
\index{puntos suspensivos!horizontales}%
En el apartado \ref{sec:vert} podr'a encontrar otro ejemplo.

\begin{example}
\begin{displaymath}
x_{1},\ldots,x_{n} \qquad
x_{1}+\cdots+x_{n}
\end{displaymath}
\end{example}
 
\section{Espaciado en modo matem'atico}

\index{espaciado en modo matematico@espaciado en modo matem'atico}
Si no est'a satisfecho con los espaciados que \TeX{} elige dentro de
una f'ormula, 'estos se pueden alterar con instrucciones especiales.
Las m'as importantes son \ci{,} para un espacio muy peque~no, %
\verb*.\ .  para una mediana (\verb*. . significa un car'acter en
blanco), \ci{quad} y \ci{qquad} para espaciados grandes y
\verb|\!|\cih{"!}  para la disminuci'on de una separaci'on.

\begin{example}
\newcommand{\rd}{\mathrm{d}}
\begin{displaymath}
\int\!\!\!\int_{D} g(x,y)
  \, \rd x\, \rd y
\end{displaymath}
en lugar de
\begin{displaymath}
\int\int_{D} g(x,y)\rd x \rd y
\end{displaymath}
\end{example}
Observe que la `d' en la diferencial se compone de modo convencional
en redondilla\footnote{En este ejemplo la `d' en redondilla se ha
  introducido a trav'es de la orden \texttt{\bs rd}, que previamente
  se ha definido con \texttt{\bs newcommand\{\bs rd\}\{\bs mathrm\{d\}\}}.
  De esta forma se evita estar introduciendo la secuencia \texttt{\bs
    mathrm\{d\}} repetidamente.}.
 
\section{Colocaci'on de signos encima de otros}
\label{sec:vert}

Para componer \textbf{matrices} y similares se tiene el entorno
\ei{array}. 'Este funciona de modo similar al entorno \texttt{tabular}.
Para dividir los renglones se utiliza la instrucci'on \verb|\\|.

\begin{example}
\begin{displaymath}
\mathbf{X} =
\left( \begin{array}{ccc}
x_{11} & x_{12} & \ldots \\
x_{21} & x_{22} & \ldots \\
\vdots & \vdots & \ddots
\end{array} \right)
\end{displaymath}
\end{example}

Tambi'en se puede usar el entorno \ei{array} para componer expresiones
de funciones que tienen ``\verb|.|'' como delimitador invisible
derecho, o sea, \ci{right}\verb|.|.

\begin{example}
\begin{displaymath}
y = \left\{ \begin{array}{ll}
 a & \textrm{si $d>c$}\\
 b+x & \textrm{por la ma~nana}\\
 l & \textrm{el resto del d'ia}
  \end{array} \right.
\end{displaymath}
\end{example}


Para las ecuaciones que ocupen varios renglones o para los sistemas de
ecuaciones \index{sistema de ecuaciones} se pueden emplear los
entornos \ei{eqnarray} y \verb|eqnarray*|. En \texttt{eqnarray} cada
rengl'on contiene un n'umero de ecuaci'on. Con \verb|eqnarray*| no se
produce ninguna numeraci'on.

Los entornos \texttt{eqnarray} y \verb|eqnarray*| funcionan como una
tabla de 3 columnas con la disposici'on \verb|{rcl}|, donde la columna
central se utiliza para el signo de igualdad, desigualdad o cualquier
otro signo que deba ir. La instrucci'on \verb|\\| divide los
renglones.

\begin{example}
\begin{eqnarray}
f(x) & = & \cos x       \\
f'(x) & = & -\sin x     \\
\int_{0}^{x} f(y) \mathrm{d}y &
 = & \sin x
\end{eqnarray}
\end{example}

\noindent Observe que existe demasiado espacio a cada lado de la
columna central, donde se encuentran los signos. Para reducir estas
separaciones se puede emplear \verb|\setlength\arraycolsep{2pt}| como
en el ejemplo siguiente.

\index{ecuaciones largas} Las \textbf{ecuaciones largas} no se dividen
autom'aticamente. Es el autor quien debe determinar en qu'e lugares se
deben fraccionar y cu'anto se debe sangrar. Los dos m'etodos
siguientes son las variantes m'as utilizadas para esto.

\begin{example}
{\setlength\arraycolsep{2pt}
\begin{eqnarray}
\sin x & = & x -\frac{x^{3}}{3!}
     +\frac{x^{5}}{5!}-{}
                    \nonumber\\
 & & {}-\frac{x^{7}}{7!}+{}\cdots
\end{eqnarray}}
\end{example}
\pagebreak[1]

\begin{example}
\begin{eqnarray}
\lefteqn{ \cos x = 1
     -\frac{x^{2}}{2!} +{} }
                    \nonumber\\
 & & {}+\frac{x^{4}}{4!}
     -\frac{x^{6}}{6!}+{}\cdots
\end{eqnarray}
\end{example}

\enlargethispage{\baselineskip}
La instrucci'on \ci{nonumber} impide que \LaTeX{} coloque un n'umero
para la ecuaci'on en la que est'a colocada la orden.


\section{Tama~no del tipo para ecuaciones}

\index{tamano del tipo@tama~no del tipo!para ecuaciones} En el modo
matem'atico \TeX{} selecciona el tama~no del tipo seg'un el contexto.
Los super'indices, por ejemplo, se ponen en un tipo m'as peque~no. Si
quiere introducir un texto en redondilla en una ecuaci'on y utiliza la
instrucci'on \verb|\textrm|, el mecanismo de cambio del tama~no del
tipo no funcionar'a, ya que \verb|\textrm| conmuta de modo temporal al
modo de texto. Entonces se debe emplear \verb|\mathrm| para que se
mantenga activo el mecanismo de cambio de tama~no. Pero preste
atenci'on, ya que \ci{mathrm} s'olo funcionar'a bien con cosas
peque~nas. Los espacios no son a'un activos y los caracteres con
acentos no funcionan\footnote{El paquete AMS-\LaTeX{} hace que la
  orden \ci{textrm} funcione bien con el cambio de tama~nos.}.

\begin{example}
\begin{equation}
2^\textrm{o} \quad 
2^\mathrm{o}
\end{equation}
\end{example}

Sin embargo, a veces es preciso indicarle a \LaTeX{} el tama~no del
tipo correcto. En modo matem'atico el tama~no del tipo se fija con las
cuatro instrucciones:
\begin{flushleft}
\ci{displaystyle}~($\displaystyle 123$),
\ci{textstyle}~($\textstyle 123$), 
\ci{scriptstyle}~($\scriptstyle 123$) y
\ci{scriptscriptstyle}~($\scriptscriptstyle 123$).
\end{flushleft}

El cambio de estilos tambi'en afecta al modo de presentar los
l'imites.

\begin{example}
\begin{displaymath}
\mathrm{corr}(X,Y)= 
 \frac{\displaystyle 
   \sum_{i=1}^n(x_i-\bar x)
   (y_i-\bar y)} 
  {\displaystyle\sqrt{
 \sum_{i=1}^n(x_i-\bar x)^2
\sum_{i=1}^n(y_i-\bar y)^2}}
\end{displaymath}    
\end{example}
% Esto no es un acento matem'atico, y ning'un libro de Matem'aticas lo
% compondr'ia de este modo.
% mathop produce el espaciado correcto.
 
\noindent 'Este es uno de los ejemplos en los que se necesitan
corchetes mayores que los normalizados que proporciona %
\verb|\left[| y \verb|\right]|.


\section{Descripci'on de variables}

Para algunas de sus ecuaciones Vd.\ podr'ia querer a~nadir una
secci'on donde se describan las variables utilizadas. El siguiente
ejemplo le podr'ia ser de ayuda para esto:

\begin{example}
\begin{displaymath}
a^2+b^2=c^2
\end{displaymath}
{\settowidth{\parindent}
   {donde:\ }

\makebox[0pt][r]
 {donde:\ }$a$, $b$ son  
los adjuntos del 'angulo recto
de un tri'angulo rect'angulo.

$c$ es la hipotenusa
del tri'angulo}
\end{example}

Si necesita componer a menudo segmentos de texto como 'este, ahora es
el momento id'oneo para practicar la instrucci'on
\verb|\newenvironment|. Empl'eela para crear un entorno especializado
para describir variables.\index{descripci'on de variables} Revise la
descripci'on al final del cap'itulo anterior.

\section{Teoremas, leyes\ldots}

Cuando se escriben documentos matem'aticos, probablemente precise de
un modo para componer ``lemas'', ``definiciones'', ``axiomas'' y
estructuras similares. \LaTeX{} facilita esto con la orden
\begin{command}
\ci{newtheorem}\verb|{|\emph{nombre}\verb|}[|\emph{contador}\verb|]{|%
         \emph{texto}\verb|}[|\emph{secci'on}\verb|]|
\end{command}
El argumento \emph{nombre} es una palabra clave corta que se utiliza
para identificar el ``teorema''. Con el argumento \emph{texto} se
define el nombre del ``teorema'' que aparecer'a en el documento final.

Los argumentos entre corchetes son opcionales. Ambos se emplean para
especificar la numeraci'on utilizada para el ``teorema''. Con el
argumento \emph{contador} se puede especificar el \emph{nombre} de un
``teorema'' declarado previamente. El nuevo ``teorema'' se numerar'a
con la misma secuencia. El argumento \emph{secci'on} le permite
indicar la unidad de secci'on con la que desea numerar su ``teorema''.

Tras ejecutar la instrucci'on \ci{newtheorem} en el pre'ambulo de su
documento, dentro del texto se puede usar la instrucci'on siguiente:


\begin{code}
\verb|\begin{|\emph{nombre}\verb|}[|\emph{texto}\verb|]|\\
Este es un teorema interesante\\
\verb|\end{|\emph{nombre}\verb|}|     
\end{code}

He aqu'i otro ejemplo de las posibilidades de este entorno:

\begin{example}
% Definiciones para el documento.
% Pre'ambulo
\newtheorem{ley}{Ley}
\newtheorem{jurado}[ley]{Jurado}
% En el documento
\begin{ley} \label{law:box}
No se esconda en la caja testigo
\end{ley}
\begin{jurado}[Los doce]
Podr'ia ser Vd. Por tanto, tenga
cuidado y vea la ley
\ref{law:box}\end{jurado}
\begin{ley}No, No, No\end{ley}
\end{example}

El teorema ``Jurado'' emplea el mismo contador que el teorema ``Ley''.
Por ello, toma un n'umero que est'a en secuencia con las otras
``Leyes''. El argumento que est'a entre corchetes se utiliza para
especificar un t'itulo o algo parecido para el teorema.

\begin{example}
\newtheorem{mur}{Ley de Murphy}[section]
\begin{mur} Si algo puede ir mal,
ir'a mal.
\end{mur}
\end{example}

El teorema ``Ley de Murphy'' obtiene un n'umero que est'a ligado con
el apartado actual. Tambi'en se podr'ia utilizar otra unidad, como,
por ejemplo, un cap'itulo o un subapartado.

\section{S'imbolos en negrita}
\index{ssssimbolos en negrita@s'imbolos en negrita}

Es bastante dif'icil obtener s'imbolos en negrita en \LaTeX\@.
Probablemente esto sea intencionado ya que los compositores de texto
aficionados tienden a abusar de ellos. La orden de cambio de tipo
\verb|\mathbf| produce letras en negrita, pero estas son redondillas
mientra que los s'imbolos matem'aticos normalmente van en versalita.
Existe una orden \ci{boldmath}, pero \emph{'esta s'olo se puede
  emplear fuera del modo matem'atico}. Tambi'en funciona con los
s'imbolos.

\begin{example}
\begin{displaymath}
\mu, M \qquad \mathbf{M} \qquad
\mbox{\boldmath $\mu, M$}
\end{displaymath}
\end{example}

\noindent
Observe que la coma tambi'en est'a en negrita, lo cual puede que no
se precise.

El paquete \pai{amsbsy} (incluido por \pai{amsmath}) hace
esto mucho m'as f'acil. Incluye una orden \ci{boldsymbol} y una
``negrita del hombre pobre'' \ci{pmb} (``\emph{poor man's bold}''),
que opera de forma an'aloga a las m'aquinas de escribir, que para
poner un texto en negrita se escribe encima del texto ya escrito.

\ifx\boldsymbol\undefined\else
\begin{example}
\begin{displaymath}
\mu, M \qquad
\boldsymbol{\mu}, \boldsymbol{M}
\qquad \pmb{\mu}, \pmb{M}
\end{displaymath}
\end{example}
\fi


\endinput

