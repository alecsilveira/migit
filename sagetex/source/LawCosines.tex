\documentclass[12pt]{article}
\usepackage{graphicx}
\usepackage{latexsym, amsmath,amsfonts,amssymb}
\usepackage{xcolor}
\usepackage[margin=.75in]{geometry}
\usepackage{kpfonts}  %Changing the default fonts
\usepackage[T1]{fontenc}
\setlength{\parskip}{1.2ex} %space between paragraphs
\setlength{\parindent}{1em} %Paragraph indentation
\clubpenalty = 10000
\widowpenalty = 10000
\newcommand\T{\rule{0pt}{2ex}} % \T will create extra space above (used to fix tables)
\newcommand\B{\rule[-1.5ex]{0pt}{0pt}}% \B will create extra space below
\linespread{1.25} %spacing between lines
\pagestyle{empty} %remove page numbers
\begin{document}
\begin{center}
\Large{\textbf{Law of Cosines Proof}}
\end{center}

\noindent {\bf Law of Cosines:} For $\triangle ABC$ with sides $a, b, c$:
\begin{center}
$a^2=b^2+c^2-2bc\cos(A)$\\
$b^2=a^2+c^2-2ac\cos(B)$\\
$c^2=a^2+b^2-2ab\cos(C)$
\end{center}
{\bf Proof:} There are two cases to consider depending on whether the triangle
is acute or obtuse.\\\\
Case $1$: $\triangle ABC$ is acute.\\\\
Arrange the triangle so that the base is $\overline{AB}$ and drop a perpendicular from point $C$ to $\overline{AB}$ to create point $D$ as shown in the following diagram:
\begin{figure}[h]
\begin{center}
\includegraphics{AcuteTri.pdf}
\caption{Proving the Law of Cosines with an acute triangle.} \label{fg:lofs}
\end{center}
\end{figure}

\noindent From $\triangle ADC$, $\cos(A)=\frac{AD}{b}$ which
is equivalent to $AD=b\cos(A)$. Applying the Pythagorean Theorem tells 
us $b^2=(h_1)^2+(AD)^2$ so that $(h_1)^2=b^2-(AD)^2$. Since $\triangle DCB$ is also a right triangle, applying the Pythagorean Theorem gives
$a^2=(h_1)^2+(DB)^2$ and then substituting in for $(h_1)^2$ gives
$a^2=\left[b^2-(AD)^2\right]+(DB)^2$. Now $DB=c-AD$ so 
$a^2=b^2-(AD)^2+(c-AD)^2$. A little algebra gives 
$a^2=b^2-(AD)^2+c^2-2c(AD)+(AD)^2$ which is equivalent to 
$a^2=b^2+c^2-2c(AD)$. Since $AD=b\cos(A)$ this becomes
$a^2=b^2+c^2-2bc\cos(A)$. In similar fashion it can be shown that
$b^2=a^2+c^2-2ac\cos(B)$ and $c^2=a^2+b^2-2ab\cos(C)$.
\newpage
\noindent Case $2$: $\triangle ABC$ is obtuse.\\\\
If the triangle is arranged so that the base is the side opposite the obtuse angle, $C$ then dropping a perpendicular to the other side as in the following diagram:
\begin{figure}[h]
\begin{center}
\includegraphics {ObtuseTri.pdf}
\caption{Proving the Law of Cosines with an obtuse triangle.} \label{fg:lofs2}
\end{center}
\end{figure}

\noindent This will be solved as in Case 1 so the only case that needs to be checked is if the triangle is rotated so that $\overline{AC}$ is the base. Drop a perpendicular from $B$ to $\overline{AC}$ to create point $D$ as shown:
\begin{figure}[h]
\begin{center}
\includegraphics {ObtuseTriB.pdf}
\end{center}
\end{figure}

\noindent Since $C$ is an obtuse angle $\cos(C)=-\cos(180^{\circ}-C)=-\frac{CD}{a}$; that is, $CD=-a\cos(C)$. Applying the Pythagorean
Theorem to $\triangle BCD$ gives $a^2=(h_2)^2+(CD)^2$ or 
$(h_2)=a^2-(CD)^2$. For $\triangle ABD$ the Pythagorean Theorem gives $c^2=(h_2)^2+(CD+b)^2$ hence $c^2=\left[a^2-(CD)^2\right]+(CD+b)^2$. This is equivalent to 
$c^2=a^2-(CD)^2+(CD)^2+2(CD)b+b^2$ which simplifies to
$c^2=a^2+2(CD)b+b^2$. Substituting $CD=-a\cos(C)$ and rearranging
the terms gives $c^2=a^2+b^2-2ab\cos(C)$.

\noindent In both cases, the Law of Cosines is established.

\end{document}