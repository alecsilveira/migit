% Copyright 2015 http://www.highschoolmathandchess.com/
% This program is free software: you can redistribute it and/or modify
%   it under the terms of the GNU General Public License as published by
%    the Free Software Foundation, either version 3 of the License, or
%    (at your option) any later version.
%    This program is distributed in the hope that it will be useful,
%   but WITHOUT ANY WARRANTY; without even the implied warranty of
%   MERCHANTABILITY or FITNESS FOR A PARTICULAR PURPOSE.  See the
%    GNU General Public License for more details.
%
%    You should have received a copy of the GNU General Public License
%    along with this program.  If not, see <http://www.gnu.org/licenses/>.

\documentclass[11pt]{article}
%\usepackage{fullpage}
\usepackage{xcolor}
\usepackage[colorlinks=true,urlcolor=blue]{hyperref}
\usepackage{latexsym,amsmath,amsfonts,amsthm,amssymb}
\usepackage{tikz}
\usepackage[most]{tcolorbox}
\tcbuselibrary{skins,listings}
\newtheorem*{cor}{Corollary}
\newtheorem*{theorem}{Theorem}
\newtheorem*{lem}{Lemma}
\pagestyle{empty}
\usetikzlibrary{shapes}
\usetikzlibrary{calc,trees,positioning,arrows,chains,shapes.geometric,%
  decorations.pathreplacing,decorations.pathmorphing,shapes,%
  matrix,shapes.symbols,plotmarks,decorations.markings,shadows}
\RequirePackage[framemethod=tikz]{mdframed}
\newmdenv[skipabove=7pt,
skipbelow=7pt,
roundcorner=10pt,
shadow=true,
shadowsize=6pt,
rightline=true,
leftline=true,
topline=true,
bottomline=true,
backgroundcolor=green!10,
outerlinewidth=.35pt,
outerlinecolor=black,
linecolor=black,
innerleftmargin=5pt,
innerrightmargin=5pt,
innertopmargin=5pt,
innerbottommargin=5pt,
leftmargin=0cm,
rightmargin=0cm,
linewidth=1pt]{hBox}

\newmdenv[skipabove=7pt,
skipbelow=7pt,
roundcorner=10pt,
shadow=true,
shadowsize=6pt,
rightline=true,
leftline=true,
topline=true,
bottomline=true,
backgroundcolor=red!20,
outerlinewidth=.35pt,
outerlinecolor=black,
linecolor=black,
innerleftmargin=5pt,
innerrightmargin=5pt,
innertopmargin=5pt,
innerbottommargin=5pt,
leftmargin=0cm,
rightmargin=0cm,
linewidth=1pt]{wBox}
\newenvironment{highlight}{\begin{hBox}}{\hfill{\color{green!20}}\end{hBox}}
\begin{document}
\begin{center}
\Huge{Negative Times Negative is Positive}
\large{\href{www.highscoolmathandchess.com}{http://www.highschoolmathandchess.com/}}
\end{center}
At some point in your study of mathematics you should have learned:\\
``Negative times negative is positive.'' as well as
``Negative times positive is negative.'' and
``Positive times negative is negative.''\\\\
Unfortunately, most students don't understand why it's true. Let's see how 
those statements follow logically from the properties of real numbers.

\begin{tcolorbox}[skin=beamer,beamer,colback=white,colframe=red!55!black,adjusted title=Properties of Real Numbers]
Let $a, b, c$ be real numbers 
\begin{enumerate}
\item Closure of Addition: $a+b$ is a real number
\item Closure of Multiplication: $ab$ is a real number
\item Commutative Property of Addition: $a+b=b+a$
\item Commutative Property of Multiplication: $ab=ba$
\item Associative Property of Addition: $a+(b+c)=(a+b)+c$
\item Associative Property of Multiplication: $a(b+c)=(a+b)c$
\item Additive Identity Property: $a+0=a$
\item Multiplicative Identity Property: $a(1)=a$
\item Additive Inverse Property: $a+(-a)=0$
\item Multiplicative Inverse Property: $a\left(\frac{1}{a}\right)=1$
\item Distributive Property: $a(b+c)=ab+ac$
\end{enumerate}
\end{tcolorbox}

Mathematicians can use deductive reasoning to discover new facts 
from the older facts. Usually those facts are called 
axioms, but in this case, the facts are the Properties of Real Numbers. 
Let's see how a negative number times a negative number follows logically 
from the rules of real numbers.

\begin{tcolorbox}[skin=beamer,beamer,colback=white,colframe=orange!55!black,adjusted title=Theorem]
\begin{theorem}
A negative number multiplied by a positive number is a negative number.
\end{theorem}
\end{tcolorbox}
\noindent\textbf{Proof:} Let $a, b, c$ be positive real numbers. First notice
that the Distributive Property says $a(b+c)=ab+ac$ and, by the Commutative
Property fo Multiplication this is $ba+ca$. Moreover, the Commutative Property
of Multiplication says $a(b+c)=(b+c)a$ so combining these gives us 
$(b+c)a = ba+ca$. With this background, it follows that $(-a)b+ab$ is equal 
to $(-a+a)b$. Now the Additive Inverses Property says this is equivalent to 
$(0)a$ which is equal to $0$. Combining all these gives $(-a)b +ab = 0$
and so $(-a)b = -(ab)$. Since $a, b, c$ are positive real numbers we have 
$-a$, a negative number, multiplied by $b$, a positive number results in 
$-(ab)$, a negative number. $\Box$

\begin{highlight}
The multiplication of numbers is commutative so it follows that 
a positive number multiplied by a negative number must be negative.
\end{highlight}

\begin{tcolorbox}[skin=beamer,beamer,colback=white,colframe=orange!55!black,adjusted title=Theorem]
\begin{theorem}
A negative number multiplied by a negative number is a positive number.
\end{theorem}
\end{tcolorbox}
\noindent\textbf{Proof:} The expression $(-a)(-b)+[-(ab)]$ is equal to 
$(-a)(b)+(-a)(b)$ since, by the previous result, a negative number times a 
positive number is a negative number. The Distributive Property tells
us that this is equivalent to $-a[(-b)+b]=-a(0)=0$. Putting all the steps 
together we have $(-a)(-b)+[-(ab)]=0$ or just $(-a)(-b)=(ab)$. That is, 
a negative number times a negative number is a positive number. $\Box$\\

Proof is the essence of mathematics and the Properties of Real Numbers 
provide us with some simple proofs that we can understand. And now you know 
\emph{why} negative times
negative must be positive.
\end{document}

