\documentclass[12pt]{article}%
\usepackage{latexsym,amsmath,amsfonts,amssymb}
\usepackage{graphicx}%insert pictures
\usepackage{geometry}%setting margins
\usepackage{fancybox}
\usepackage{xcolor}
\usepackage[colorinlistoftodos,shadow]{todonotes}
\usepackage{tikz}
\pagestyle{empty}
\linespread{1.6}% double spacing to give room for the binomial coefficients
%%%%%%%%%%%%%%%%%%%%%%%%%%%%%%%%% NOTATION %%%%%%%%%%%%%%%%%%%%%%%%
\newcommand{\FPC}{Fundamental Principle of Counting }
%%%%%%%%%%%%%%%%%%%%%% END OF PREAMBLE %%%%%%%%%%%%%%%%%%%%%%%%%%%%%%
\begin{document}
\noindent A \textbf{full house} refers to a hand of five cards that consists 
of three of a kind and one pair. Suppose $5$ cards are dealt from a standard deck 
of $52$ cards. Find
$P(\mbox{full house})$.\\\\
\textbf{Solution:} Since order is not important and repetition of 
cards is not allowed, this is a problem that involves combinations and
the denominator is $\binom{52}{5}$. A full house is $3$ of a kind and
$1$ pair. Use the \FPC to determine the numerator. 
\begin{center}
\begin{tabular}{lcc}
Event & Outcomes & Example\\
\hline
$E_1:$ Choose the kind for $3$ of a kind & $\binom{13}{1}$ & $8$ \\
$E_2:$ Choose the specific cards for $3$ of a kind & $\binom{4}{3}$ & 
$8\,\clubsuit$, $8\,\heartsuit$, $8\,\diamondsuit$\\
$E_3:$ Choose the kind for the pair & $\binom{12}{1}$ & 
$8\,\clubsuit$, $8\,\heartsuit$, $8\,\diamondsuit$, $7$\\
$E_4:$ Choose the specific cards & $\binom{4}{2}$ & 
$8\,\clubsuit$, $8\,\heartsuit$, $8\,\diamondsuit$, $7\,\diamondsuit$,
$7\,\spadesuit$\\
\end{tabular}
\end{center}
By the \FPC there $\binom{13}{1}\binom{4}{3}\binom{12}{1}\binom{4}{2}$ different
ways to choose a full house. Therefore, $P(\mbox{full house})=\frac{\binom{13}{1}\binom{4}{3}\binom{12}{1}\binom{4}{2}}{\binom{52}{5}}=\frac{3744}{2598960} \approx 0.0014405762$.\\\\
\end{document}
