\documentclass[12pt]{article}
\usepackage{latexsym, amsmath,amsfonts,amssymb}
\usepackage[margin=.75in]{geometry}
\usepackage{kpfonts}  %Changing the default fonts
\usepackage[T1]{fontenc}
\setlength{\parskip}{1.2ex} %space between paragraphs
\setlength{\parindent}{1em} %Paragraph indentation
\linespread{1.25} %spacing between lines
\newtheorem{Theorem}{Theorem}
\pagestyle{empty}
\begin{document}
\begin{center}
\Large{\textbf{The Real Numbers are Uncountable}}
\end{center}
\textbf{Theorem:} $[0,1]$ is uncountable.\\\\
\textbf{"Proof":} If $[0,1]$ is countable then the 
real numbers can be listed:\\\\
$r_1=.r_{1,1}r_{1,2}r_{1,3}r_{1,4}r_{1,5}\,\ldots$\\
$r_2=.r_{2,1}r_{2,2}r_{2,3}r_{2,4}r_{2,5}\,\ldots$\\
$r_3=.r_{3,1}r_{3,2}r_{3,3}r_{3,4}r_{3,5}\,\ldots$\\
$r_4=.r_{4,1}r_{4,2}r_{4,3}r_{4,4}r_{4,5}\,\ldots$\\
$r_5=.r_{5,1}r_{5,2}r_{5,3}r_{5,4}r_{5,5}\,\ldots$\\\\
and so on, where $r_{i,j}$ is the jth digit of the ith real 
number. Consider the real number $x$ where the nth digit 
is $1$ if  $r_{i,i} \neq 1$ and $2$ if $r_{i,i} = 1$.
Since $x \in [0,1]$, it must be one of the $r_i$ listed above 
but from the way $x$ is defined the nth digit of $x$ is 
different from $r_{n,n}$. Therefore $x$ does not appear on 
the list. This contradiction implies that $[0,1]$ is 
uncountable.

You've noticed, I hope that I typed "proof" because this 
isn't really a proof. That's because it ignores the fact that the 
decimal representation of two real numbers isn't unique. 
Including that into the argument with any decent standard 
of rigor is going to make the proof more complicated; 
for example, it forces you think about numbers as the limit 
of infinite sums which puts the proof out of the reach of most 
high school students. Better to tell them that 
mathematicians have determined that the only way 
two different decimals can represent the same number is
if one decimal ends in a block of repeating nines. 

You'll need to make the "proof" more concrete, though, 
because it's still too abstract for a typical class.
So give them a list of real numbers in $[0,1]$:\\\\
$r_1=.10112\,\ldots$\\
$r_2=.23123\,\ldots$\\
$r_3=.34784\,\ldots$\\
$r_4=.95687\,\ldots$\\
$r_5=.45565\,\ldots$\\\\
and so on. There's no rhyme or reason why those as to 
what the actual digits are of each number and in this 
specific example $x=.21112\ldots$. It's clear that
$x \neq r_1$ since the first digit is different. Similarly, 
$x \neq r_2$ since the second digit is different. Likewise, 
$x \neq r_3$ since the third digit is different. Now it should
be a little clearer that the very way that $x$ is defined 
means the ith digit of $x$ is different from the ith digit 
of $r_i$ and $x$ doesn't end with a block of repeating nines.
From this it follows that $x$ cannot appear in the list and 
this contradiction means that the real numbers 
in $[0,1]$ are uncountable.

To get a feel for why two different decimal can represent
the same real number only if one ends with a repeating
block of nines, let $r_1$ and $r_2$ be two real numbers with 
different decimal representations. Then we can write 
$r_1=\lim_{n\to\infty} \sum_{i=0}^{n}\frac{a_i}{10^i}$
and $r_2=\lim_{n\to\infty}\sum_{i=0}^{n}{b_i}{10^i}$
and there exists a smallest integer $k$ such that 
$a_k\neq b_k$. Since, by hypothesis, $r_1=r_2$ we know 
$r_1-r_2=\lim_{n\to\infty}\sum_{i=0}^{n}\frac{a_i}{10^i}-
\lim_{n\to\infty}\sum_{i=0}^{n}\frac{b_i}{10^i}=
\lim_{n\to\infty}\sum_{i=0}^{n}\frac{a_i-b_i}{10^i}=0$
but after adding up the sums to the $k^{th}$ digit the 
numbers differ by $\left|\frac{a_k-b_k}{10^k}\right|\geq\frac{1}{10^k}$. Since the remaining sum 
$\lim_{n\to\infty}\sum_{i=k+1}^{n}\left|\frac{a_i-b_i}{10^i}\right|
\leq  \lim_{n\to\infty}\sum_{i=k+1}^{n}\frac{9}{10^i}=\frac{1}{10^k}$ the only way that two different decimals
represent the same number is if one decimal has repeating
block of nines. The construction in the "proof" above 
creates a decimal with no nines so it can't have appeared on
the list. Without this fact, $x$ not appearing on the list could
still be equivalent to a number in the list and that needs to
be avoided.
\end{document}

