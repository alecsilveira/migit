\documentclass[12pt]{article}
\usepackage{graphicx}
\usepackage{latexsym, amsmath,amsfonts,amssymb}
\usepackage{xcolor}
\usepackage{wrapfig}
\usepackage[margin=.75in]{geometry}
\usepackage{kpfonts}  %Changing the default fonts
\usepackage[T1]{fontenc}
\setlength{\parskip}{1.2ex} %space between paragraphs
\setlength{\parindent}{1em} %Paragraph indentation
\clubpenalty = 10000
\widowpenalty = 10000
\newcommand\T{\rule{0pt}{2ex}} % \T will create extra space above (used to fix tables)
\newcommand\B{\rule[-1.5ex]{0pt}{0pt}}% \B will create extra space below (used to fix tables)
\linespread{1.25} %spacing between lines
\pagestyle{empty}
\begin{document}
\begin{center}
\textbf{{\Large The Number of Primes is Infinite}}
\end{center}
\textbf{Definition:} If $m$ and $n$ are integers then 
$m$ \textbf{divides} $n$ means $mk=n$ for some
integer $k$.\\\\
\textbf{Examples:} \\\\
$3$ divides $12$ because $(3)(4)=12$.\\
$-7$ divides $14$ because $(-7)(-2)=14$.\\
$5$ divides $0$ because $(5)(0)=0$.\\
$-5$ divides $-25$ because $(-5)(5)=-25$.\\

\noindent\textbf{Lemma}: Let $a$, $b$, and $m$ be 
integers. If $m$ divides $a$ and $m$ divides $b$ then 
$m$ divides $a+b$ and $m$ divides $a-b$.

\noindent\textbf{Proof:} Since $m$ divides $a$ and 
$b$ we have $mk_1=a$ and $mk_2=b$ for some integers
$k_1$ and $k_2$. It follows that $a+b=mk_1+mk_2
=m(k_1+k_2)$ and since $k_1+k_2$ must be an integer, 
$m$ divides $a+b$ as claimed. Likewise, 
$a-b=mk_1-mk_2=m(k_1-k_2)$ and since $k_1-k_2$ must also
be an integer $m$ divides $a-b$.\\

\noindent\textbf{Theorem} (Euclid): There are infinitely 
many primes.

\noindent\textbf{Proof:} Suppose, by way of contradiction, 
there are only $m$ prime numbers. Let 
$p_1$, $p_2$, \ldots, $p_m$ represent these $m$ prime 
numbers and consider $n=p_1p_2 \ldots p_m+1$. 
Since $n$ is bigger than the biggest prime, $n$ is composite 
hence divisible by some prime $p_k$. Now $p_k$ also
divides $p_1p_2 \ldots p_m$ so by the Lemma, $p_k$
divides $n-p_1p_2 \ldots p_m$. Therefore, $p_k$ divides
$1$ but this is clearly a contradiction because all primes
are greater than $1$. It follows that the number of primes
is infinite.
\end{document}