\documentclass[12pt]{article}
\usepackage{amsmath} 
\usepackage[margin=.75in]{geometry} %change the margin
\usepackage{kpfonts}  %Changing the default fonts
\usepackage[T1]{fontenc}
\linespread{1.25} %spacing between lines
\pagestyle{empty} %no page numbers
\begin{document}
\begin{center}
	{\Large \textbf{The Square Root of 2}}\\
\end{center}
\textbf{Requirement:} The students should know the definition of rational and irrational
numbers. \\\\
{\bf Theorem} The $\sqrt{2}$ is not a rational number. \\\\
\textbf{Proof:} Suppose $\sqrt{2}$ is rational number.
\begin{enumerate}
\item If $\sqrt{2}$ is rational then it can be written as $\sqrt{2}=\frac{a}{b}$ where 
$a$ and $b$ are positive integers and $\frac{a}{b}$ is in lowest terms.
\item If $\sqrt{2}=\frac{a}{b}$ then $2=\frac{a^2}{b^2}$
\item If $2=\frac{a^2}{b^2}$ then $a^2=2b^2$.
\item If $a^2=2b^2$ then $a^2$ is divisible by 2.
\item If $a^2$ is divisible by 2 then $a$ is divisble by 2.
\item If $a$ is divisble by 2 then $a=2r$ where $r$ is an integer.
\item If $a=2r$ then $4r^2=2b^2$.
\item If $4r^2=2b^2$ then $2r^2=b^2$.
\item If $2r^2=b^2$ then $b^2$ is divisible by 2.
\item If $b^2$ is divisible by 2 then $b$ is divisible by 2. 
\end{enumerate}
The assumption that $a$ and $b$ are in lowest terms means $a$ and $b$ could not have both been
divisible by $2$. This contradiction means that $\sqrt{2} \neq \frac{a}{b}$ where $a$ and $b$ are
positive integers and $\frac{a}{b}$ is in lowest terms. It follows that $\sqrt{2}$ is an
irrational number.
\end{document}

