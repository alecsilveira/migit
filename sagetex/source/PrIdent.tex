\documentclass[12pt]{article}
\usepackage{graphicx,hyperref,amsmath,natbib,microtype,todonotes}
\usepackage[a4paper,margin=.5in]{geometry}
\usepackage[compact,small]{titlesec}
\setlength{\parskip}{1.3ex} %space between paragraphs
\setlength{\parindent}{2em} %Paragraph indentation
\clubpenalty = 10000
\widowpenalty = 10000
\newcommand\T{\rule{0pt}{3ex}} % \T will create extra space above (used to fix tables)
\newcommand\B{\rule[-1.5ex]{0pt}{0pt}}% \B will create extra space below (used to fix tables)
\usepackage{kpfonts}  %Changing the default fonts
\usepackage[T1]{fontenc}
\usepackage{epsfig,latexsym, amsfonts,amssymb,color}
\usetikzlibrary{arrows}

\pagestyle{empty}
\baselineskip = 1.5\baselineskip

\begin{document}
\noindent Name \rule{80 mm}{.2pt} \hspace{35 mm} Date \rule{25 mm}{.2pt}\\\\
Class \rule{30 mm}{.2pt} \hspace{15 mm}\textbf{Proving Trigonometric Identities}\\\\
Mathematicians spend a lot of time discovering relationships and when they think they've found one, they'll try
to prove it. Trigonometric identities are a great way for you to get a glimpse of this process. Just like the
mathematician trying to prove something new, you aren't allowed to assume the trigonometric identity is true.
Mathematicians know \textbf{the proper procedure is to start with one side and, through algebraic manipulation,
show that it's equal to the other side.}

In this worksheet we'll see why it's wrong to work with both sides at the same time. Start with:

	\begin{minipage}[h]{.1 \textwidth}
	\vspace{0pt}	%\align the tops of minipages
	\[a(x)=\left\{ \begin{array}{ll} 1 & \mbox{if $x \in \mathbb{Z}$}\\
		0 & \mbox{if $x \not \in \mathbb{Z}$} \end{array}\right.\]\\
	\end{minipage}%              The % prevents an interword space from causing a potential problem
	\begin{minipage}[h]{1 \textwidth}
	\vspace{0pt}	%align the tops of minipages
	%\centering
 	\[b(x)=\left\{ \begin{array}{ll} 0 & \mbox{if $x \in \mathbb{Z}$}\\
		1 & \mbox{if $x \not \in \mathbb{Z}$} \end{array}\right.\]\\
	\end{minipage}
		
	\begin{minipage}[h]{.1\textwidth}
	\vspace{0pt}	%align the tops of minipages
	%\centering
 	\[c(x)=\left\{ \begin{array}{ll} 1 & \mbox{if $x \geq 0$}\\
		0 & \mbox{if $x < 0$} \end{array}\right.\]\\
	\end{minipage}
	\begin{minipage}[h]{1\textwidth}
	\vspace{0pt}	%align the tops of minipages
 	\[d(x)=\left\{ \begin{array}{ll} 0 & \mbox{if $x \geq 0$}\\
		1 & \mbox{if $x < 0$} \end{array}\right.\]\\
	\end{minipage}

1. Graph the functions $a(x)$, $b(x)$, $c(x)$, and $d(x)$ below:

\begin{tikzpicture}
  \draw[<->] (0,0)--(6,0);
  \draw[<->] (3,2)--(3,-2);
  \draw[<->] (9,0)--(15,0);
  \draw[<->] (12,2)--(12,-2);
\end{tikzpicture}

\vspace{.4in}

\begin{tikzpicture}
  \draw[<->] (0,0)--(6,0);
  \draw[<->] (3,2)--(3,-2);
  \draw[<->] (9,0)--(15,0);
  \draw[<->] (12,2)--(12,-2);
\end{tikzpicture}

Now suppose someone asks you to whether it's true that $\frac{a(x)}{d(x)}= \frac{c(x)}{b(x)}$ and you 
cross multiply (even though you aren't supposed to) to get $a(x)b(x)=c(x)d(x)$. Since $a(x)>0$ 
whenever $b(x)=0$, we know $a(x)b(x)=0$ for 
every value of $x$. A similar argument shows that $c(x)d(x)=0$ for all values of $x$. In other words, 
both $a(x)b(x)$ and $c(x)d(x)$ are identical to the line $y=0$; the domains are $(-\infty, \infty)$ and
the ranges are $\{0\}$. So it's clear that $\frac{a(x)}{d(x)}= \frac{c(x)}{b(x)}$? Not so fast!
\newpage
2. Graph $\frac{a(x)}{d(x)}$ and $\frac{c(x)}{b(x)}$.\\\\
\begin{tikzpicture}
  \draw[<->] (0,0)--(6,0);
  \draw[<->] (3,2)--(3,-2);
  \draw[<->] (9,0)--(15,0);
  \draw[<->] (12,2)--(12,-2);
\end{tikzpicture}

\vspace{.4in}


3. Find the domain and range of $\frac{a(x)}{d(x)}$ and $\frac{c(x)}{b(x)}$.\\\\\\
Domain of $\frac{a(x)}{d(x)}$ \hspace{.75in} Range of $\frac{a(x)}{d(x)}$ \hspace{1in} 
Domain of $\frac{c(x)}{b(x)}$ \hspace{.75in} Range of $\frac{c(x)}{b(x)}$\\\\\\\\\\

Notice that the answers are much different now!  The domains are not $(-\infty, \infty)$ and they're
different. The problem is the original identity isn't true and when you cross multiply it out like it
is true, you get into trouble.

So if you have a trigonometric identity
 \[\frac{\cos(\theta)}{1-\sin(\theta)}=\frac{1+\sin(\theta)}{\cos(\theta)}\] \\
you are not allowed to solve it like this:\\
$\cos(\theta)\cos(\theta)= (1-\sin(\theta))(1+\sin(\theta))$ which simplifies to $\cos^2(\theta)= 1-\sin^2(\theta)$ and then to $\cos^2(\theta)+ \sin^2(\theta)=1$. 

In this case the identity is true, but look a little closer. The lefthand side is not defined whenever
$\sin(\theta)=1$; i.e., $\theta= \frac{\pi}{2} + 2n\pi$. The righthand side is not defined when 
$\cos(\theta)=0$, which is when $\theta= \frac{\pi}{2} + n\pi$. The lefthand side and righthand side have
different domains, but whenever both are defined they are the same. When you cross multiply to get
$\cos(\theta)\cos(\theta)=(1-\sin(\theta))(1+\sin(\theta))$ both the lefthand side and righthand side are
defined for all real number values of $\theta$. 

4. Solve the trigonometric identity
\[\frac{\cos(\theta)}{1-\sin(\theta)}=\frac{1+\sin(\theta)}{\cos(\theta)}\] 
by working from left to right.
\end{document}
