\documentclass[12pt]{article}
\usepackage{graphicx}
\usepackage{latexsym, amsmath,amsfonts,amssymb}
\usepackage{xcolor}
\usepackage{fancyhdr}
\usepackage{fancybox}
\usepackage{nccmath}%to get small medium and large math
\usepackage{natbib,todonotes}
\usepackage{wrapfig}
\usepackage[margin=1in]{geometry}
\usepackage{kpfonts}  %Changing the default fonts
\usepackage[T1]{fontenc}
\usepackage{tkz-euclide}
\setlength{\parskip}{1.2ex} %space between paragraphs
\setlength{\parindent}{1em} %Paragraph indentation
\clubpenalty = 10000
\widowpenalty = 10000
\newcommand\T{\rule{0pt}{2ex}} % \T will create extra space above (used to fix tables)
\newcommand\B{\rule[-1.5ex]{0pt}{0pt}}% \B will create extra space below (used to fix tables)
\usetikzlibrary{calc,trees,positioning,arrows,chains,shapes.geometric,%
  decorations.pathreplacing,decorations.pathmorphing,shapes,%
  matrix,shapes.symbols,plotmarks,decorations.markings,shadows}
\linespread{1.25} %spacing between lines
\newtheorem{Theorem}{Theorem}
\pagestyle{empty}
\begin{document}
\begin{center}
\Large{\textbf{Pythagorean Theorem}}
\end{center}
\textbf{Pythagorean Theorem:} If $T$ is a right triangle with sides of length
$a$ and $b$ and a hypotenuse of length $c$ then $c^2=a^2+b^2$.

\noindent\textbf{Proof:} Draw a square with sides of length $a+b$ and label
the corners $A$, $B$, $C$, $D$.  Next, go to each of the four sides of the 
square in a counterclockwise manner, marking off the distance $b$ on each side,
to create the points $E$, $F$, $G$, and $H$. Finally, draw the line segments $\overline{EF}$, $\overline{FG}$,
$\overline{GH}$, $\overline{HE}$ to create the diagram below.

\begin{center}
\begin{tikzpicture}[scale = 1.75]
\draw (0,4)--(4,4)--(4,0)--(0,0)--cycle;
\draw (3,0)--(4,3)--(1,4)--(0,1)--cycle;
\coordinate [label=below left:$A$] (A) at (0,0);
\coordinate [label=below:$B$] (B) at (4,0);
\coordinate [label=above right:$C$] (C) at (4,4);
\coordinate [label=above left:$D$] (D) at (0,4);
\coordinate [label=below:$E$] (E) at (3,0);
\coordinate [label=right:$F$] (F) at (4,3);
\coordinate [label=above:$G$] (G) at (1,4);
\coordinate [label=left:$H$] (H) at (0,1);
\foreach \point in {A,B,C,D,E,F,G,H}
\fill [black,opacity=1] (\point) circle (1pt);
\coordinate [label=below:$b$] (S1) at (1.5,0);
\coordinate [label=right:$b$] (S2) at (4,1.5);
\coordinate [label=above:$b$] (S3) at (2.5,4);
\coordinate [label=left:$b$] (S4) at (0,2.5);
\coordinate [label=below:$a$] (T1) at (3.5,0);
\coordinate [label=right:$a$] (T2) at (4,3.5);
\coordinate [label=above:$a$] (T1) at (0.5,4);
\coordinate [label=left:$a$] (T1) at (0,0.5);
\coordinate [label=below:$c$] (U1) at (1.5,.8);
\coordinate [label=right:$c$] (U2) at (3.2,1.5);
\coordinate [label=below:$c$] (U3) at (2.5,3.5);
\coordinate [label=left:$c$] (U4) at (.75,2.5);
\end{tikzpicture}
\end{center}

The triangles $AHE$, $BEF$, $CFG$, and $DGH$ each contain a right angle
of the square ABCD, so they are congruent right triangles as well by SAS. 
You can verify that $EFGH$ is a square by observing, for example, that $m(\angle AEH)+m(\angle\, HEF) + m(\angle\, FEB) = 180^{\circ}$. Since the Triangle Sum allows us to conclude $m(\angle\, AEH)+m(\angle\, FEB) = 90^{\circ}$ we know $m(\angle\, HEF)= 90^{\circ}$. By showing that
each interior angle of $EFGH$ is a right angle through a similar argument 
it will follow that $EFGH$ is  also a square. Since 
$$\mbox{area of square $ABCD\, - $ area of the four triangles $ = $ area of square $EFGH$}$$
and the area of each triangle is $\frac{1}{2}ab$ it follows that 
$(a+b)^2 - 4\frac{1}{2}ab = c^2$ which simplifies to $a^2+2ab+b^2-2ab
=c^2$. Simplifying further and rearranging the terms yields $c^2=a^2+b^2$ as claimed.

\end{document}

