\documentclass[12pt]{article}
\usepackage{latexsym, amsmath,amsfonts,amssymb}
\usepackage[margin=.75in]{geometry}
\usepackage{kpfonts}  %Changing the default fonts
\usepackage[T1]{fontenc}
\usepackage{graphicx}
\newcommand\T{\rule{0pt}{4ex}} % \T will create extra space above (used to fix tables)
\newcommand\B{\rule[-2.5ex]{0pt}{0pt}}% \B will create extra space below (used to fix tables)
\setlength{\parskip}{1.2ex} %space between paragraphs
\setlength{\parindent}{1em} %Paragraph indentation
\linespread{1.5} %spacing between lines
\pagestyle{empty}
\begin{document}
\begin{center}
\Large{\textbf{Series and Integrals}}
\end{center}
{\bf Question:} What is the area between the graph of $y=x^2$ and the $x$-axis, from 0 to 2?\\
\begin{figure}[h!!]
\begin{center}
  \includegraphics[width=2.1in]{Intx2Region.pdf} 
\end{center}
\caption{What's the area under the curve $y=x^2$.}
\end{figure}

Before calculus, we learned to calculate the formulas for various
polygons: square, rectangle, triangles, parallelograms, 
trapezoids, and so on. These shapes had sides that were formed
from line segments. We can, however, approximate the area 
under the curve using rectangle. If we make sure to use rectangles
that are always below the curve, then the estimate of the area
will be too low. If we use rectangle that are always above the 
curve then the estimate of the area is too high. 
\begin{figure}[h!]
	\begin{minipage}[c]{.5\textwidth}
	\vspace{0pt}	%\align the tops of minipages
	\centering
	  \includegraphics[width=2.1in]{Intx2Low.pdf} 
	\end{minipage}%              The % prevents an interword space from causing a potential problem
	\begin{minipage}[c]{.5\textwidth}
	\vspace{0pt}	%align the tops of minipages
	  \includegraphics[width=2.1in]{Intx2Up.pdf} 
	\end{minipage}
\caption{An underestimate and overestimate of the area under
the curve $y=x^2$.}
\end{figure}

By increasing the number of rectangles we can get a better 
estimate of the area both on the low side and on the high side. The 
incredible power of calculus comes from being able to use limits
to get the exact area by looking at an infinite number of 
rectangles. Since high school students are supposed
to know that $\sum_{i=1}^{n}i^2=\frac{n(n+1)(2n+1)}{6}$ \,a
great example to illustrate how series can give us the exact
answer is the function $y=x^2$. This is accomplished by getting 
an underestimate of the area and an overestimate of the area. 
If $A$ is the area under the curve then we know it is greater 
than or equal to 0. On the other hand, the height of $x^2$ is at 
most 4 between 0 and 2. Therefore the area is at most 8. To get 
a better estimate we use more rectangles. 
Our underestimate and overestimte will be more accurate.

\begin{center}
\begin{tabular}{|l|l|l|}
\hline
$R_n$ & Underestimate of A & Overestimate of A\\
\hline
$1$ & $2(0)=0$ & $2(4)=8$\\
\hline
$2$ & $1(0)+1(1)=1$ & $1(1)+1(4)=5$\\
\hline
3 & $\frac{2}{3}(0)+\frac{2}{3}\left(\frac{2}{3}\right)^2+\frac{2}{3}\left(\frac{4}{3}\right)^2$ &
$\frac{2}{3}\left(\frac{2}{3}\right)^2+\frac{2}{3}\left(\frac{4}{3}\right)^2+\frac{2}{3}\left(\frac{6}{3}\right)^2$\T\B\\
\hline
4 & $\frac{2}{4}(0)+\frac{2}{4}\left(\frac{2}{4}\right)^2+\frac{2}{4}\left(\frac{4}{4}\right)^2+\frac{2}{4}\left(
\frac{6}{4}\right)$ & $\frac{2}{4}\left(\frac{2}{4}\right)^2+\frac{2}{4}\left(\frac{4}{4}\right)^2+
\frac{2}{4}\left(\frac{6}{4}\right)^2+\frac{2}{4}\left(\frac{8}{4}\right)^2$\T \B\\
\hline
5 & $\frac{2}{5}\left(\left(\frac{2}{5}\right)^2+\left(2\frac{2}{5}\right)^2+\left(3\frac{2}{5}\right)^2+ 
\left(4\frac{2}{5}\right)^2\right)$ & $\frac{2}{5}\left(\left(\frac{2}{5}\right)^2+\left(2\frac{2}{5}\right)^2+
\left(3\frac{2}{5}\right)^2+\left(4\frac{2}{5}\right)^2+\left(5\frac{2}{5}\right)^2\right)$\T  \B\\
\hline
\vdots & \vdots & \vdots\\
\hline
k & $\frac{2}{k}\left(\left(\frac{2}{k}\right)^2+\left(2\frac{2}{k}\right)^2+\ldots+
\left((k-1)\frac{2}{k}\right)^2\right)$ & $\frac{2}{k}\left(\left(\frac{2}{k}\right)^2+\left(2\frac{2}{k}\right)^2+\ldots+
\left(k\frac{2}{k}\right)^2\right)$\T\B\\
\hline
\end{tabular}
\end{center}

The process of estimating the area under a curve is shown below 
for the case with $5$ rectangles. Notice that in the 
underestimate there are $4$ rectangles. You can't see the fourth 
rectangle because it has a height of $0$.

Let's simplify the underestimate and overestimate for $k$ rectangles. The underestimate of the area is
$\left(\frac{2}{k}\right)^3\left[1^2+2^2+\ldots+(k-1)^2\right]=\left(\frac{2}{k}\right)^3\left[\frac{(k-1)(k)(2(k-1)+1)}{6}\right]=\frac{8}{k^2}\frac{(k-1)(2k-1)}{6}=\frac{8k^2-12k+4}{3k^2}$ with a little bit of algebra.

The overestimate of the area is $\left(\frac{2}{k}\right)^3\left[1^2+2^2+\ldots+k^2\right]=\left(\frac{2}{k}\right)^3\left[\frac{(k-1)(k)(2k+1)}{6}\right]=\frac{8}{k^2}\frac{(k+1)(2k+1)}{6}$ which finally
simplifies to $\frac{8k^2+12k+4}{3k^2}$.

Putting these two results together gives shows the area is 
bounded: $\frac{8k^2-12k+4}{3k^2} \leq A \leq \frac{8k^2+12k+4}{3k^2}$ which means
$\lim_{k \to \infty}\frac{8k^2-12k+4}{3k^2} \leq A \leq \lim_{k \to \infty}\frac{8k^2+12k+4}{3k^2}$ hence
$A=\frac{8}{3}$.  We've found the {\em exact} area under a 
curve using limits BUT we needed a "nice" function with a
series that we knew in order to get the answer. 
Mathematicians have found: If $f$ is a continuous function on $\left[a,b\right]$ and $f(x) \geq 0$ on 
$\left[a,b\right]$ then the {\bf area, A, under the curve} is given by $A=\lim_{n \to \infty}\sum_{i=1}^{n}f(x_i)
(\Delta x)$ where the $x_i$ are the endpoints of the rectangles.
That's a complicated looking formula, but it's just the way that
mathematicians describe the process of adding up the area
of the rectangles that we illustrated. The {\bf definite integral} of 
{\bf any} function $f$ defined from $a$ to $b$ is $\int_a^bf(x)dx=
\lim_{n \to \infty} \sum_{i=1}^nf(c_i)(\Delta x)$ provided the 
limit exists. The $c_i$ are any points in the rectangle. That is, the 
evaluation points don't have to be equally spaced. When 
the limit exists we say that $f$ is {\bf integrable} on $\left[a,b
\right]$.

Having a conrete example of using series to figure out the area
under a curve will help the student make sense
of $A=\lim_{n \to \infty}\sum_{i=1}^{n}f(x_i)(\Delta x)$.
\end{document}

