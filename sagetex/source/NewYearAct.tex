% Copyright 2014 http://www.highschoolmathandchess.com/
% This program is free software: you can redistribute it and/or modify
%   it under the terms of the GNU General Public License as published by
%    the Free Software Foundation, either version 3 of the License, or
%    (at your option) any later version.
%    This program is distributed in the hope that it will be useful,
%   but WITHOUT ANY WARRANTY; without even the implied warranty of
%   MERCHANTABILITY or FITNESS FOR A PARTICULAR PURPOSE.  See the
%    GNU General Public License for more details.
%
%    You should have received a copy of the GNU General Public License
%    along with this program.  If not, see <http://www.gnu.org/licenses/>.
\documentclass{article}
\usepackage{amsmath, amsfonts, amssymb}
\usepackage[T1]{fontenc}% change the fonts
\usepackage[margin=.5in]{geometry}%  sets the margins
\usepackage{sagetex}%  use Sage for it's math ability
\usepackage{supertabular}
\pagestyle{empty}
\begin{document}
\fontsize{15pt}{17pt}\selectfont %sets font size and line spacing
\noindent
\begin{tabular}{@{}l p{1.6in}r @{}}
Name: \vrule height 0 pt depth 0.4 pt width 3.0 in & & Period: \vrule height 0 pt depth 0.4 pt width .75 in \\\\
Date: \vrule height 0 pt depth 0.4 pt width 1.0 in
\end{tabular}
\begin{center}
{\LARGE $2015$ Math Activity}
\end{center}
\noindent \textbf{Instructions: }Use the digits of the year: $0$,
$1$, $2$, $5$ one time each along with operations ($+$, 
$-$, $\times$, $\div$, and exponentiation: $\wedge$) 
to write an expression for each of the numbers from $1$ to $100$. You 
may group digits (e.g. $215$) and use decimals (e.g. $.02$). 
Example: To get $0$ we could write the expression $(12\wedge 5)\times 0$. 
How many can you get?
\begin{sagesilent}
N = 100
lines = ceil(N/3)
extra = N%3
output = ""
output += r"\begin{supertabular}{p{5.5cm}|p{5.5cm}|p{5.5cm}}"
for i in range(1, lines):
    output += r"%d & %d & %d \\ "%(3*(i-1)+1,3*(i-1)+2,3*(i-1)+3)
    output += r" &  & \\ "
    output += r"\hline "
    
if extra == 0:
    output += r"%d & %d & %d \\ "%(3*(lines-1)+1,3*(lines-1)+2,3*(lines-1)+3)
    output += r" &  & \\ "
elif extra == 1:
    output += r"%d &  & \\ "%(3*(lines-1)+1)
else:
    output += r"%d & %d & \\ "%(3*(lines-1)+1, 3*(lines-1)+2)
output += r"\end{supertabular}"
\end{sagesilent}
\begin{center}
\sagestr{output}
\end{center}
\end{document}
