\documentclass[12pt]{article}
\usepackage[margin=.75in]{geometry}
\usepackage{amssymb} %needed for the check mark
\usepackage{textcomp} %needed for the X mark
\begin{document}
\begin{center}
  \Large{\textbf{Exponents quiz: do you know the fine print?}}
\end{center}
\textbf{Requirements:} Students must have learned the rules of exponents previously.\\
\textbf{Background:} Here's an informal quiz you can give to see
how well your class understands the rules of exponents. I also use it 
as a springboard to impress upon students that the details of a theorem matter. I also use it to expose some of the weaknesses 
in their ``trusty'' calculator.\\

Start the class by giving an informal ``quiz". Put the following problems on the board and give
the class 5-10 minutes to work out the answers.\\\\
Simplify as much as possible.\\
a) $-8^{\frac{1}{3}}$ \hspace{60pt} b) $-5^0$ \hspace{60pt} c) $0^1$ \hspace{60pt} 
d) $0^0$\\
e) $-2^6$ \hspace{64pt} f) $(-1/64)^{(-1/3)}$  \hspace{12pt} g) $(a-b)^0$
\hspace{31pt} h) $(-1)^{\frac{6}{10}}$\\

After the time is up, choose different students to give their answer to a particular problem. Write
the answer next to each problem. Before you review each problem have students raise their hand
if they think the answer is correct. Next, have students who think the answer is incorrect to 
raise their hands. After your poll of each question has been conducted, put a check mark 
$\checkmark$ by the answer if it is correct and an \texttimes \,by it if it's incorrect. The
answers are:\\\\
a) $-2$ \hspace{35pt} b) $-1$ \hspace{35pt} c) $0$ \hspace{35pt} 
d) undefined  \hspace{37pt} e) $-64$\\
f) $-4$ \hspace{37pt} g) $1$ if $(a-b) \neq 0$
\hspace{35pt} h) undefined \\

If your class didn't do very well, that's okay! They're more likely to 
listen carefully now. I expect even the best students to miss some 
of the details (for example in problem ``g''). Problem ``h'' isn't 
standard at all, though. I've known math 
teachers who have stumbled on it. I included it specifically 
to motivate a teaching point, but more about that in a later post.
\end{document}