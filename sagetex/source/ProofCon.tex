\documentclass[12pt]{article}
\usepackage{latexsym, amsmath,amsfonts,amssymb}
\usepackage[margin=.75in]{geometry}
\usepackage{kpfonts}  %Changing the default fonts
\usepackage[T1]{fontenc}
\setlength{\parskip}{1.2ex} %space between paragraphs
\setlength{\parindent}{1em} %Paragraph indentation
\linespread{1.25} %spacing between lines
\pagestyle{empty} %no page numbers
\begin{document}
\begin{center}
\Large{\textbf{A Useful Proof by Contradiction}}
\end{center}
Every high school student must master the basic skill of factoring
a number, at least in theory.  Those learning the process 
will often try factoring $149$ by attempting to divide it by the 
numbers $2, 3, 4, \ldots$ and so on all the way up to $149$ (if 
they have the patience) until, hopefully, someone tells them they 
don't have to try dividing by all those numbers. Instead, they
just need to check the prime numbers between $2$ and 
$\sqrt{149} <13$. Why is that so? It's a consequence of the 
next theorem, which is the shortest meaningful proof by 
contradiction I know of.\\\\
\textbf{Theorem:} Let $p$ and $q$  be positive integers. If 
$N=pq$ then either $p \leq \sqrt{N}$ or $q \leq \sqrt{N}$.\\\\
Proof: Suppose $p > \sqrt{N}$ and $q > \sqrt{N}$. It follows
 that $pq > \sqrt{N}\sqrt{N}=N$ hence either $p \leq \sqrt{N}$
 or $q \leq \sqrt{N}$.\\
 
If $N$ is composite number, it has a factor less than $\sqrt{N}$.
That factor must also be divisible by a prime number. It follows
that every composite number $N$ is divisible by a prime number 
between $2$ and $\sqrt{N}$. Therefore, to check whether $149$ is
prime we need to test whether it is divisible by $2, 3, 5, 7$, and 
$11$. Knowing this will save students a lot of time when they
are factoring numbers. 
\end{document}