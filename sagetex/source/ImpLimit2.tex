\documentclass[12pt]{article}
\usepackage{graphicx}
\usepackage{latexsym, amsmath,amsfonts,amssymb}
\usepackage{xcolor}
\usepackage{hyperref}
\usepackage{fancyhdr}
\usepackage{fancybox}
%\usepackage{sagetex}
\usepackage{nccmath}%to get small medium and large math
\usepackage{framed}
%\usepackage{microtype}
\usepackage{natbib,bm,url,todonotes}
\usepackage{wrapfig}
\usepackage[ugly]{units}
\usepackage[margin=.75in]{geometry}
\usepackage[compact,small]{titlesec}
\usepackage{kpfonts}  %Changing the default fonts
\usepackage[T1]{fontenc}
\hypersetup{colorlinks=true,linkcolor=black} %take off the box, have the link black instead of red
\setlength{\parskip}{1.2ex} %space between paragraphs
\setlength{\parindent}{1em} %Paragraph indentation
\clubpenalty = 10000
\widowpenalty = 10000
\newcommand\T{\rule{0pt}{2ex}} % \T will create extra space above (used to fix tables)
\newcommand\B{\rule[-1.5ex]{0pt}{0pt}}% \B will create extra space below (used to fix tables)
\usetikzlibrary{calc,trees,positioning,arrows,chains,shapes.geometric,%
  decorations.pathreplacing,decorations.pathmorphing,shapes,%
  matrix,shapes.symbols,plotmarks,decorations.markings,shadows}
\linespread{1.25} %spacing between lines
\pagestyle{empty}
\begin{document}
\begin{center}
\Large{\textbf{An Important Limit}}
\end{center}
Proof, the essence of mathematics, is typically beyond the range of most high school students.
When you're able to find a proof that they can understand, even if it's just a mathematical
argument, it's a good idea to expose them to it. It's a break from the tedious calculations and
will give students at least some appreciation of the what mathematicians do. One argument
that every Calculus students should see is $\underset{x \to 0}{\lim}\,\frac{\sin(x)}{x}=1$.
Determining the limit forces students to recall important information about similar triangles and 
sectors of circles and apply it to a solving a different problem.

\begin{wrapfigure}{l}{7.cm}
\includegraphics[width=6.5cm]{LimSin.pdf}
\caption{Start with the unit circle.} \label{fg:ucatO}
\end{wrapfigure}
To calculate the limit, draw a unit circle centered at the origin. I've chosen $\theta$ to 
be big enough that we can get a nice picture. The first important observation is that every 
angle $\theta$ can be associated with a point $(x, y)$, which is indicated with the red 
point. The point $(x, y)$ can be expressed as $(\cos(\theta), \sin(\theta))$. Dropping 
a perpendicular from the point $(x, y)$ to the origin creates a right triangle 
with a hypotenuse of length $1$. Corresponding to this triangle is a similar triangle 
which contains the point $(1,0)$. This is illustrated more clearly in Figure 2; the two 
triangles are $\triangle OSR$ and $\triangle OTU$, respectively.

Since the radius of the circle is $1$, the area of the unit circle is just $\pi(1)^2=\pi$ so the 
area of sector $OSU$ is $\frac{\theta}{2\pi}\pi=\frac{\theta}{2}$. Figure 2 makes is clear that 
area of triangle $\triangle OSR$ is less than or equal to area of sector $OSU$  which is less 
than or equal to area of triangle $\triangle OTU$. The area of triangle $OSR$ is just 
$\frac{1}{2}\cos(\theta)\sin(\theta)$ and the area of the sector is $\frac{1}{2}\theta$. 

\begin{wrapfigure}{l}{7cm}
\includegraphics[width=5.75cm]{LimSin2.pdf}
\caption{A sector of a circle.} \label{fg:sec}
\end{wrapfigure}
Finally, notice that for $\triangle OTU$ we have 
$\tan(\theta)=\frac{TU}{1}=TU$, so the area of $\triangle OTU$ is just $\frac{1}{2}\tan(\theta)=\frac{\sin(\theta)}{2\cos(\theta)}$.
Put it all together and we get the inequality chain:
\[0 \leq \frac{1}{2}\cos(\theta)\sin(\theta) \leq \frac{\theta}{2} \leq \frac{1}{2}\tan(\theta)=\frac{\sin(\theta)}{2\cos(\theta)}\]
which is equivalent to  $0 \leq \cos(\theta)\sin(\theta) \leq \theta \leq \tan(\theta)=\frac{\sin(\theta)}{\cos(\theta)}$. Since $\theta \neq 0$,  $\sin(\theta) \neq 0$ either, so we can divide by $\sin(\theta)$ to get
$\cos(\theta) \leq \frac{\theta}{\sin(\theta)} \leq \frac{1}{\cos(\theta)}$. If $\cos(\theta) \neq 0$ then $\frac{1}{\cos(\theta)} \geq
 \frac{\sin(\theta)}{\theta} \geq \cos(\theta)$. Take the limit of each term as $x$ approaches $0$ to get 
$\lim_{x \to 0}\cos(\theta) \leq \lim_{x \to 0}\frac{\sin(\theta)}{\theta}  \leq \lim_{x \to 0}\frac{1}{\cos(\theta)}$. This simplifies to 
$1 \leq \lim_{x \to 0}\frac{\sin(\theta)}{\theta} \leq 1$, hence $\lim_{x \to 0}\frac{\sin(\theta)}{\theta} =1$.
\end{document}