\documentclass[12pt]{article}
\usepackage{graphicx}
\usepackage{latexsym, amsmath,amsfonts,amssymb}
\usepackage{fancybox}
\usepackage{varioref}  % This is for refencing pages, tables, figures, etc.
%\usepackage[usenames,dvipsnames]{color}
\usepackage{xcolor}
\usepackage{fancyhdr}
\usepackage{framed}
\usepackage{booktabs}
%\input{MyPackages}
\begin{document}
\begin{center}
\Large{\textbf{$\frac{1}{0}$ is Undefined }}
\end{center}
$\frac{6}{3} = 2$ because $(3)(2)=6$.\\\\
$\frac{15}{3} = 5$ because $(3)(5)=15$.\\\\
$\frac{200}{25} = 8$ because $(25)(8)=200$.\\\\
$\frac{7}{8} = 0.875$ because $(8)(0.875)=7$.\\

This relationship will hold between any any three numbers involved in division. Therefore, 
if $\frac{1}{0}$ was defined, say $\frac{1}{0} = k$, then $0(k)=1$. Since $0(k)$ must
equal $0$, for any number $k$, the assumption that  $\frac{1}{0}$ was defined is false.
\end{document}