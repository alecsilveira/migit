\documentclass[11pt,a4paper]{article}
\usepackage[spanish]{babel}
\usepackage[ansinew]{inputenc}
\usepackage{graphicx}
\usepackage{amsmath}
\usepackage[amsmath]{maxiplot}

\title{Maxiplot: Maxima y Gnuplot en \LaTeX.\\}

\def\Maxima{\emph{Maxima}}
\def\Gnuplot{\emph{Gnuplot}}
\makeatletter % changes the catcode of @ to 11
%\def\@setpar#1{\def\par{#1}\def\@par{#1}}
%\def\@par{\let\par\@@par\par}
%\def\@restorepar{\def\par{\@par}}
\let\@@par\par
\makeatother % changes the catcode of @ back to 12


\begin{document}

%%%%%%%%%%%%%%%%%%%%%

\[
 \begin{maxima}
 f:1/n^2,
 numer:true,
 tex('sum(f,n,1,100)),
 print("="),
 tex(nusum(f,n,1,100))
 \end{maxima}
 \] 
%%%%%%%%%%%%
$$
 \begin{maxima}
 A:matrix([1,2,3],[4,8,5],[9,5,4]),
 B:matrix([0,1,8],[1,7,0],[3,1,0]),
 tex(A),
 tex(B),
 print("="),
 tex(A.B)
 \end{maxima}
$$


% %%%%%%%%%%%%%%%
 La matriz 
$\begin{maxima}
  A:matrix([3,1],[2,4]),
  tex('A),
  print("="),
  tex(A)
\end{maxima}$
tiene por adjunta
$
\begin{maxima}
  print("\\mbox{Adj}(A)="),
  tex(adjoint(A))
\end{maxima}
$ 
cuyo determinante es 
$
\mbox{Det}(A)=
\imaxima{tex(determinant(A))}
$ 
y por consiguiente su inversa resulta ser
$$
\begin{maxima}
  print("\\mbox{Inv}(A)="),
  tex(invert(A))
\end{maxima}
$$

El polinomio caracter\'istico de la matriz $A$ es
$
\mbox{Det}(A-xI)=
\begin{maxima}
P:expand(charpoly(A,x)), 
tex(P)
\end{maxima}
$ 
cuyas ra\'ices son 
$\imaxima{tex(solve(P))}$ 
%%%%%%%%%%%%%
%%%%%%%%%%%%%%%
%%%%%%%%%%%%%%
%%%%%%%%%%%%
%%%%%%%%%%%%
%%%%%%%%%%%

%

\pagebreak
\begin{maximacmd}
  tangente(fx,a):=expand(ev(fx,x=a)
                  +subst(a,x,diff(fx,x))*(x-a))$
  plot2d([sin(x),tangente(sin(x),%pi/3)], [x,-3,3],
         [gnuplot_preamble,"set zeroaxis;"],
         [gnuplot_term, png],
         [gnuplot_out_file,"./\jobname2D.png"])$
\end{maximacmd}
\begin{center}
  \mxpIncludegraphics[scale=0.60]{\jobname2D.png}
\end{center}
\pagebreak
Los entornos presentados hasta ahora pueden contener comandos \LaTeX{} que ser\'an
sustituidos antes de pasar al archivo \texttt{mac}. Ahora bien, esta capacidad
puede, a veces, no ser deseable o puede dar problemas con algunas secuencias
de caracteres. Para evitar esto est\'an los entornos \texttt{vmaxima} y \texttt{vmaximacmd},
cuyo uso es similar a los anteriores, pero el volcado es literal. Estos entornos
est\'an basados en el paquete \texttt{verbatim} de \LaTeX{}. 

\pagebreak

\subsubsection{Gnuplot.}
Si bien \Maxima{} permite la inclusi\'on de gr\'aficos a trav\'es de \Gnuplot{}, a veces puede ser preferible trabajar directamente con \'este \'ultimo. Para esto usaremos los entornos \texttt{gnuplot} y su versi\'on `verbatim' \texttt{vgnuplot}.

Un ejemplo 3D
\begin{verbatim}
\begin{gnuplot}
  set term png crop enhanced font "calibri, 10"
  set output "toros.png"
  set parametric
  set urange [0:2*pi]
  set vrange [-pi:pi]
  set isosamples 36,24
  set hidden3d
  set view 75,15,1,1
  unset key
  set ticslevel 0
  x1(u,v)=cos(u)+.5*cos(u)*cos(v)
  y1(u,v)=sin(u)+.5*sin(u)*cos(v)
  z1(u,v)=.5*sin(v)
  x2(u,v)=1+cos(u)+.5*cos(u)*cos(v)
  y2(u,v)=.5*sin(v)
  z2(u,v)=sin(u)+.5*sin(u)*cos(v)
  set multiplot
  splot x1(u,v), y1(u,v), z1(u,v) w pm3d, x2(u,v), y2(u,v), z2(u,v) w pm3d
  splot x1(u,v), y1(u,v), z1(u,v) lt 3,   x2(u,v), y2(u,v), z2(u,v) lt 5 
\end{gnuplot}
\begin{center}
  \mxpIncludegraphics[scale=0.75]{toros.png}
\end{center}
\end{verbatim}


\begin{gnuplot}
  set term png crop enhanced font "calibri, 10"
  set output "toros.png"
   set parametric
  set urange [0:2*pi]
  set vrange [-pi:pi]
  set isosamples 36,24
  set hidden3d
  set view 75,15,1,1
  unset key
  set ticslevel 0
  x1(u,v)=cos(u)+.5*cos(u)*cos(v)
  y1(u,v)=sin(u)+.5*sin(u)*cos(v)
  z1(u,v)=.5*sin(v)
  x2(u,v)=1+cos(u)+.5*cos(u)*cos(v)
  y2(u,v)=.5*sin(v)
  z2(u,v)=sin(u)+.5*sin(u)*cos(v)
  set multiplot
  splot x1(u,v), y1(u,v), z1(u,v) w pm3d, x2(u,v), y2(u,v), z2(u,v) w pm3d
  splot x1(u,v), y1(u,v), z1(u,v) lt 3,   x2(u,v), y2(u,v), z2(u,v) lt 5 
  unset multiplot
\end{gnuplot}
\begin{center}
  \mxpIncludegraphics[scale=0.75]{toros.png}
\end{center}
Observemos el comando \verb|\mxpIncludegraphics|: el uso es el mismo que \verb|includegraphics| del
paquete \verb|graphicx|, de hecho lo que hace es asegurarse de que existe el archivo gr\'afico e invocar dicha macro.
\pagebreak
\begin{gnuplot}
set terminal png transparent nocrop enhanced size 450,320 font "arial,8" 
set output "simple1.png"
unset parametric
set samples 100, 100
set title "Simple Plots" 
set title  font ",20" norotate
set xrange [-10:10]
plot[x=-10:10] sin(x),atan(x),cos(atan(x))
unset multiplot
\end{gnuplot}
\begin{center}
  \mxpIncludegraphics[scale=0.75]{simple1.png}
\end{center}
%%%%%%%%%%%%5
Observemos el comando \verb|\mxpIncludegraphics|: el uso es el mismo que \verb|includegraphics| del
paquete \verb|graphicx|, de hecho lo que hace es asegurarse de que existe el archivo gr\'afico e invocar dicha macro.
\pagebreak

\end{document}