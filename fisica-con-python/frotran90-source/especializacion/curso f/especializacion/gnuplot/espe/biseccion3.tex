
\documentclass{article}
\usepackage{amssymb}

%%%%%%%%%%%%%%%%%%%%%%%%%%%%%%%%%%%%%%%%%%%%%%%%%%%%%%%%%%%%%%%%%%%%%%%%%%%%%%%%%%%%%%%%%%%%%%%%%%%%
\usepackage{graphicx}
\usepackage{amsmath}

%TCIDATA{OutputFilter=LATEX.DLL}
%TCIDATA{Created=Fri Oct 24 14:50:50 2003}
%TCIDATA{LastRevised=Fri Oct 24 17:09:00 2003}
%TCIDATA{<META NAME="GraphicsSave" CONTENT="32">}
%TCIDATA{<META NAME="DocumentShell" CONTENT="General\Blank Document">}
%TCIDATA{CSTFile=LaTeX article (bright).cst}

\newtheorem{theorem}{Theorem}
\newtheorem{acknowledgement}[theorem]{Acknowledgement}
\newtheorem{algorithm}[theorem]{Algorithm}
\newtheorem{axiom}[theorem]{Axiom}
\newtheorem{case}[theorem]{Case}
\newtheorem{claim}[theorem]{Claim}
\newtheorem{conclusion}[theorem]{Conclusion}
\newtheorem{condition}[theorem]{Condition}
\newtheorem{conjecture}[theorem]{Conjecture}
\newtheorem{corollary}[theorem]{Corollary}
\newtheorem{criterion}[theorem]{Criterion}
\newtheorem{definition}[theorem]{Definition}
\newtheorem{example}[theorem]{Example}
\newtheorem{exercise}[theorem]{Exercise}
\newtheorem{lemma}[theorem]{Lemma}
\newtheorem{notation}[theorem]{Notation}
\newtheorem{problem}[theorem]{Problem}
\newtheorem{proposition}[theorem]{Proposition}
\newtheorem{remark}[theorem]{Remark}
\newtheorem{solution}[theorem]{Solution}
\newtheorem{summary}[theorem]{Summary}
\newenvironment{proof}[1][Proof]{\textbf{#1.} }{\ \rule{0.5em}{0.5em}}
\input{tcilatex}

\begin{document}


METODO DE BISECCION

\bigskip 

Si f es una funci\'{o}n continua en el intervalo $\left[ a.b\right] $ y si $%
f\left( a\right) f\left( b\right) <0$, entonces $f$ debe tener un cero en $%
\left( a,b\right) .$ Esta es una consecuencia del teorema del valor
intermedio para funciones continuas. 

El m\'{e}todo de bisecci\'{o}n explota esta idea de la siguiente manera , si 
$f\left( a\right) f\left( b\right) <0,$ se calcula $c=1/2(a+b)$ y se
averigua si $f\left( a\right) f\left( c\right) <0,$ si lo es, entonces $f$
tiene un cero en $\left[ a,c\right] .$ A continuaci\'{o}n se renombra $c$
como $b$ y se comienza una vez mas con elintervalo $\left[ a,b\right] ,$
cuya longitud es igual a la mitad de la longitud del intervalo original.

Si $f\left( a\right) f\left( c\right) >0,$ entoonces $f\left( a\right)
f\left( b\right) <0,$y en este caso se renombra $c$ como $a$ , en ambos
casos se ha generado un nuevo intervalo que contiene un cero de $f$ y el
proceso puede repetirse.Claro est\'{a} que si $f\left( a\right) f\left(
c\right) =0,$ entonces $f\left( c\right) =0$ y con ello se ha encontrado un
cero . Sin embargo por los errores de redondeo es poco factible que $f\left(
c\right) =0,$ as\'{i} el criterio para concluir no deber\'{a} depender de
que $\ f\left( c\right) $ sea cero . Se debe permitir una tolerancia
razonable tal como $f\left( c\right) <10^{-5}.$ Si hay varios ceros en el
intervalo dado, el m\'{e}todo de la bisecci\'{o}n encuentra uno cada vez.
Este m\'{e}todo tambi\'{e}n se conoce como el m\'{e}todo de la bipartici\'{o}%
n.

\bigskip 

A la hora de programar es conveniente contar con varios criterios que
detengan el programa, uno es m\'{a}ximo  n\'{u}mero de pasos que se
permitiran, esto reduce las posibilidades de que el programa se quede en un
ciclo infinito. Por otra parte la ejecuci\'{o}n del programa se puede
detener ya sea cuando el error sea lo suficientemente peque\~{n}o o cuando
lo sea el valor de $f\left( c\right) .$

\bigskip 

ANALISIS DE ERRORES

\bigskip Si $\left[ a_{0},b_{0}\right] ,\left[ a_{1},b_{1}\right] $, ...,$%
\left[ a_{n},b_{n}\right] $ son los intervalos que resultan del proceso, se
pueden hacer las siguientes observaciones:

\begin{eqnarray*}
a_{0} &\leqslant &a_{1}\leqslant a_{2}\leqslant ...\leqslant a_{n}\text{ }%
\left( 1\right)  \\
b_{0} &\geqslant &b_{1}\geqslant b_{2}\geqslant ...\geqslant b_{1}\text{ }%
\left( 2\right)  \\
b_{n+1}-a_{n+1} &=&\frac{1}{2}\left( b_{n}-a_{n}\right) \text{ \ \ }\left(
n\geqslant 0\right) \text{ }\left( 3\right) 
\end{eqnarray*}

La sucesi\'{o}n $\left[ a_{n}\right] $ converge debido a que es creciente y
est\'{a} acotada superiormente , la sucesi\'{o}n $\left[ b_{n}\right] $ tambi%
\'{e}n converge por razones analogas .Si se utiliza $\left( 3\right) $
repetidamente se llega a que :

\begin{equation*}
b_{n}-a_{n}=2^{-n}\left( b_{0}-a_{0}\right) 
\end{equation*}

As\'{i}

\begin{equation*}
\lim_{n\rightarrow \infty }b_{n}-\lim_{n\rightarrow \infty
}a_{n}=\lim_{n\rightarrow \infty }2^{n}\left( b_{0}-a_{0}\right) =0
\end{equation*}

Si se escribe

\begin{equation*}
r=\lim_{n\rightarrow \infty }b_{n}=\lim_{n\rightarrow \infty }a_{n}
\end{equation*}

entonces tomando l\'{i}mite en la desigualdad $0\geqslant f\left(
a_{n}\right) f\left( b_{n}\right) ,$ entonces se obtiene $\ 0\geqslant \left[
f\left( r\right) \right] ^{2},$ y por tanto $f\left( r\right) =0.$

\bigskip 

Sup\'{o}ngase que en cierta etapa del proceso \ se ha definido el intervalo $%
\left[ a_{n},b_{n}\right] .$ Si se detiene el proceso en este momento, la
raiz se encontrar\'{a} en este intervalo, en esta etapa la mejor estimaci%
\'{o}n de la raiz es el punto medio del intervalo.

El error se acota de la siguiente forma:

\begin{equation*}
\left| r-c_{n}\right| \leq \frac{1}{2}\left| b_{n}-a_{n}\right| \leq
2^{-\left( n+1\right) }\left( b_{0}-a_{0}\right) 
\end{equation*}

Resumiendo lo anterior \ se tiene el siguiente 

TEOREMA:

Si $\left[ a_{0},b_{0}\right] ,\left[ a_{1},b_{1}\right] $, ...,$\left[
a_{n},b_{n}\right] ,$..., denotan los intervalos en el m\'{e}todo de la
bisecci\'{o}n, entonces los l\'{i}mites $\lim_{n\rightarrow \infty }b_{n}$ y 
$\lim_{n\rightarrow \infty }a_{n},$ existen, son iguales y representan un
cero de $f.$ Si $r=\lim_{n\rightarrow \infty }c_{n}$ y $c_{n}=\frac{1}{2}%
\left( a_{n}+b_{n}\right) $, entonces:
\begin{verbatim}
\begin{equation*}
\left| r-c_{n}\right| \leq 2^{-\left( n+1\right) }\left( b_{0}-a_{0}\right) 
\end{equation*}
\end{verbatim}

En el siguiente ejemplo muestra un programa que resuelve una ecuaci\'{o}n
cuadr\'{a}tica introducida por un usuario.
\begin{verbatim}
c     **Este programa halla las raices de una ecuaci\UNICODE{0xa2}n cuadrtica por el
c     mtodo de la secante**
      program biseccion
      
c     **Declaraci\UNICODE{0xa2}n de variables**
 
      real G,H,I,D,R,u,v,w,x,a,b,c,e,dis
      integer n,j,resp,k
      D=0.00001
      R=0.00001
      n=300
      resp=1
      do while(resp.EQ.1)
        print*, 'Este programa resuelve una ecuaci\UNICODE{0xa2}n cuadrtica:'
        print*, 'ax**2 + bx  +c '
        Print*, 'Digite el coeficiente a'
        read*, G
        Print*, 'Digite el coeficiente b'
        read*, H
        Print*, 'Digite  c'
        read*, I
        dis=(H**2)-(4*G*I)
        if(dis.GT.0)then
         Print*, 'Digite los l\UNICODE{0xa1}mites del intervalo'
         read*, a
         read*, b
 
         u= (G*(a**2))+(H*a)+I
         v= (G*(b**2))+(H*b)+I
         if(u.EQ.0)then
          print*,'La raiz es',a
         else
          if(v.EQ.0)then
           print*,'La raiz es',b
          else
            k=1
            do while((v*u.GE.0).and.(k.LT.30))
             k=k+1
             a=a+0.1
             u= (G*(a**2))+(H*a)+I
             v= (G*(b**2))+(H*b)+I
            enddo
            e=b-a
            j=1
            if((u*v).LT.0)then
               do while(j.LT.n)
                 j=j+1
                 e=e/2
                 c=a+e
                 w=(G*(c**2))+(H*c)+I
 
                 if(abs(e).GE.D)then
                   if(abs(w).GE.R)then
                     if((u*w).LT.0)then
                      b=c
                      v=w
                     else
                      a=c
                      u=w
                     endif
                   else
                     j=n
                   endif
                 else
                 j=n
                 endif
               enddo
               print*, 'La raiz es',c
 
            else
               print*, 'No existen raices en el intervalo dado,'
               print*,'o el la longitud del intervalo es muy grande'
            endif
           endif
          endif
        else
         print*,'Esta ecuaci\UNICODE{0xa2}n no tiene raices reales'
        endif
        print*,'Si desea continuar digite 1, sino digite 0'
        read*,resp
      enddo
 
 
      end
\end{verbatim}


\end{document}
