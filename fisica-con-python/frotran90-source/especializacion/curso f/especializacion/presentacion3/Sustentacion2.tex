\documentclass[letter]{article}
\usepackage[pdftex]{graphicx}
\usepackage[ansinew]{inputenc}
\usepackage{amsmath}
\usepackage{amsbsy}
\usepackage{amssymb}
\usepackage[spanish]{babel}
\usepackage{float}
\usepackage{fixseminar}
\usepackage[display]{texpower}
\usepackage[ams]{pdfslide}
\usepackage{amsfonts}
\setcounter{MaxMatrixCols}{30} \pagestyle{title}
\begin{document}

\title{\color{black}Soluci\'{o}n del problema de \textit{Dirichlet} por medio de la integral de \textit{Poisson}.}
\author{\scalebox{1}[1.3]{\emph{\color{black}\textbf{Autor: Jorge E. Ospino Portillo} } }}


%\convenio{Convenio}
\orgname{\color{black}Universidad Nacional de Colombia\\
(Sede Medell\'{i}n)}
%\orgurl{\protect\color{black}http://www.unal.edu.co}

\udea{\color{black}Universidad del Norte\\
(Barranquilla)}

\director{\scalebox{1}[1.3]{\color{black}\emph{\textbf{Director:
Ph. D. Pedro Isaza Jaramillo.}} }}



%\notes{\emph{Esta presentaci�n fue desarrollada usando pdfslide class bajo Mik\TeX{}}.
%\newline {\tiny{Copyright\copyright 2001 Roque Lobo Torres. All
%rights reserved. \today{}}}}


\overlay{fondo.jpg}
\maketitle \pagedissolve{Wipe /D 3 /Dm /V
/M/O} \overlay{fondo.jpg}

\sffamily {\Large} \color{black} \headskip=20pt

{\section{{\protect\Large \color{black}{Contenido:}}}\pause}

\begin{itemize}
\item[$\checkmark$] {\Large \emph{Objetivo.}\pause}

\item[$\checkmark$] {\Large \emph{Introducci\'{o}n.}\pause}

\item[$\checkmark$] {\Large \emph{Existencia de una vecindad
tubular para $S$.}\pause}

\item[$\checkmark$] {\Large \emph{Forma generalizada de la segunda
identidad de \textit{Green}.}\pause}

\item[$\checkmark$] {\Large \emph{Existencia de derivadas normales
interiores.}\pause}

\item[$\checkmark$] {\Large \emph{Resultado.}\pause}


\end{itemize}

{\Large \headskip=20pt }

\section{{\protect\Large\textcolor[rgb]{1.00,1.00,1.00}{Objetivo}}}

{\Large \pagedissolve{Wipe /D 3 /Di /V /M /O} \overlay{fondo.jpg} }{\large
%\color{section2}
%\color{section1}
\color{white} }

\begin{itemize}
\item[$\checkmark$] {\Large \emph{\textbf{Objetivo:}\pause } }

{Probar usando herramientas de la teor\'{i}a cl\'{a}sica del
potencial que para un dominio regular $\Omega
\subset\mathbb{R}^{n}$ con frontera $S=\partial\Omega$ de clase
$C^{2}$, y $g\in C(S)$, la soluci\'{o}n $u\in C^{2}(\Omega)\cap
C(\overline{\Omega})$, del problema \pause}

{\begin{equation}
\left\{ \begin{array}{ccc}
\Delta u=0 & \mbox{en} & \Omega, \\
u=g & \mbox{en} & S,
\end{array} \right.
\end{equation} \pause}
{est\'{a} dada por
\begin{equation}
u(x)=\int_{S}\partial_{\nu_{y}}G(x,y)g(y)d\sigma(y),  x\in\Omega,
\end{equation}\pause}

\section{{\protect\Large\textcolor[rgb]{1.00,1.00,1.00}{Dominio regular}}}

\begin{itemize}
\item{$\Omega$ ser\'{a} un \textit{dominio regular} de clase
$C^{2}$ en $\mathbb{R}^{n}$ con $\,n\geq3$ $,$ es decir, un
subconjunto de $\mathbb{R}^{n}$ no vac\'{\i}o, abierto y acotado
de frontera $S,$ para el cual se cumplen las siguientes
propiedades: Para todo $x_{0}\in S$ existen una vecindad $U$ de
$x_{0}$ en $\mathbb{R}^{n}$ y una funci\'{o}n $\phi
:U\rightarrow\mathbb{R}$ de clase $C^{2}$ con
$\nabla\phi(x)\neq0,\,$para toda
$x\in S$ y tal que%
\[
S\cap U=\left\{  \left.  x\in U\right|  \,\phi(x)=0\right\}
\]
y%
\[
\Omega\cap U=\left\{  x\in U\left|  \,\phi(x)<0\right.  \right\} .
\]}
\end{itemize}

\section{{\protect\Large\textcolor[rgb]{1.00,1.00,1.00}{Introducci�n}}}

\item[$\checkmark$] {\Large \emph{\textbf{Introducci\'{o}n:}\pause
} }

 {Si supi\'{e}ramos que que $u$ no s\'{o}lo pertenece a
$C^{2}(\Omega)\cap C(\overline{\Omega})$ si no que adem\'{a}s
pertenece a $ C^{2}(\Omega)\cap C^{1}(\overline{\Omega})$ para
$x\in\Omega$ usando la segunda identidad de Green cl\'{a}sica, la
propiedad del valor medio para funciones arm\'{o}nicas y el
potencial newtoniano $N$ para tener \pause}
{\begin{equation}
u(x)=\int\limits_{S}\partial_{\nu_{y}}M(x,y)g(y)d\sigma(y)-\int\limits_{S}%
M(x,y)\partial_{\nu_{y}}u(y)d\sigma(y).%
\end{equation} \pause}

{$G:\Omega\times\overline{\Omega}\rightarrow\mathbb{R}$ que
permita obtener una f\'{o}rmula similar a $(3)$ con $G$ en lugar
de $M$ y con $G(x,y)=0$ para $y\in S$. As\'{i} se obtiene \pause}
{$$u(x)=\int_{S}\partial_{\nu_{y}}G(x,y)g(y)d\sigma(y),$$ \pause}

\section{{\protect\Large\textcolor[rgb]{1.00,1.00,1.00}{Vecindad}}}

\item[$\checkmark$] {\Large \emph{\textbf{Existencia de una
vecindad tubular para $S$:}}\pause}

\vspace{6mm}

{Sea $\Omega$ un dominio regular de clase $C^{2}$.
Entonces:\pause}

\vspace{4mm}

\begin{itemize}
\item[$\blacklozenge$] {Existen una vecindad $V$ de
$S=\partial\Omega$ en $\mathbb{R}^{n}$ y un $\epsilon>0$ tales que
la funci\'{o}n $F:S\times(-\epsilon,\epsilon)\rightarrow V$\
definida por $F(x,t)=x+t\nu (x)$\ es un homeomorfismo de clase
$C^{1}$.\pause}

\vspace{2mm}

\item[$\blacklozenge$] {Para $t\in(-\epsilon,0)$ $S_{t}:=\left\{
x+t\nu(x)\left|  \,x\in S\right.  \right\}  $\ es una superficie
de clase $C^{1},$\ y existe $\delta\in(0,\epsilon)$ tal que si
$t\in(-\delta,0)$, $S_{t}$ es la frontera de un dominio regular
$\Omega_{t}$ con $\overline{\Omega_{t}}\subset\Omega$.\pause}

\vspace{2mm}

\item[$\blacklozenge$] {Para $x\in S$ y $t\in(-\delta,0),$ si\
$x^{\prime}=x+t\nu(x)$, entonces el vector normal unitario
exterior a $S_{t}$\ en $x^{\prime}$\ es $\nu(x)$.\pause}

\end{itemize}

\section{{\protect\Large\textcolor[rgb]{1.00,1.00,1.00}{Vector normal}}}

\begin{itemize}
\item{\Large{Sea $x\in S$. Diremos que un vector\\
$\nu\in\mathbb{R}^{n}\left\backslash \left\{ 0\right\} \right. ,$
normal a $S$ en $x,$ es \textit{exterior,} si existe $\delta>0$
tal que $x+t\nu\in\Omega^{\prime}$, para toda $t\in(0,\delta)$ y $x+t\nu\in\Omega$, para toda $t\in(-\delta,0)$.\\
La funci\'{o}n $\nu:S\rightarrow\mathbb{R}^{n},$ es de clase
$C^{1}.$\\
El vector $\nu,$ normal a $S$ en $x,$ viene dado por
$\nu=\nabla\phi(x)$, donde $\phi$ es una funci\'{o}n que define a
$S$ en la vecindad de $x$ como en la definici\'{o}n de dominio
regular.}}
\end{itemize}

\section{{\protect\Large \textcolor[rgb]{1.00,1.00,1.00}{Segunda identidad de Green}}}

%\begin{itemize}
\item[$\checkmark$] {\Large \emph{\textbf{Segunda identidad de
\textit{Green} generalizada:}}\pause }

\vspace{6mm}

 {Si $u,\,v\in C^{2}(\Omega)\cap
C(\overline{\Omega}),\,\Delta u,\,\Delta v\in L^{2}(\Omega)\,$y\
$u\,$y\ $v$ tienen derivada normal interior en $S$ ,\ entonces \pause}%

{\begin{equation} \int_{\Omega}\left(  u\Delta v-v\Delta u\right)
\,dx=\int_{S} \left(  u\frac{\partial
v}{\partial\nu^{-}}-v\frac{\partial u}{\partial\nu^{-}}\right)
\,d\sigma.
\end{equation} \pause}

\section{{\protect\Large \textcolor[rgb]{1.00,1.00,1.00}{Primera identidad de Green}}}

\begin{itemize}
\item {\Large \textbf{Primera identidad de \textit{Green}
generalizada:\pause}}

\vspace{4mm}

{Si $S$ es de clase $C^{2}$, $u\in C^{2}(\Omega),\,\Delta u\in
L^{2}(\Omega),\,u$ tiene derivada normal interior en $S$ y $v \in
C^{1}(\Omega)\cap{C}(\overline{\Omega})$, entonces \pause}%

{\begin{equation} \int_{\Omega}v\Delta u\,dx=-\int_{\Omega}\nabla
v\cdot\nabla u\,dx+\int_{S}v\frac{\partial u}{\partial\nu^{-}%
}\,d\sigma.
\end{equation} \pause}

\end{itemize}


\section{{\protect\Large \textcolor[rgb]{1.00,1.00,1.00}{Neumann}}}

\begin{itemize}

\item{{\Large{Problema exterior de Neumann en $\Omega$
\begin{equation}
\left\{
\begin{array}
[c]{c}%
w\in C^{2}(\Omega^{\prime})\cap C(\overline{\Omega^{\prime}}),\\
\Delta w=0\,\mbox{en }\Omega^{\prime},\\
\frac{\partial w}{\partial\nu^{+}}=-g\mbox{ en }S,\\
w(x)\longrightarrow0\mbox{ si }\left\|  x\right\|
\rightarrow\infty,
\end{array}
\right.
\end{equation} \pause}}}
\end{itemize}
\newpage

\section{{\protect\Large \textcolor[rgb]{1.00,1.00,1.00}{Capa simple}}}
\begin{itemize}
\item{{Propiedades del potencial de capa simple asociado a
$\psi\in L^{\infty}(\Omega)$,
\[
w(x)=\int_{S}M(x,y)\psi(y)\,d\sigma(y),\,\,x\in \mathbb{R}^{n},
\] \pause}

{\begin{enumerate}

\item[i)] $\omega$ es arm\'{o}nica en $\mathbb{R}^{n}\setminus S$
y $\omega(x)\rightarrow0$ cuando $\left\|  x\right\|
\rightarrow\infty.$

\item[ii)] $\omega\in C(\mathbb{R}^{n}).$

\item[iii)] $\omega$ tiene derivadas normales interior y exterior
en $S$,
dadas por:%
$$\frac{\partial\omega(x)}{\partial\nu^{-}}=-\frac{1}{2}\psi(x)+\int
_{S}\partial_{\nu_{x}}M(y,x)\psi(y)d\sigma(y),\,\,x\in S,$$
$$\frac{\partial\omega(x)}{\partial\nu^{+}}=\frac{1}{2}\psi(x)+\int
_{S}\partial_{\nu_{x}}M(y,x)\psi(y)d\sigma(y),\,\ x\in S.$$
\end{enumerate}} \pause}
\end{itemize}

\section{{\protect\Large \textcolor[rgb]{1.00,1.00,1.00}{Derivadas normales}}}
%\begin{itemize}
\item[$\checkmark$] {\Large \emph{\textbf{Existencia de derivadas
normales interiores:}}\pause}

\vspace{6mm}
\begin{itemize}
\item{{Si $f\in C^{1}(\Omega)\cap C(\overline{\Omega})$ y
$$
u(x)=\int_{\Omega}G(x,y)f(y)\,dy,\, x\in\overline{\Omega},
$$\pause}
{donde $G$ es la funci\'{o}n de \textit{Green} para el operador
$\Delta,$ entonces $u$ tiene derivada normal interior en
$S$.\pause}}

\section{{\protect\Large \textcolor[rgb]{1.00,1.00,1.00}{Corolario 1}}}

\item{\Large{{Si $g\in C^{3}(\mathbb{R}^{n})$ y $u\in
C^{2}(\Omega)\cap
C(\overline{\Omega})$ es la soluci\'{o}n de \pause}%
{$$ \left\{
\begin{array}
[c]{c}%
\Delta u=0\mbox{ en }\Omega,\\
u=g\mbox{ en }S,
\end{array}
\right. $$ \pause}
{entonces $\frac{\partial u}{\partial\nu^{-}}$
existe en $S.$ \pause}}}

\end{itemize}

%\newpage

\section{{\protect\Large\textcolor[rgb]{1.00,1.00,1.00}{Resultado}}}
\item[$\checkmark$] {\Large \emph{\textbf{Resultado:}}\pause}

\vspace{10mm}

{Sea $\Omega$ en $\mathbb{R}^{n}$ un dominio regular de clase
$C^{2},$ con frontera $S$ y con $\Omega^{\prime}$ conexo; y sea
$g$ en $C(S)$. Entonces la soluci\'{o}n $u\in C^{2}(\Omega)\cap
C(\overline{\Omega})$ del problema de \textit{Dirichlet} \pause}%
{$$ \left\C
\begin{array}
[c]{c}%
\Delta u=0\mbox{ \ \ en \ }\Omega,\\
u=g\mbox{ \ \ en \ }S,

\right. $$ \pause}
{est\'{a} dada por%
$$
u(x)=\int\limits_{S}\partial_{\nu_{y}}G(x,y)g(y)d\sigma(y),\,\,\
x\in\Omega. $$ \pause}
\end{array}
\end{itemize}
%\newpage
\section{{\protect\Large \textcolor[rgb]{1.00,1.00,1.00}{Final}}}
{\Large{$\{g_{j}\}_{j}$ sucesi\'{o}n de funciones en
$C^{\infty}(\mathbb{R}^{n})$ que converja uniformemente a $g$ en
$S$.}\\\pause}

{\Large{Problema de Dirichlet cl\'{a}sico } \pause}

{\Large{$$
\left\{\begin{array}[c]{c}%
\Delta u_{j}=0\mbox{ \ \ en \ }\Omega,\\
u_{j}=g_{j}\mbox{ \ \ en \ }S.
\end{array}
\right. $$} \pause}

%Gr�fica
\headskip=20pt
\section{\textcolor[rgb]{1.00,1.00,1.00}{Figura 1}}
\pagedissolve{Wipe /D 1 /Di /H /M /O} \overlay{fondo.jpg}
\color{white}
\begin{figure}[H]
\begin{center}
\includegraphics[width=0.7\textwidth]{fig7.jpg}\newline
{\color{black}{figura 1.}}
\end{center}
\end{figure}

\end{document}
%%%%%%%%%%%%%%%%%%%%%%%%%%%%%%%%%%%%%%%%%%%%%%%%%%%%%%%%%%%%%%%%%%%%%%%%%%%%%%

%%%%%%%%%%%%%%%%%%%%%%%%%%%%%%%%%%%%%%%%%%%%%%%%%%%%%%%%%%%%%%%%%%
