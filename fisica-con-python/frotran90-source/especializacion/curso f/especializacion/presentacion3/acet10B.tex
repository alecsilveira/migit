
Sea $\Omega $ un dominio acotado en $\mathbb{R}^{n}$, $L$ el operador
diferencial definido en $\left( 9\right) $ y $\left( S_{\Omega }\right) $ la
condici\'{o}n:
\begin{equation*}
\left( S_{\Omega }\right) \left\{
\begin{array}{c}
\text{Existen constantes }\lambda _{0}\text{ y }\Lambda _{0}\text{ tales que:%
} \\
\lambda _{0}\in \xi ^{2}\leq \xi ^{T}a\left( x\right) \xi \leq \Lambda
_{0}\xi ^{2}\text{ para todo }\xi \in \mathbb{R}^{n}, \\
\left| b_{i}\left( x\right) \right| \leq \Lambda _{0}\text{ y }-\Lambda
_{0}\leq c\left( x\right) \text{ en }\Omega \text{.}
\end{array}
\right.
\end{equation*}
Adicionalmente se considera la siguiente definici\'{o}n.

\begin{definition}
Usando la notaci\'{o}n $d\left( x\right) =dist\left( x,\partial \Omega
\right) $, $x\in \Omega $, se dice que el conjunto $\Omega $ satisface la
\textbf{condici\'{o}n de bola interior uniforme en el sentido fuerte}, si
para alg\'{u}n $r>0$, y todo $x\in \Omega $ con $d\left( x\right) \leq r$ le
corresponde un punto m\'{a}s cercano $y\in \partial \Omega $, $d\left(
x\right) =\left| x-y\right| $,con la propiedad que $B_{r}\left( z\right)
\subset \Omega $, donde $z=y+\frac{r\left( x-y\right) }{\left| x-y\right| }$.
\end{definition}
