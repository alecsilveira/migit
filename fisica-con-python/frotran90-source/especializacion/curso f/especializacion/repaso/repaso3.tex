%% This document created by Scientific Word (R) Version 3.5

%TCIDATA{LaTeXparent=0,0,Est12.tex}
%TCIDATA{ChildDefaults=%
%chapter:2,page:75
%}


\section{Teor\'{\i}a de probabilidades}

\section{Experimentos y sucesos aleatorios}%

\begin{definition}
Diremos que un experimento es aleatorio si se veri\-fican las siguientes
condiciones,
\begin{enumerate}
\item Se puede repetir indefinidamente, siempre con las mismas condiciones.
\item Antes de realizarlo no se puede predecir el resultado.  \item El
resultado ''$e"$ que se obtiene pertenece a un conjunto de resultados posibles
conocido previamente el cual llamaremos espacio muestral y lo denotaremos con
la letra $\Omega\;o\;S$. Los elementos del espacio muestral se denominan
sucesos elementales.
\end{enumerate}
\end{definition} 

Es decir si $e_{1},e_{2}\in S\Longrightarrow e_{1},e_{2}$ son sucesos
elementales o puntos muestrales.En otras palabras: Un suceso es elemental si
su ocu\-rrencia o no ocurrencia no est\'{a} relacionada con ning\'{u}n otro suceso.

Cualquier subconjunto $A$ de $S$ se denomina suceso o evento aleatorio.

El espacio muestral puede ser de dos tipos:

\begin{itemize}
\item Discreto si est\'{a} formado por un conjunto finito o numerable de resultados.

\item Continuo si est\'{a} compuesto por un conjunto no numerable de elementos.
\end{itemize}%

\begin{definition}
[Suceso determinista]Se denomina experimento determinista a el experimento que
al realizarlo varias veces con las mismas condiciones iniciales obtenemos
siempre el mismo resultado
\end{definition} 

Cuando en un experimento no se puede predecir el resultado final, decimos que
el experimento es aleatorio

\section{Probabilidad}

\subsection{Probabilidad de laplace}

Si un experimento cualquiera se puede repetir obteniendo un n\'{u}mero finito
de resultados posibles y no existe ninguna raz\'{o}n para pensar que un
resultado tiene privilegios sobre otro, se calcula la probabi\-lidad del
suceso $A$ seg\'{u}n la regla de Laplace
\[
P\left(  A\right)  =\frac{n\acute{u}mero\,de\,casos\,favorables\,para\,A}%
{n\acute{u}mero\,de\,casos\,posibles}=\frac{n\left(  A\right)  }{n\left(
S\right)  }%
\]

\subsection{Definici\'{o}n axiom\'{a}tica de la probabilidad}

Como en toda rama de las matem\'{a}ticas debemos establecer una serie de
axiomas y definiciones b\'{a}sicas%

\begin{definition}
\textbf{Definici\'{o}n axiom\'{a}tica de probabilidad}\\[D]ado un espacio
muestral $S$, y una $\sigma-\acute{a}lgebra$ de sucesos $\mathcal{A}$ sobre
\'{e}l, diremos que $P$ es una probabilidad sobre $\mathcal{A}$ si cumple las
siguientes propiedades
\begin{description}
\item[A$_{1}$] La probabilidad es una funci\'{o}n definida sobre
$\ \mathcal{A}$, que toma solo valores positivos comprendidos entre 0 y 1, es
decir \[
\begin{array}
[c]{ccc}
P:\mathcal{A} & \rightarrow& [0,1]\subset IR\\ A\in\mathcal{A} & \mapsto&
1\geq P\left(   A\right)   \geq0
\end{array}
\]   \item[A$_{2}$] La probabilidad del suceso seguro es 1 \[ P\left(
S\right)   =1 \]   \item[A$_{3}$] Para cualquier sucesi\'{o}n infinita
$A_{1},A_{2},A_{3},\cdots$ de sucesos disjuntos de $\mathcal{A}$ se tiene que
la probabilidad de el evento $\bigcup_{i=1}^{\infty}A_{i}$ es la serie
infinita de las probabilidades, es decir
\end{description}
\end{definition} %

\[
P\left(  \bigcup_{i=1}^{\infty}A_{i}\right)  =\sum_{i=1}^{\infty}P\left(
A_{i}\right)
\]%

\begin{theorem}
$P\left(   \phi\right)   =0$
\end{theorem} %

\begin{theorem}
Para cualquier sucesi\'{o}n finita de $n$ eventos disjuntos $A_{1},A_{2}
,A_{3},\cdots A_{n}\in\mathcal{A}$ \[ P\left(   \bigcup_i=1^nA_i\right)
=\sum_i=1^nP\left(   A_i\right)  \]
\end{theorem} %

\begin{theorem}
para cualquier suceso $A\in\mathcal{A}$ se tiene que $P\left(   A^{\prime
}\right)   =1-P\left(   A\right)   $
\end{theorem} %

\begin{theorem}
Si $A,B\in\mathcal{A}$ y $A\subset B$, entonces $P\left(   B\right)   \geq
P\left(   A\right)   $
\end{theorem} 

\begin{center}
\end{center}%

\begin{theorem}
Para dos sucesos $A,B\subset\mathcal{A}$ cualesquiera \[ P\left(   A\cup
B\right)   =P\left(   A\right)   +P\left(   B\right)   -P\left(  A\cap
B\right)  \]
\end{theorem} 

\section{T\'{e}cnica para la enumeraci\'{o}n de puntos muestrales}%

\begin{theorem}
Consid\'{e}rese un experimento que tiene las dos cara\-cter\'{\i}sticas
siguientes
\end{theorem} 

\begin{itemize}
\item \textit{El experimento se realiza en dos partes }

\item \textit{La primera parte del experimento tiene }$m$ \textit{resultados
posibles }$x_{1},x_{2},x_{3},\cdots,x_{m}$ \textit{independientemente del
resultado} $x_{i}$ \textit{obtenido la segunda parte del experimento tiene
}$n$\textit{\ resultados posibles }$y_{1},y_{2},y_{3},\cdots,y_{n}%
$\textit{.\newline Cada resultado del espacio muestral S del experimento
ser\'{a} por tanto, un par de la forma }$\left(  x_{i},y_{j}\right)
$\textit{\ es decir }
\begin{align*}
S &  =\left\{  \left(  x_{i},y_{j}\right)  |i=1,2,3,\cdots m;j=1,2,3\cdots
n\right\}  \\
n\left(  S\right)   &  =mn
\end{align*}
Permutaciones
\end{itemize}%

\begin{definition}
Una permutaci\'{o}n es un arreglo de objetos distintos de tal manera que una
permutaci\'{o}n difiere de otra si el orden del arreglo o su contenido
difieren
\end{definition} 

Conviene observar que el orden es una caracter\'{\i}stica de especial
importancia en una permutaci\'{o}n. Cuando cambiamos el orden de los elementos
de este arreglo, se dice que permutamos dichos elementos

\subsection{Muestreo sin reemplazo}

Consideremos un experimento en el cual se selecciona un objeto de $n$ objetos
distintos, y luego se selecciona un segundo objeto de los $n-1$ objetos
restantes, y as\'{\i} sucesivamente hasta seleccionar el \'{u}ltimo objeto
Este proceso \ se llama muestreo sin reemplazo\ de acuerdo con el teorema
anterior los $n$ objetos se pueden seleccionar de $n\left(  n-1\right)
\left(  n-2\right)  \cdots3\cdot2\cdot1$ $=n!$ formas diferentes\ 

Ahora si no se escogen todos los objetos, si no $k$ objetos, los $k$ objetos
se pueden seleccionar
\begin{align*}
p_{n,k}  &  =n\left(  n-1\right)  \left(  n-2\right)  \cdots\left(
n-k+1\right)  =\frac{n!}{\left(  n-k\right)  !}\;\;r\leq n\\
0!  &  =1\\
1!  &  =1\\
p_{n,n}  &  =n\left(  n-1\right)  \left(  n-2\right)  \cdots1=n!
\end{align*}

\subsection{Combinaci\'{o}n}

Una combinaci\'{o}n es un arreglo de objetos distintos donde una
combinaci\'{o}n difiere de otra si difiere el contenido del arreglo. Si nos
interesa determinar el n\'{u}mero de combinaciones cuando en $n$ objetos
distintos deben seleccionarse $r$ a la vez entonces
\[
C_{n,r}=\frac{p_{n,r}}{r!}=\left(
\begin{array}
[c]{c}%
n\\
r
\end{array}
\right)  \;\;r\leq n
\]
ya que el numerador es el n\'{u}mero de permutaciones al escoger $r$ objetos
de $n$ posibles, pero hay que descontar los casos en que el orden determina
para la combinaci\'{o}n el mismo elemento, que es exactamente $r!.$

\subsubsection{}

\section{Probabilidad condicionada e independencia de eventos}%

\begin{definition}
Sean $A,B\in\mathcal{A}$ y sea $B$ un evento de probabilidad no nula para el
evento $A$, llamamos probabilidad condicionada de $A$ a $B$ a la cantidad que
representamos $P\left(   A|B\right)   $ y que definimos \[ P\left(
A|B\right)   =\frac{P\left(   AB\right)   }{P\left(   B\right)   }, \] la
cantidad $P\left(   A|B\right)   $ se lee la probabilidad de $A$ dada la
ocu\-rrencia de $B$
\end{definition} %

\begin{definition}
Sean $A,B\in\mathcal{A}$ dos eventos de probabilidad no nula se dice que son
independientes si y solo si \[ P\left(   AB\right)   =P\left(   A\right)
P\left(   B\right)  \]
\end{definition}%

\begin{theorem}
(Bayes) Sea $A_{1},A_{2},A_{3},\cdots A_{n}\in\mathcal{A}$ un sistema
exhaustivo y excluyente de eventos. Sea $B\subset\mathcal{A}$ un suceso del
que conocemos todas las cantidades \[ P\left(   B|A_i\right)   ,i=1,2,3,\cdots
,n \] a las que denominamos verosimilitudes, entones se verifica \[ \forall
j=1,2,3,\cdots,n,\qquad P\left(   A_j|B\right)   =\frac{P\left(
B|A_j\right)   P\left(   A_j\right)   }{\sum_i=1^nP\left(   B|A_i
\right)   P\left(   A_i\right)   }
\]
\end{theorem} 