\documentstyle{article}
\newlength{\defaultparindent}
\setlength{\defaultparindent}{\parindent}
 
%
%*****************************************************************%
\begin{document}




c     **Este programa halla las raices de una ecuaci?n cuadr?tica 
por el

c     m?todo de la secante**

      program biseccion

      

c     **Declaraci?n de variables**



      real G,H,I,D,R,u,v,w,a,b,c,e,dis

      integer n,j,resp,k

      D=0.00001

      R=0.00001

      n=300

      resp=1

      do while(resp.EQ.1)

        print*, 'Este programa resuelve una ecuaci?n cuadr?tica:'

        print*, 'ax**2 + bx  +c '

        Print*, 'Digite el coeficiente a'

        read*, G

        Print*, 'Digite el coeficiente b'

        read*, H

        Print*, 'Digite  c'

        read*, I

        dis=(H**2)-(4*G*I)

        if(dis.GT.0)then

         Print*, 'Digite el l?mite inferior del intervalo'

         read*, a

         Print*, 'Digite el l?mite superior del intervalo'

         read*, b

         write(*,*)'La funcion es : ',G,'x**2 ',H,'x ', I



         u= (G*(a**2))+(H*a)+I

         v= (G*(b**2))+(H*b)+I

         if(u.EQ.0)then

          print*,'La raiz es',a

         else

          if(v.EQ.0)then

           print*,'La raiz es',b

          else

            k=1

            do while((v*u.GE.0).and.(k.LT.30))

             k=k+1

             a=a+0.1

             u= (G*(a**2))+(H*a)+I

             v= (G*(b**2))+(H*b)+I

            enddo

            e=b-a

            j=1

            if((u*v).LT.0)then

               do while(j.LT.n)

                 j=j+1

                 e=e/2

                 c=a+e

                 w=(G*(c**2))+(H*c)+I



                 if(abs(e).GE.D)then

                   if(abs(w).GE.R)then

                     if((u*w).LT.0)then

                      b=c

                      v=w

                     else

                      a=c

                      u=w

                     endif

                   else

                     j=n

                   endif

                 else

                 j=n

                 endif

               enddo

               write(*,*)'*********************************************'

               print*, 'El numero de iteraciones es',j

               print*, 'La raiz es',c

               print*, 'El error es',e

               write(*,*)'*********************************************'

            else

               print*, 'No existen raices en el intervalo dado,'

               print*,'o el la longitud del intervalo es muy 
grande'

            endif

           endif

          endif

        else

         print*,'Esta ecuaci?n no tiene raices reales'

        endif

      open(10,file='biseccion.txt',status='unknown')

      write(10,201)'***************************************************'

      write(10,200)'La funcion es : ',G,'x**2 ',H,'x  ', I

      write(10,202) ' El numero de iteraciones es : ',j

      write(10,203) ' La raiz es: ',c

      write(10,204) ' El error es ',e

      write(10,205)'***************************************************'

201   format(a50)

200   format(a15,f8.6,a5,f8.3,a2,2f8.3)

202   format(a25,i4)

203   format(a15,f8.6)

204   format(a15,f8.6,a2)

205   format(a50)

        print*,'Si desea continuar digite 1, sino digite 0'

        read*,resp

      enddo





      end

\end{document}