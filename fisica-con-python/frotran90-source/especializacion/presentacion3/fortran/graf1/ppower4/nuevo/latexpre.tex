\documentclass[ps,eliz,slideColor,nocolorBG,accumulate]{prosper}
\bibliographystyle{plain}                                                          
\usepackage{pst-plot}
\usepackage[metapost]{mfpic}
\opengraphsfile{figs}

\newcommand{\codeA}[2]{{\bfseries\texttt{\textbackslash #1%
    #2}\index{#1@\texttt{\textbackslash #1}}}}        
                          
\email{ebrown@hsph.harvard.edu}

\title{Making Powerpoint-like Presentations with \LaTeX}
\author{Elizabeth Brown}
\setlength{\axisheadlen}{0pt}

\begin{document}
\maketitle

\overlays{7}{
\begin{slide}{Overview}
{\yellow Goal}: easily make presentation-quality slides \\
\FromSlide{2}{\yellow Options}: \FromSlide{3}
  \begin{itemize}
  \item PowerPoint \FromSlide{4}- what to do about equations?
\FromSlide{5}
  \item \LaTeX $\rightarrow$ pdf  -several packages will do this
    \begin{itemize}
\FromSlide{6}
    \item pdfslide or P$^4$ with pdflatex 
\FromSlide{7}
    \item \texttt{prosper} - most straight-forward; can easily convert existing \texttt{seminar} slides
    \end{itemize}
  \end{itemize}
\end{slide}
}
\overlays{5}{
\begin{slide}{Required packages for \texttt{prosper}}
\LaTeX 
  \begin{itemstep}
  \item \texttt{seminar} 
  \item \texttt{prosper} 
  \item \texttt{hyperref} 
  \item \texttt{pstricks} 
  \end{itemstep}
\FromSlide{5}
Other
\begin{itemize}
\item dvips
\item ps2pdf (ghostscript)
\item Adobe Acrobat Reader (acroread)
\end{itemize}
\end{slide}
}

\begin{slide}{}
\vspace{-1in}
\begin{center}
  \psframe[linestyle=none,fillstyle=solid,fillcolor=white](-2,-2)(12.5,11)
  \resizebox{!}{3.8in}{\includegraphics{prosper-structure.eps}}
\end{center}  
\end{slide}

%%% *********************************************************  %%%%
\begin{slide}[Glitter]{slides}
In seminar slides are constructed using
\begin{verbatim}
\begin{slide}
slide contents here
\end{slide}
\end{verbatim}
Since prosper uses seminar, slides are built the same way.  However, now we include a title.
\begin{verbatim}
\begin{slide}{title}
slide contents here
\end{slide}
\end{verbatim}
If the title is left out the letter s will become the title.
\end{slide}

%%% *********************************************************  %%%%
\begin{slide}[Glitter]
like this
\begin{verbatim}
\begin{slide}[Glitter]
like this
\end{slide}
\end{verbatim}
\end{slide}

\begin{slide}[Dissolve]{}
For no title, use
\begin{verbatim}
\begin{slide}[Dissolve]{}
For no title, use
\end{slide}
\end{verbatim}

\end{slide}
%%% *********************************************************  %%%%

\begin{slide}[Wipe]{Transitions}
\begin{itemize}
\item \texttt{Split}: two lines sweep across the screen 
\item \texttt{Blinds}: multiple lines appear and synchronously sweep in the same direction 
\item \texttt{Box}: a box sweeps from the center
\item \texttt{Wipe}: a single line sweeps across the screen from one
  edge to the other
\item \texttt{Dissolve}: the old page image dissolves 
\item \texttt{Glitter}: similar to \texttt{Dissolve}, except the
  effect sweeps across the image 
\item \texttt{Replace}: the effect is simply to replace the old page
  with the new page.
\end{itemize}      
\end{slide}

%%% *********************************************************  %%%%

\overlays{5}{
\begin{slide}{Overlays}
With overlays, you can
\begin{itemstep}
\item incrementally add items to a page
\item replace text or images on a slide in a smooth transition
\end{itemstep}
\FromSlide{3}
How it works
\begin{itemstep}
\item each overlay treated as separate page
\FromSlide{4} \item material is properly alligned from page to page to create illusion of adding material to slide
\FromSlide{5} \item indexed as only one slide
\end{itemstep}
\end{slide}
}
%%% *********************************************************  %%%%

\begin{slide}{Overlay commands}
Every slide with overlays must be preceded by \codeA{overlays}{\{n\}}
			and enclosed in brackets
\begin{verbatim}
\overlays{2}{
  \begin{slide}
    contents of slide
  \end{slide}
}
\end{verbatim}
\end{slide}
%%% *********************************************************  %%%%

\overlays{2}{
\begin{slide}{Overlay commands, cont.}
\begin{itemize}
\item \codeA{fromSlide}{\{p\}\{mat\}}. Puts \texttt{mat} on slides
  \texttt{p} through \texttt{n};
\item \codeA{onlySlide}{\{p\}\{mat\}}. Puts \texttt{mat} on slide \texttt{p}
  only;
\item \codeA{untilSlide}{\{p\}\{mat\}}. Puts \texttt{mat} on slides
  \texttt{1} through
  \texttt{p};
\FromSlide{2}
\item  \codeA{FromSlide}{\{p\}}. All the material following this statement will be put on slides \texttt{p} through \texttt{n};
\item \codeA{OnlySlide}{\{p\}}. All the material following this statement be put on slide \texttt{p} only;
\item \codeA{UntilSlide}{\{p\}}. All the material following this statement  will be put on slides \texttt{1} through
  \texttt{p}.
\end{itemize}                                                               
\end{slide}
}

%%% *********************************************************  %%%%
\begin{slide}[Dissolve]{Overlay commands, cont.}
\begin{itemize}
\item \codeA{fromSlide{\yellow *}}{\{p\}\{mat\}} Puts \texttt{mat} on slides
  \texttt{p} through \texttt{n};
\item \codeA{onlySlide{\yellow *}}{\{p\}\{mat\}}. Puts \texttt{mat} on slide \texttt{p}
  only;
\item \codeA{untilSlide{\yellow *}}{\{p\}\{mat\}}. Puts \texttt{mat} on slides
  \texttt{1} through
  \texttt{p};
\end{itemize}
Adding an {\yellow $*$} to these commands causes the material to be replaced by the material on the next overlay.
\end{slide}
%%% *********************************************************  %%%%

\overlays{4}{
\begin{slide}{Text overlay -- no replacement}
\begin{itemstep}
\item one
\item two
\item three
\item four
\end{itemstep}
\tiny
\begin{minipage}[b]{2in}
\textbackslash overlays \{ 4 \}\{ \\
\textbackslash begin \{slide\}\{Text overlay -- no replacement\}  \\
\textbackslash begin\{itemstep\}\\
\textbackslash item one\\
\textbackslash item two\\
\textbackslash item three\\
\textbackslash item four\\
\textbackslash end \{itemstep\}\\
\textbackslash end  \{slide\}\}\\
\end{minipage}
\hspace{.25in}
\begin{minipage}[b]{2in}
\textbackslash overlays \{ 4 \}\{ \\
\textbackslash begin \{slide\}\{Text overlay -- no replacement\}  \\
\textbackslash begin\{itemize\}\\
\textbackslash item one\\
\textbackslash FromSlide{2}
\textbackslash item two\\
\textbackslash FromSlide{3}
\textbackslash item three\\
\textbackslash FromSlide{4}
\textbackslash item four\\
\textbackslash end \{itemize\}\\
\textbackslash end  \{slide\}\}\\
\end{minipage}
\end{slide}}
%%% *********************************************************  %%%%

\overlays{3}{
\begin{slide}{Plot overlay example}
\begin{center}
\onlySlide*{1}{
\begin{mfpic}[50][300]{-3.2}{3.2}{0}{.5}
\pen{5pt}
\axes
\function{-3,3,0.1}{1/sqrt(2*pi)*exp(-.5*x*x)}
\lines{(-3,0),(3,0)}
\end{mfpic}}
\onlySlide*{2}{
\begin{mfpic}[50][300]{-3.2}{3.2}{0}{.5}
\pen{5pt}
\axes
\function{-3.2,3.2,0.1}{1/sqrt(2*pi)*exp(-.5*x*x)}
\gfill\btwnfcn{1.5,3,0.1}{1/sqrt(2*pi)*exp(-.5*x*x)}{0*x}
\lines{(-3,0),(3,0)}
\end{mfpic}}
\onlySlide*{3}{
\begin{mfpic}[50][300]{-3.2}{3.2}{0}{.5}
\pen{5pt}
\axes
\function{-3,3,0.1}{1/sqrt(2*pi)*exp(-.5*x*x)}
\lines{(-3,0),(3,0)}
\pen{2pt} \drawcolor{blue} \headcolor{blue}
\gfill\btwnfcn{1.5,3,0.1}{1/sqrt(2*pi)*exp(-.5*x*x)}{0*x}
\arrow[l10]% [r45]
\rotatepath{(1.7,.025),-120}\lines{(2.5,.02),(1.7,.025)}
\tlabel(0,-.12){\blue Rejection region}
\end{mfpic}\\
\begin{flushleft}
  \hyperlink{code}{{\green code}}
\end{flushleft}
}
\end{center}
\normalsize
\hypertarget{back}{ }
\end{slide}
}

\begin{slide}{\texttt{hyperref} package}
Use this package to creat links to help you move around the document and load webpages
\begin{description}
\item [\textbackslash href\{\textit{URL}\}\{text\}] creates a link to a web page
\item [\textbackslash hypertarget\{\textit{name}\}\{text\}] creates a target for a link in your document
\item [\textbackslash hyperlink\{\textit{name}\}\{text\}] creates a link to a place in the presentation defined by \textbackslash hypertarget
\end{description}
\end{slide}
\begin{slide}{\texttt{hyperref} examples}
Examples:
  \begin{itemize}
  \item 
    \href{http://www.biostat.harvard.edu/}{{\green Department homepage}}
\tiny \begin{verbatim}
    \href{http://www.biostat.harvard.edu/}
          {{\green Department homepage}}
\end{verbatim}
\normalsize
\item 
\href{http://hsph-dl.harvard.edu:8080/ramgen/lecture16/trainer.rm}{{\green open applications}}
\item citations\\
a book I'm reading~\cite{garland}
  \end{itemize}

\end{slide}

\begin{slide}{Making the presentation file}
\psframe[linestyle=none,fillstyle=solid,fillcolor=white](-.5,-1)(10.5,2.5)
\resizebox{4in}{!}{\includegraphics{compilation.eps}}

\end{slide}


\begin{slide}{More advanced topics}
  \begin{itemize}
  \item creating your own background
  \end{itemize}
\end{slide}

\begin{slide}{Helpful links}
\href{http://biosun1.harvard.edu/~ebrown/cwg6-6-01.html}{{\green Webpage for this talk.}}
\end{slide}

\begin{slide}{}
\bibliography{pre}
\Acrobatmenu{GoBack}{{\magenta \small Go Back}}
\end{slide}

%%%%   **********   EXTRA SLIDES  *********************  %%%%%%%%%%
\begin{slide}{code}
\hypertarget{code}{ }
\tiny\vspace{-.35in}
\begin{verbatim}
\overlays{3}{
\begin{slide}{Plot overlay example}
\begin{center}
\onlySlide*{1}{
mfpic commands}
\onlySlide*{2}{
mfpic commands}
\onlySlide*{3}{
\begin{mfpic}[50][300]{-3.2}{3.2}{0}{.5}
\pen{5pt}
\axes
\function{-3,3,0.1}{1/2/sqrt(pi)*exp(-.5*x*x)}
\lines{(-3,0),(3,0)}
\pen{2pt} \drawcolor{blue} \headcolor{blue}
\gfill\btwnfcn{1.5,3,0.1}{1/sqrt(2*pi)*exp(-.5*x*x)}{0*x}
\arrow[l10]
\rotatepath{(1.7,.025),-120}\lines{(2.5,.02),(1.7,.025)}
\tlabel(0,-.12){\blue Rejection region}
\end{mfpic}} \end{center} \end{slide} }
\end{verbatim}
\hyperlink{back}{{\green back}}
\end{slide}

\end{document}
