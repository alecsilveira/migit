
\documentclass[12pt]{article}
%%%%%%%%%%%%%%%%%%%%%%%%%%%%%%%%%%%%%%%%%%%%%%%%%%%%%%%%%%%%%%%%%%%%%%%%%%%%%%%%%%%%%%%%%%%%%%%%%%%%%%%%%%%%%%%%%%%%%%%%%%%%
\usepackage{graphicx}
\usepackage{amsmath}
\usepackage{zahlen}

%TCIDATA{OutputFilter=LATEX.DLL}
%TCIDATA{Created=Thu Nov 01 09:02:45 2001}
%TCIDATA{LastRevised=Mon Sep 29 10:11:12 2003}
%TCIDATA{<META NAME="GraphicsSave" CONTENT="32">}
%TCIDATA{<META NAME="DocumentShell" CONTENT="Journal Articles\Standard LaTeX Article">}
%TCIDATA{CSTFile=LaTeX article (bright).cst}

\newtheorem{theorem}{Theorem}
\newtheorem{acknowledgement}[theorem]{Acknowledgement}
\newtheorem{algorithm}[theorem]{Algorithm}
\newtheorem{axiom}[theorem]{Axiom}
\newtheorem{case}[theorem]{Case}
\newtheorem{claim}[theorem]{Claim}
\newtheorem{conclusion}[theorem]{Conclusion}
\newtheorem{condition}[theorem]{Condition}
\newtheorem{conjecture}[theorem]{Conjecture}
\newtheorem{corollary}[theorem]{Corollary}
\newtheorem{criterion}[theorem]{Criterion}
\newtheorem{definition}[theorem]{Definition}
\newtheorem{example}[theorem]{Example}
\newtheorem{exercise}[theorem]{Exercise}
\newtheorem{lemma}[theorem]{Lemma}
\newtheorem{notation}[theorem]{Notation}
\newtheorem{problem}[theorem]{Problem}
\newtheorem{proposition}[theorem]{Proposition}
\newtheorem{remark}[theorem]{Remark}
\newtheorem{solution}[theorem]{Solution}
\newtheorem{summary}[theorem]{Summary}
\newenvironment{proof}[1][Proof]{\textbf{#1.} }{\ \rule{0.5em}{0.5em}}
\input{tcilatex}

\begin{document}


\section{AN\'{A}LISIS GR\'{A}FICO}

El \ an\'{a}lisis gr\'{a}fico es un m\'{e}todo que nos sirve para determinar
la relaci\'{o}n matem\'{a}tica entre dos o m\'{a}s variables, ( en nuestro
caso cantidades f\'{i}sicas) partiendo de una gr\'{a}fica realizada a partir
de datos experimentales tomados al azar. Existen programas computacionales
que con una serie de datos pueden encontrar la curva que mejor su ajusta a
los datos y adem\'{a}s la ecuaci\'{o}n que la caracteriza, mediante
regresiones lineales o polin\'{o}micas

El estudio estudio de este tema lo limitaremos \ a la elaboraci\'{o}n,
obtenci\'{o}n y an\'{a}lisis de gr\'{a}ficas utilizando las herramientas que
ofrece el programa \textbf{Data Studio De Pasco}. Con este pr\'{o}posito
dividiremos el tema entres partes:

\textbf{1.) }Edici\'{o}n de tablas \ y trazado de la gr\'{a}fica, \textbf{%
2.) }Escogencia del modelo matam\'{a}tico y ajuste de curva. \textbf{3.) }%
Establecimiento de la relaci\'{o}n matem\'{a}tica.

\subsection{Edici\'{o}n de tablas y trazado de la gr\'{a}fica}

Generalmente en matem\'{a}ticas nos presentan una relaci\'{o}n metem\'{a}%
tica a partir de la cual obtenemos una taba de datos (coordenadas) y luego
trazamos la gr\'{a}fica, pero la naturaleza en cambio nos presenta un fen%
\'{o}meno, no una relaci\'{o}n, por lo que tenemos que medir las magnitudes f%
\'{i}sicas que describen el fen\'{o}meno f\'{i}sico, obteniendo as\'{i} una
tabla de datos, la cual la podemos introducir en el Data studio (o cualquier
otro progama que trabaje con bases de datos) de dos maneras:

Usando un sensor o editando una nueva tabla de datos.

En el caso de utilizar un sensor, este transforma su lectura en datos num%
\'{e}ricos usando la interface como puente entre el sensor, el computador y
el software, lo que quiere decir que los datos son introducidos
automaticamente.

El procedimiento computacional es el siguiente :

a. Haga click en el icono de table, observar\'{a} algo parecido a lo que
muestra la figura 1

\textbf{b. }En la barra de men\'{u} est\'{a}ndar se\~{n}ale experiment y
haga click en new empty data table, como indica la figura 2.

luego aparece lo que observa \ en la figura 3

para poder escribir tiene que activar el l\'{a}piz y desactivar el candado
que aparece en la barra de herramientas del display. Usando tab $%
\overleftarrow{\rightarrow }$ en el teclado puede cambiar de celda para
introducir dato por dato hasta obtener algo parecido a

\textbf{c. }En caso de que se equivoque se puede borrar como en cualquier
programa, pero si se da cuenta despu\'{e}s del error, hay que se\~{n}alar
toda la fila , hace click con el bot\'{o}n derecho del mouse y escoga remove.

Para editar una tabla en Data Studio haga lo siguiente 

\textbf{i. }Cuando digite los datos observe que en la parte superior derecha
de la ventana aparecieron los iconos que muestra la figura 5.

haga click sobre el l\'{a}piz, luego aparece la ventana, 

\textbf{ii. }haga doble click en en YVariable y XVariable para colocar en
label el nombre de la variable (colocarlos en \'{i}ngles si est\'{a}
trabajando tambi\'{e}n con sensores), en units se colocan las unidades en el
S.I

\textbf{iii. }El titulo se coloca en Measurement label, en los campos
accuracy se coloca la precisi\'{o}n y en display precision se coloca el n%
\'{u}mero de cifras decimales.

\textbf{d.}Haciendo doble click sobre el tri\'{a}ngulo o la figura que
aparece debajo del l\'{a}piz (si no aparece haga click sobre el signo mas
que aparece junto al l\'{a}piz ) y manteni\'{e}ndolo arrastralo hasta graph
como indica la gr\'{a}fica.

para obtener

En el caso que usemos un sensor los pasos son los mismos, pero trabajamos
con el icono 

\subsection{Modelos matem\'{a}ticos}

Conociendo la forma de la curva podemos suponer un modelo matem\'{a}tico, en
esta secci\'{o}n presentaremos los cuatro m\'{a}s usados

\begin{enumerate}
\item  $y=ax^{m}\;\;a\in \rz
,\;m,a\neq 0,\;m\in \qz ^{+}$

\begin{enumerate}
\item  Se le llama Potencial si tiene la forma \FRAME{dtbpFU}{3in}{2.0003in}{%
0pt}{\Qcb{$m\in \gz ^{+}\;m\neq 1$}}{}{}{\special{language "Scientific
Word";type "MAPLEPLOT";width 3in;height 2.0003in;depth 0pt;display
"PICT";plot_snapshots TRUE;function \TEXUX{$x^{3}$};linecolor
"black";linestyle 1;linethickness 1;pointstyle "point";xmin "-5";xmax
"5";xviewmin "0.001";xviewmax "5.000000";yviewmin "-0.001";yviewmax
"10.100000";viewset"XY";rangeset"X";phi 45;theta 45;plottype
4;plottickdisable TRUE;numpoints 49;axesstyle "normal";xis
\TEXUX{x};var1name \TEXUX{$x$};valid_file "T";tempfilename
'GSM0IL00.wmf';tempfile-properties "PR";}}

\item  Radical\FRAME{dtbpFU}{3in}{2.0003in}{0pt}{\Qcb{$m\in \qz ^ {+}- \gz %
^{+}$}}{}{}{\special{language "Scientific Word";type "MAPLEPLOT";width
3in;height 2.0003in;depth 0pt;display "PICT";plot_snapshots TRUE;function
\TEXUX{$\sqrt{x}$};linecolor "black";linestyle 1;linethickness 1;pointstyle
"point";xmin "-5";xmax "5";xviewmin "0.001";xviewmax "5.000000";yviewmin
"0.428825";yviewmax "2.272213";viewset"XY";rangeset"X";phi 45;theta
45;plottype 4;numpoints 49;axesstyle "normal";xis \TEXUX{x};var1name
\TEXUX{$x$};valid_file "T";tempfilename 'GSM0UK01.wmf';tempfile-properties
"PR";}}

\item  Inverso \FRAME{dtbpFU}{3in}{2.0003in}{0pt}{\Qcb{$m\in \gz ^{-}$}}{}{}{%
\special{language "Scientific Word";type "MAPLEPLOT";width 3in;height
2.0003in;depth 0pt;display "PICT";plot_snapshots TRUE;function
\TEXUX{$x^{-1}$};linecolor "black";linestyle 1;linethickness 1;pointstyle
"point";xmin "-5";xmax "5";xviewmin "0.01";xviewmax "5.000000";yviewmin
"0.001";yviewmax "5.000";viewset"XY";rangeset"X";phi 45;theta 45;plottype
4;numpoints 49;axesstyle "normal";xis \TEXUX{x};var1name
\TEXUX{$x$};valid_file "T";tempfilename 'GSM10Y02.wmf';tempfile-properties
"PR";}}

\item  Lineal \FRAME{dhFU}{3in}{2.0003in}{0pt}{\Qcb{$m=1$}}{}{}{\special%
{language "Scientific Word";type "MAPLEPLOT";width 3in;height 2.0003in;depth
0pt;display "PICT";plot_snapshots TRUE;function \TEXUX{$3x$};linecolor
"black";linestyle 1;linethickness 1;pointstyle "point";xmin "-5";xmax
"5";xviewmin "0.001";xviewmax "5.000000";yviewmin "0.000000";yviewmax
"15.612000";viewset"XY";rangeset"X";phi 45;theta 45;plottype 4;numpoints
49;axesstyle "normal";xis \TEXUX{x};var1name \TEXUX{$x$};valid_file
"T";tempfilename 'GSM16G03.wmf';tempfile-properties "PR";}}
\end{enumerate}

\item  Se llama modelo exponencial a $y=ae^{mx},\;\;a,m\in \rz,\;a,m\neq 0$%
\newline
$
\begin{tabular}{ll}
\FRAME{itbpF}{4.6063cm}{5.0742cm}{0cm}{}{}{}{\special{language "Scientific
Word";type "MAPLEPLOT";width 4.6063cm;height 5.0742cm;depth 0cm;display
"USEDEF";plot_snapshots TRUE;function \TEXUX{$e^{x}$};linecolor
"black";linestyle 1;linethickness 1;pointstyle "point";xmin "-5";xmax
"5";xviewmin "0.000012";xviewmax "5.000000";yviewmin "1.000";yviewmax
"3.440650";viewset"XY";rangeset"X";phi 45;theta 45;plottype 4;numpoints
49;axesstyle "normal";xis \TEXUX{x};var1name \TEXUX{$x$};valid_file
"T";tempfilename 'GSM1OR09.wmf';tempfile-properties "PR";}} & \FRAME{itbpF}{%
4.6063cm}{5.0742cm}{0cm}{}{}{}{\special{language "Scientific Word";type
"MAPLEPLOT";width 4.6063cm;height 5.0742cm;depth 0cm;display
"USEDEF";plot_snapshots TRUE;function \TEXUX{$e^{-x}$};linecolor
"black";linestyle 1;linethickness 1;pointstyle "point";xmin "-5";xmax
"5";xviewmin "0.000000";xviewmax "5.000000";yviewmin "0.004";yviewmax
"1.440650";viewset"XY";rangeset"X";phi 45;theta 45;plottype 4;numpoints
49;axesstyle "normal";xis \TEXUX{x};var1name \TEXUX{$x$};valid_file
"T";tempfilename 'GSM1PL0A.wmf';tempfile-properties "PR";}}
\end{tabular}
$

\item  Modelo logar\'{i}tmico $y=\frac{1}{m}\ln x\;m\in \rz,\;m\neq 0$\FRAME{%
dhF}{3in}{2.0003in}{0pt}{}{}{}{\special{language "Scientific Word";type
"MAPLEPLOT";width 3in;height 2.0003in;depth 0pt;display
"USEDEF";plot_snapshots TRUE;function \TEXUX{$\log x$};linecolor
"black";linestyle 1;linethickness 1;pointstyle "point";xmin "-5";xmax
"5";xviewmin "1.000000";xviewmax "5.000000";yviewmin "0.001";yviewmax
"2.018052";viewset"XY";rangeset"X";phi 45;theta 45;plottype 4;numpoints
49;axesstyle "normal";xis \TEXUX{x};var1name \TEXUX{$x$};valid_file
"T";tempfilename 'GSM1VJ0B.wmf';tempfile-properties "PR";}}

\item  Modelo lineal $y=mx+b\;\;m,b\in \rz,\;m\neq 0$\FRAME{dhF}{3in}{%
2.0003in}{0pt}{}{}{}{\special{language "Scientific Word";type
"MAPLEPLOT";width 3in;height 2.0003in;depth 0pt;display
"USEDEF";plot_snapshots TRUE;function \TEXUX{$x+3$};linecolor
"black";linestyle 1;linethickness 1;pointstyle "point";xmin "-5";xmax
"5";xviewmin "-5.000000";xviewmax "5.000000";yviewmin "-2.200000";yviewmax
"8.204000";phi 45;theta 45;plottype 4;numpoints 49;axesstyle "normal";xis
\TEXUX{x};var1name \TEXUX{$x$};valid_file "T";tempfilename
'GSM1ZT0C.wmf';tempfile-properties "PR";}}
\end{enumerate}

\subsection{Ajuste de la curva}

Al observar las gr\'{a}ficas notamos que muchas se parecen y a veces es
d\'{i}ficil estar seguro si el modelo que escogemos es adecuado, es m\'{a}s
no conocemos todos los modelos. en esta secci\'{o}n explicaremos dos
m\'{e}todos para ayudarnos a escoger un modelo que se aproxime de buena
forma a los datos.

\begin{enumerate}
\item  Linealizaci\'{o}n :\newline
Este m\'{e}todo consiste en encontrar dos relaciones $h\left( y\right) $ y $%
f\left( x\right) $ tales que al graficar $h\left( y\right) $ vs $f\left(
x\right) $ se obtenga una linea recta, es decir si esto sucede el modelo es
satisfactorio.\newline
Por ejemplo si tenemos los datos 
\begin{equation*}
\begin{tabular}{|l|l|l|l|l|l|}
\hline
$y\left( cm\right) $ & 1 & 4 & 9 & 16 & 25 \\ \hline
$x(s)$ & 1 & 2 & 3 & 4 & 5 \\ \hline
\end{tabular}
\end{equation*}
\FRAME{dhF}{2.4111in}{1.8343in}{0pt}{}{}{potencial.wmf}{\special{language
"Scientific Word";type "GRAPHIC";maintain-aspect-ratio TRUE;display
"PICT";valid_file "F";width 2.4111in;height 1.8343in;depth
0pt;original-width 3.0415in;original-height 2.3056in;cropleft "0";croptop
"1";cropright "1";cropbottom "0";filename 'potencial.WMF';file-properties
"NPEU";}}De acuerdo con la forma el modelo es $y=ax^{m}$, pero si analizamos
los datos ellos nos hacen sospechar que $m=2,$ por lo que podr\'{i}amos
hacer $h\left( y\right) =y$ y $f\left( x\right) =x^{2}$ de lo que obtenemos
la siguiente tabla 
\begin{equation*}
\begin{tabular}{llllll}
$h\left( y\right) $ & 1 & 4 & 9 & 16 & 25 \\ 
$f\left( x\right) $ & 1 & 4 & 9 & 16 & 25
\end{tabular}
\end{equation*}
\FRAME{dhF}{3.3088in}{2.2762in}{0pt}{}{}{cuadra.wmf}{\special{language
"Scientific Word";type "GRAPHIC";maintain-aspect-ratio TRUE;display
"PICT";valid_file "F";width 3.3088in;height 2.2762in;depth
0pt;original-width 3.2638in;original-height 2.2364in;cropleft "0";croptop
"1";cropright "1";cropbottom "0";filename 'cuadra.WMF';file-properties
"NPEU";}}Como se obtuvo una recta podemos decir que el modelo es aceptable y
al determinar la ecuaci\'{o}n de la recta nos que al tomar dos puntos
cualesquiera 
\begin{equation*}
a=\frac{y_{1}-y_{2}}{x_{1}^{2}-x_{2}^{2}}=\frac{16-9}{16-9}=1
\end{equation*}
como la relaci\'{o}n matem\'{a}tica es $h\left( y\right) =af\left( x\right)
, $ entonces queda $y=x^{2}.$\newline
Este m\'{e}todo no es pr\'{a}ctico en el sentido de que si no sospechamos
nada acerca de la relaci\'{o}matem\'{a}tica es casi imposible determinar $%
h\left( y\right) $ y $f\left( x\right) ,$ afortunadamente para los modelos
planteados ya existen estas relaciones

\begin{enumerate}
\item  Para el modelo $y=ax^{m}$ son 
\begin{equation*}
h\left( y\right) =\log y,\;f\left( x\right) =\log x
\end{equation*}
al obtener la recta \FRAME{dtbpFU}{3.435in}{2.2485in}{0pt}{\Qcb{$\log
Y=m\log x+b,a=10^{b}$}}{}{loglog.wmf}{\special{language "Scientific
Word";type "GRAPHIC";maintain-aspect-ratio TRUE;display "PICT";valid_file
"F";width 3.435in;height 2.2485in;depth 0pt;original-width
3.3892in;original-height 2.2087in;cropleft "0";croptop "1";cropright
"1";cropbottom "0";filename 'loglog.WMF';file-properties "NPEU";}}podemos
encontrar 
\begin{eqnarray*}
m &=&\frac{\log y_{i}-\log y_{j}}{\log x_{i}-\log x_{j}} \\
a &=&\frac{y_{i}}{x_{i}^{m}}.
\end{eqnarray*}
En la pr\'{a}ctica debido a los errores al tomar diferentes parejas $\left(
x_{i},y_{i}\right) $ los valores de $m$ y $a$ no son constantes, por lo que
hay que determinarlos varias veces y promediar los resultados, as\'{i} la
relaci\'{o}n matem\'{a}tica queda $y=\bar{a}x^{\bar{m}}$

\item  Para el modelo $y=ae^{mx}$ se escogen las relaciones 
\begin{equation*}
h\left( y\right) =\ln y,\;f\left( x\right) =x
\end{equation*}
y si obtenemos la recta \FRAME{dhFU}{3.435in}{2.2485in}{0pt}{\Qcb{$\ln
Y=mX+b;\;a=e^{b}$}}{}{semilog.wmf}{\special{language "Scientific Word";type
"GRAPHIC";maintain-aspect-ratio TRUE;display "PICT";valid_file "F";width
3.435in;height 2.2485in;depth 0pt;original-width 3.3892in;original-height
2.2087in;cropleft "0";croptop "1";cropright "1";cropbottom "0";filename
'semilog.WMF';file-properties "NPEU";}}podemos utilizar 
\begin{eqnarray*}
m &=&\frac{\ln y_{i}-\ln y_{j}}{x_{i}-x_{j}} \\
a &=&\frac{y_{i}}{e^{mx_{i}}}.
\end{eqnarray*}
y calculamos varios valores de $m$ y $a$ para promediarlos y obtener la
relaci\'{o}n matem\'{a}tica $y=\bar{a}e^{\bar{m}x}$

\item  Si el modelo es $y=\frac{1}{m}\ln x$ las relaciones son 
\begin{equation*}
h\left( y\right) =y,\;f\left( x\right) =\ln x
\end{equation*}
si la gr\'{a}fica $f\left( x\right) \;vs\;h\left( y\right) $ es lineal como
se muestra en la figura \FRAME{dhF}{2.6593in}{1.8334in}{0pt}{}{}{logm.wmf}{%
\special{language "Scientific Word";type "GRAPHIC";maintain-aspect-ratio
TRUE;display "PICT";valid_file "F";width 2.6593in;height 1.8334in;depth
0pt;original-width 6.1531in;original-height 4.222in;cropleft "0";croptop
"1";cropright "1";cropbottom "0";filename 'logm.WMF';file-properties "NPEU";}%
}podemos usar 
\begin{equation*}
m=\frac{\ln x_{i}-\ln x_{j}}{y_{i}-y_{j}}
\end{equation*}
calculamos $m$ varias veces y obtenemos $y=\frac{1}{\bar{m}}\ln x.$

\item  El modelo lineal es trivial por lo que dejamos de tarea al lector
\end{enumerate}

\item  Ajuste lineal\newline
El ajuste lineal es un m\'{e}todo estad\'{i}stico, parecido al anterior,
pero usando herramientas diferentes.\newline
En este caso no explicaremos mucho , s\'{o}lo plantearemos algunas
f\'{o}rmulas e interpretaremos los resultados \newline
Si tenemos dos conjuntos de datos 
\begin{eqnarray*}
&&x_{1},x_{2},x_{3},\cdots ,x_{n} \\
&&y_{1},y_{2},y_{3},\cdots ,y_{n}
\end{eqnarray*}
definiremos la

\begin{enumerate}
\item  Media aritm\'{e}tica 
\begin{eqnarray*}
\bar{x} &=&\frac{\sum_{i=1}^{n}x_{i}}{n} \\
\bar{y} &=&\frac{\sum_{i=1}^{n}y_{i}}{n}
\end{eqnarray*}

\item  Varianza 
\begin{eqnarray*}
S_{xx}^{2} &=&\frac{\sum_{i=1}^{n}(x_{i}-\bar{x})^{2}}{n-1} \\
S_{yy}^{2} &=&\frac{\sum_{i=1}^{n}(y_{i}-\bar{y})^{2}}{n-1}
\end{eqnarray*}

\item  Covarianza 
\begin{equation*}
S_{xy}=\frac{\sum_{i=1}^{n}(x_{i}-\bar{x})(y_{i}-\bar{y})}{n-1}
\end{equation*}

\item  Coeficiente de correlaci\'{o}n 
\begin{equation*}
r=\frac{S_{xy}}{\sqrt{S_{xx}^{2}S_{yy}^{2}}}
\end{equation*}
Cuando estudiamos dos variables $X\;$y $Y$, en realidad estas variables
estas variables tienen desde el punto de vista geom\'{e}trico el
comportamiento de dos vectores (rayos con direcci\'{o}n). Y se dice que
existe un ajuste lineal si los vectores representados por las desviaciones
est\'{a}n alineados o son paralelos.\newline
Para determinar si esto sucede podemos utilizar la trigonometr\'{i}a para
determinar el \'{a}ngulo entre los vectores representados en la figura 
\FRAME{dhF}{1.8152in}{1.4659in}{0pt}{}{}{veca.wmf}{\special{language
"Scientific Word";type "GRAPHIC";maintain-aspect-ratio TRUE;display
"PICT";valid_file "F";width 1.8152in;height 1.4659in;depth
0pt;original-width 1.7781in;original-height 1.4304in;cropleft "0";croptop
"1";cropright "1";cropbottom "0";filename 'veca.WMF';file-properties "NPEU";}%
}
\end{enumerate}
\end{enumerate}

Utilizando una rama de las matem\'{a}ticas llamada algebra lineal se puede
comprobar que si $\theta $ es el \'{a}ngulo entre los vectores entonces 
\begin{equation*}
\cos \theta =r
\end{equation*}
Es decir este coeficiente determina la relaci\'{o}n lineal entre las
desviaciones de las variables y esta dependencia es perfecta cuando $r=\pm 1$
existe el ajuste lineal en la practica esto no sucede debido a los errores
pero decimos que el ajuste es aceptable si $r^{2}>0.95.\;$\newline
Si el ajuste existe deben existir dos relaciones $h\left( y\right) $ y $%
f\left( x\right) $ tal que el modelo 
\begin{equation*}
h\left( y\right) =mf\left( x\right) +b
\end{equation*}
debe tener un $r$ adecuado \ y se puede calcular 
\begin{equation*}
r=\frac{n\sum_{i.j=1}^{n}f\left( x_{i}\right) h\left( y_{j}\right)
-\sum_{i=1}^{n}f\left( x_{i}\right) \sum_{j=1}^{n}h\left( y_{i}\right) }{%
\sqrt{\left[ n\sum_{i=1}^{n}f^{2}\left( x_{i}\right) -\left(
\sum_{i=1}^{2}f\left( x_{i}\right) ^{2}\right) \right] }\left[
n\sum_{j=1}^{n}h^{2}\left( y_{j}\right) -\left( \sum_{j=1}^{2}h\left(
y_{j}\right) ^{2}\right) \right] }
\end{equation*}

\begin{eqnarray*}
m &=&\frac{n\sum_{i.j=1}^{n}f\left( x_{i}\right) h\left( y_{j}\right)
-\sum_{i=1}^{n}f\left( x_{i}\right) \sum_{j=1}^{n}h\left( y_{i}\right) }{%
n\sum_{i=1}^{n}f^{2}\left( x_{i}\right) -\left( \sum_{i=1}^{2}f\left(
x_{i}\right) ^{2}\right) } \\
b &=&\frac{\sum_{j=1}^{n}h\left( y_{j}\right) -m\sum_{i=1}^{n}f\left(
x_{i}\right) }{n}
\end{eqnarray*}
donde \ $a$ se obtiene

\begin{itemize}
\item  $a=10^{b}$ en el modelo $y=ax^{m}$

\item  $a=e^{b}$ en el modelo $y=ae^{mx}$
\end{itemize}

\subsection{Interpretaci\'{o}n geom\'{e}trica de la covarianza}

Si consideramos una nube de puntos formados por las parejas de los datos
concretos de dos variables $X$ e $Y$ $\left( x_{i},y_{i}\right) $ el centro
de gravedad de esta nube de puntos es $\left( \overset{-}{x},\overset{-}{y}%
\right) $, ahora si trasladamos los ejes de tal forma que este punto sea el
centro, la nube queda dividida en cuatro cuadrantes los que indica que los
puntos que se encuentran en el pimer y tercer cuadrate contribuyen
positivamente al valor de la covarianza y los que se encuentran en los otros
dos cuadrantes contribuyen negativamente. como lo indica la figura.a y si
los puntos se reparten con igual proporci\'{o}n la covarianza ser\'{a}
negativa como indica la segunda gr\'{a}fica de la figura a\FRAME{ftbpFU}{%
3.0528in}{1.817in}{0pt}{\Qcb{fig a}}{}{fig4.wmf}{\special{language
"Scientific Word";type "GRAPHIC";maintain-aspect-ratio TRUE;display
"USEDEF";valid_file "F";width 3.0528in;height 1.817in;depth
0pt;original-width 5.6351in;original-height 3.333in;cropleft "0";croptop
"1";cropright "1";cropbottom "0";filename 'fig4.wmf';file-properties "NPEU";}%
}y no hay relaci\'{o}n matem\'{a}tica si la nube de puntos no tiene ninguna
tendencia como en la fig b\FRAME{dhFU}{2.7743in}{1.817in}{0pt}{\Qcb{fig b}}{%
}{fig4cb.wmf}{\special{language "Scientific Word";type
"GRAPHIC";maintain-aspect-ratio TRUE;display "USEDEF";valid_file "F";width
2.7743in;height 1.817in;depth 0pt;original-width 5.6455in;original-height
3.6772in;cropleft "0";croptop "1";cropright "1";cropbottom "0";filename
'fig4cb.wmf';file-properties "NPEU";}}

\end{document}
