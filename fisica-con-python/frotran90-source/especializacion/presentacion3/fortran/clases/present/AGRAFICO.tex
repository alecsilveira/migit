%\documentclass[12pt]{article}%
%\usepackage{graphicx}
%\usepackage{amsmath}
%\usepackage{zahlen}%
%\setcounter{MaxMatrixCols}{30}%
%\usepackage{amsfonts}%
%\usepackage{amssymb}
%%TCIDATA{OutputFilter=latex2.dll}
%%TCIDATA{Version=4.00.0.2312}
%%TCIDATA{CSTFile=LaTeX article (bright).cst}
%%TCIDATA{Created=Thu Nov 01 09:02:45 2001}
%%TCIDATA{LastRevised=Saturday, August 09, 2003 00:47:08}
%%TCIDATA{<META NAME="GraphicsSave" CONTENT="32">}
%%TCIDATA{<META NAME="DocumentShell" CONTENT="Journal Articles\Standard LaTeX Article">}
%\newtheorem{theorem}{Theorem}
%\newtheorem{acknowledgement}[theorem]{Acknowledgement}
%\newtheorem{algorithm}[theorem]{Algorithm}
%\newtheorem{axiom}[theorem]{Axiom}
%\newtheorem{case}[theorem]{Case}
%\newtheorem{claim}[theorem]{Claim}
%\newtheorem{conclusion}[theorem]{Conclusion}
%\newtheorem{condition}[theorem]{Condition}
%\newtheorem{conjecture}[theorem]{Conjecture}
%\newtheorem{corollary}[theorem]{Corollary}
%\newtheorem{criterion}[theorem]{Criterion}
%\newtheorem{definition}[theorem]{Definition}
%\newtheorem{example}[theorem]{Example}
%\newtheorem{exercise}[theorem]{Exercise}
%\newtheorem{lemma}[theorem]{Lemma}
%\newtheorem{notation}[theorem]{Notation}
%\newtheorem{problem}[theorem]{Problem}
%\newtheorem{proposition}[theorem]{Proposition}
%\newtheorem{remark}[theorem]{Remark}
%\newtheorem{solution}[theorem]{Solution}
%\newtheorem{summary}[theorem]{Summary}
%\newenvironment{proof}[1][Proof]{\textbf{#1.} }{\ \rule{0.5em}{0.5em}}
%\begin{document}
\section{AN\'{A}LISIS GR\'{A}FICO}

El \ an\'{a}lisis gr\'{a}fico es un m\'{e}todo que nos sirve para determinar
la relaci\'{o}n matem\'{a}tica entre dos o m\'{a}s variables, ( en nuestro
caso cantidades f\'{\i}sicas) partiendo de una gr\'{a}fica realizada a partir
de datos experimentales tomados al azar.

Este estudio lo vamos presentar en tres partes :\newline Presentaci\'{o}n de
la gr\'{a}fica, escogencia de un modelo matem\'{a}tico y establecimiento de la
relaci\'{o}n emp\'{\i}rica entre las variables, pero antes explicaremos como
se introducen los datos en el software llamado Data Studio, el cual
utilizaremos en todo el semestre
\begin{enumerate}
\item[a.] Usando el sensor.\newline El sensor transforma su lectura en datos
num\'{e}ricos usando la interface como puente entre el sensor y el software,
lo que quiere decir que los datos son introducidos automaticamente, y si hace
click en el icono table observara algo parecido a la\ref{Figura 1}
%TCIMACRO{\FRAME{dhFU}{1.1381in}{1.8922in}{0pt}{\Qcb{Figura 1}}{\Qlb{figura1}%
%}{table.jpg}{\special{ language "Scientific Word";  type "GRAPHIC";
%maintain-aspect-ratio TRUE;  display "USEDEF";  valid_file "F";
%width 1.1381in;  height 1.8922in;  depth 0pt;  original-width 1.1044in;
%original-height 1.8542in;  cropleft "0";  croptop "1";  cropright "1";
%cropbottom "0";  filename 'table.jpg';file-properties "XNPEU";}}}%
%BeginExpansion
\centering
\includegraphics[
height=1.8922in,
width=1.1381in
]%
{table.eps}%
\\
Figura 1
\label{figura1}%
%EndExpansion
\item[b.] Editando una tabla.\newline En la barra de men\'{u}
est\'{a}ndar
se\~{n}ale experiment y haga click en new empty data table(\ref{Figura 2}),%
%TCIMACRO{\FRAME{dhFU}{1.4313in}{1.6613in}{0pt}{\Qcb{Nueva tabla}%
%}{\Qlb{figura2}}{table2.jpg}{\special{ language "Scientific Word";
%type "GRAPHIC";  maintain-aspect-ratio TRUE;  display "USEDEF";
%valid_file "F";  width 1.4313in;  height 1.6613in;  depth 0pt;
%original-width 1.3958in;  original-height 1.625in;  cropleft "0";
%croptop "1";  cropright "1";  cropbottom "0";
%filename 'table2.jpg';file-properties "XNPEU";}}}%
%BeginExpansion
\centering
\includegraphics[
height=1.6613in,
width=1.4313in
]%
{table2.eps}%
\\
Nueva tabla
\label{figura2}%
%EndExpansion
luego aparece una tabla como la de\ref{Figura 2}.
%TCIMACRO{\FRAME{dhF}{2.7025in}{2.0332in}{0pt}{}{\Qlb{Figura2}}{datos1.jpg}%
%{\special{ language "Scientific Word";  type "GRAPHIC";
%maintain-aspect-ratio TRUE;  display "USEDEF";  valid_file "F";
%width 2.7025in;  height 2.0332in;  depth 0pt;  original-width 8.3333in;
%original-height 6.25in;  cropleft "0";  croptop "1";  cropright "1";
%cropbottom "0";  filename 'datos1.jpg';file-properties "XNPEU";}}}%
%BeginExpansion
\centering
\includegraphics[
height=2.0332in,
width=2.7025in
]%
{datos1.eps}%
\label{Figura2}%
%EndExpansion
Usando tab puede cambiar de celda para introducir datos.
\end{enumerate}
En caso de que se equivoque se puede borrar, se\~{n}alando y
haciendo click con el boton derecho en remove
\subsection{Presentaci\'{o}n de la gr\'{a}fica}
De acuerdo con el nivel de estas notas s\'{o}lo presentaremos una
introducci\'{o}n del an\'{a}lisis gr\'{a}fico en dos variables.
las cuales representaremos en un plano cartesiano de la siguiente
forma:
\begin{enumerate}
\item Titulo: El titulo lo colocamos en la parte superior as\'{\i}
.\newline(nombre de la variable en el eje $Y$) VS (nombre de la
variable en el eje $X$). \item En los ejes las letras que
representan a las variables y a sus unidades. Se indican las
escalas las cuales deben ser m\'{u}ltiplos de 2, 5 o 10 \item
Luego se grafican los puntos siguiendo algunos de los siguientes
m\'{e}todos
\begin{enumerate}
\item Se grafican los puntos y luego se intercala la gr\'{a}fica entre los
puntos, teniendo en cuenta que la gr\'{a}fica sea suave ( es decir que no
tenga v\'{e}rtices) y que la misma cantidad de puntos quede a un lado que al
otro de la gr\'{a}fica. por ejemplo%
%TCIMACRO{\FRAME{dtbpFU}{8.8194cm}{3.5981cm}{0pt}{\Qcb{Figura 1}}{\Qlb{Figura
%1}}{ajuste1.wmf}{\special{ language "Scientific Word";  type "GRAPHIC";
%maintain-aspect-ratio TRUE;  display "PICT";  valid_file "F";
%width 8.8194cm;  height 3.5981cm;  depth 0pt;  original-width 5.4864in;
%original-height 2.028in;  cropleft "0";  croptop "1";  cropright "1";
%cropbottom "0";  filename 'D:/Tacho/ajuste1.WMF';file-properties "XNPEU";}}}%
%BeginExpansion
\centering
\includegraphics[
height=3.5981cm,
width=8.8194cm
]%
{ajuste1.eps}%
\\
Figura 1
\label{Figura 1}%
%EndExpansion
No se deba trazar as\'{\i}:%
%TCIMACRO{\FRAME{dtbpFU}{11.1237cm}{4.4372cm}{0pt}{\Qcb{Figura 2}}{\Qlb{Figura
%2}}{ajuste2.wmf}{\special{ language "Scientific Word";  type "GRAPHIC";
%maintain-aspect-ratio TRUE;  display "PICT";  valid_file "F";
%width 11.1237cm;  height 4.4372cm;  depth 0pt;  original-width 5.4864in;
%original-height 2.1664in;  cropleft "0";  croptop "1";  cropright "0.9998";
%cropbottom "0";  filename 'D:/Tacho/ajuste2.WMF';file-properties "XNPEU";}}}%
%BeginExpansion
\centering
\includegraphics[
trim=0.000000in 0.000000in 0.001097in 0.000000in,
height=4.4372cm,
width=11.1237cm
]%
{ajuste2.eps}%
\\
Figura 2
\label{Figura 2}%
%EndExpansion
En las figura 1 observamos que se pueden trazar infinitas
gr\'{a}ficas que cumplen las condiciones establecidas. \newline Lo
que nos indica que la gr\'{a}fica que trazamos no es la
gr\'{a}fica real, \'{e}sta s\'{o}lo nos indica su forma.
\item Si
al tomar los datos podemos fijar cada medida de una variable Por
ejemplo: fijamos cada medida $x_{i}$ de $X$ entonces realizamos
varios ensayos para determinar cada valor $y_{i}$ de la variable
$Y$, obteniendo la media $\bar{y}_{i}$ para cada medida, luego
graficamos para cada medida los puntos $$\left(
x_{i},\bar{y}_{i}\right)  ,\left(  x_{i},y_{M\acute{a}x}\right)$$
y $$\left(  x_{i},y_{m\acute{\imath}n}\right)$$ como indica la
figura
%TCIMACRO{\FRAME{dtbpFU}{2.0245in}{2.0107in}{0pt}{\Qcb{Figura 3}}{}%
%{ydes.wmf}{\special{ language "Scientific Word";  type "GRAPHIC";
%maintain-aspect-ratio TRUE;  display "PICT";  valid_file "F";
%width 2.0245in;  height 2.0107in;  depth 0pt;  original-width 1.9865in;
%original-height 1.9726in;  cropleft "0";  croptop "1";  cropright "1";
%cropbottom "0";  filename 'D:/Tacho/ydes.WMF';file-properties "XNPEU";}}}%
%BeginExpansion
\centering
\includegraphics[
height=2.0107in,
width=2.0245in
]%
{ydes.eps}%
%EndExpansion
Luego con una serie de datos trazamos una gr\'{a}fica que pase por la
mayor\'{\i}a de los puntos medios o que toque todos los segmentos, as\'{\i}:%
%TCIMACRO{\FRAME{dtbpFU}{1.8922in}{1.8334in}{0pt}{\Qcb{Figura 4}}{}%
%{gdes.wmf}{\special{ language "Scientific Word";  type "GRAPHIC";
%maintain-aspect-ratio TRUE;  display "PICT";  valid_file "F";
%width 1.8922in;  height 1.8334in;  depth 0pt;  original-width 2.6247in;
%original-height 2.5417in;  cropleft "0";  croptop "1";  cropright "1";
%cropbottom "0";  filename 'D:/Tacho/gdes.WMF';file-properties "XNPEU";}}}%
%BeginExpansion
\centering
\includegraphics[
height=1.8334in,
width=1.8922in
]%
{gdes.eps}%
%EndExpansion
observemos que los segmentos indican la variabilidad de cada
medida, es decir la medida real puede ser cualquier valor sobre el
segmento, lo que nos indica que la curva real es una de las
infinitas que podemos trazar entre las gr\'{a}ficas punteadas, lo
que indica que este m\'{e}todo me determina la forma de la curva
real. \item Si cada medida $\left(  x_{i},y_{i}\right)  $ la
obtenemos al realizar varios ensayos, entonces graficamos cada
punto $\left(  \bar{x}_{i},\bar {y}_{i}\right)  $ tal que este en
la intersecci\'{o}n de las diagonales del rect\'{a}ngulo formado
por los lados de longitudes $\Delta x_{i},\Delta y_{i}$
como indica la figura 5%
%TCIMACRO{\FRAME{dtbpFU}{2.0833in}{1.5696in}{0pt}{\Qcb{Figura 5}}{}%
%{destg.wmf}{\special{ language "Scientific Word";  type "GRAPHIC";
%maintain-aspect-ratio TRUE;  display "PICT";  valid_file "F";
%width 2.0833in;  height 1.5696in;  depth 0pt;  original-width 2.0557in;
%original-height 1.542in;  cropleft "0";  croptop "1";  cropright "1";
%cropbottom "0";  filename 'D:/Tacho/destg.WMF';file-properties "XNPEU";}}}%
%BeginExpansion
\centering
\includegraphics[
height=1.5696in,
width=2.0833in
]%
{destg.eps}%
%EndExpansion
Con una serie de datos la curva debe pasar por todos los rect\'{a}ngulos como
indica la figura 6.%
%TCIMACRO{\FRAME{dphFU}{6.6031cm}{6.379cm}{0pt}{\Qcb{Figura 6}}{}%
%{gdes1.wmf}{\special{ language "Scientific Word";  type "GRAPHIC";
%maintain-aspect-ratio TRUE;  display "PICT";  valid_file "F";
%width 6.6031cm;  height 6.379cm;  depth 0pt;  original-width 6.4307in;
%original-height 6.2085in;  cropleft "0";  croptop "1";  cropright "1";
%cropbottom "0";  filename 'D:/Tacho/gdes1.WMF';file-properties "XNPEU";}}}%
%BeginExpansion
\centering
\includegraphics[
height=6.379cm,
width=6.6031cm
]%
{gdes1.eps}%
\\
Figura 6
%EndExpansion
De la figura observamos lo mismo que en los dos m\'{e}todos anteriores, que la
gr\'{a}fica que podemos trazar no es \'{u}nica
%\end{enumerate}
\item Si usamos Data Studio, se hace lo siguiente.
%\begin{enumerate}
\item Cuando digit\'{o} los datos observe que en la parte superior derecha de
la ventana aparecieron los siguientes iconos(\ref{figura3}),,
%TCIMACRO{\FRAME{dhF}{2.5728in}{1.8706in}{0pt}{}{\Qlb{figura3}}{datas.jpg}%
%{\special{ language "Scientific Word";  type "GRAPHIC";
%maintain-aspect-ratio TRUE;  display "USEDEF";  valid_file "F";
%width 2.5728in;  height 1.8706in;  depth 0pt;  original-width 2.5313in;
%original-height 1.8334in;  cropleft "0";  croptop "1";  cropright "1";
%cropbottom "0";  filename 'datas.jpg';file-properties "XNPEU";}}}%
%BeginExpansion
\centering
\includegraphics[
height=1.8706in,
width=2.5728in
]%
{datas.eps}%
\label{figura3}%
%EndExpansion
haga clic sobre el l\'{a}piz, para que aparezca%
%TCIMACRO{\FRAME{dhF}{2.7025in}{2.0332in}{0pt}{}{}{editable.jpg}%
%{\special{ language "Scientific Word";  type "GRAPHIC";
%maintain-aspect-ratio TRUE;  display "USEDEF";  valid_file "F";
%width 2.7025in;  height 2.0332in;  depth 0pt;  original-width 8.3333in;
%original-height 6.25in;  cropleft "0";  croptop "1";  cropright "1";
%cropbottom "0";  filename 'editable.jpg';file-properties "XNPEU";}}}%
%BeginExpansion
\centering
\includegraphics[
height=2.0332in,
width=2.7025in
]%
{editable.eps}
\label{figura3}%
%EndExpansion
\begin{enumerate}
\item Haciendo click en xlabel, ylabel puede colocar en label el
nombre de las variables,(en \'{\i}ngles si trabaja con sensores y
el programa no est\'{a} traducido), en units coloca las unidades
corespondientes a la variable en el sistema internacional (SI)
\item El t\'{\i}tulo se coloca en measurement, en accuracy se
coloca la precisi\'{o}n y en display precision el n\'{u}mero de
cifras decimales \item Haciendo click sobre el tri\'{a}ngulo o
figura que aparece debajo del l\'{a}piz y se arrastra hasta graph
como indica la \ref{figura4}
%TCIMACRO{\FRAME{dhF}{0.7187in}{1.0862in}{0pt}{}{\Qlb{figura4}}{graph.jpg}%
%{\special{ language "Scientific Word";  type "GRAPHIC";
%maintain-aspect-ratio TRUE;  display "USEDEF";  valid_file "F";
%width 0.7187in;  height 1.0862in;  depth 0pt;  original-width 0.6875in;
%original-height 1.0525in;  cropleft "0";  croptop "1";  cropright "1";
%cropbottom "0";  filename 'graph.jpg';file-properties "XNPEU";}}}%
%BeginExpansion
\centering
\includegraphics[
height=1.0862in,
width=0.7187in
]%
{graph.eps}%
\label{figura4}%
%EndExpansion
%\end{enumerate}
\end{enumerate}
\end{enumerate}
\subsection{Modelos matem\'{a}ticos}
Conociendo la forma de la curva podemos suponer un modelo
matem\'{a}tico, en esta secci\'{o}n presentaremos los cuatro
m\'{a}s usados
%\begin{enumerate}
\item Si el modelo es %$$y=ax^{m}\;\;a\in\rz ,\;m,a\neq0,\;m\in\qz
%^{+}$$
\begin{enumerate}
\item Se le llama Potencial si tiene la forma
%TCIMACRO{\FRAME{dtbpFU}{3in}{2.0003in}{0pt}{\Qcb{$m\in\gz^{+}\;m\neq1$}}%
%{}{Plot}{\special{ language "Scientific Word";  type "MAPLEPLOT";  width 3in;
%height 2.0003in;  depth 0pt;  display "PICT";  plot_snapshots TRUE;
%mustRecompute FALSE;  lastEngine "MuPAD";  xmin "-5";  xmax "5";
%xviewmin "0.001";  xviewmax "5.000000";  yviewmin "-0.001";
%yviewmax "10.100000";  viewset "XY";  plottype 4;  plottickdisable TRUE;
%numpoints 49;  plotstyle "patch";  axesstyle "normal";  xis \TEXUX{x};
%yis \TEXUX{y};  var1name \TEXUX{$x$};  var2name \TEXUX{$y$};
%function \TEXUX{$x^{3}$};  linecolor "black";  linestyle 1;
%pointstyle "point";  linethickness 1;  lineAttributes "Solid";
%var1range "-5,5";  num-x-gridlines 49;  curveColor "[flat::RGB:0000000000]";
%curveStyle "Line";  rangeset "X";  valid_file "T";
%tempfilename 'HJC61301.wmf';tempfile-properties "XPR";}}}%
%BeginExpansion
\centering
\includegraphics[
height=2.0003in,
width=3in
]%
{HJC61301.eps}%
\\
%$m\in\gz^{+}\;m\neq1$%
%EndExpansion
%\item Radical%
%%TCIMACRO{\FRAME{dtbpFU}{3in}{2.0003in}{0pt}{\Qcb{$m\in\qz^{+}-\gz^{+}$}}%
%%{}{Plot}{\special{ language "Scientific Word";  type "MAPLEPLOT";  width 3in;
%%height 2.0003in;  depth 0pt;  display "PICT";  plot_snapshots TRUE;
%%mustRecompute FALSE;  lastEngine "MuPAD";  xmin "-5";  xmax "5";
%%xviewmin "0.001";  xviewmax "5.000000";  yviewmin "0.428825";
%%yviewmax "2.272213";  viewset "XY";  plottype 4;  numpoints 49;
%%plotstyle "patch";  axesstyle "normal";  xis \TEXUX{x};  yis \TEXUX{y};
%%var1name \TEXUX{$x$};  var2name \TEXUX{$y$};  function \TEXUX{$\sqrt{x}$};
%%linecolor "black";  linestyle 1;  pointstyle "point";  linethickness 1;
%%lineAttributes "Solid";  var1range "-5,5";  num-x-gridlines 49;
%%curveColor "[flat::RGB:0000000000]";  curveStyle "Line";  rangeset "X";
%%valid_file "T";  tempfilename 'HJC6YV00.wmf';tempfile-properties "XPR";}}}%
%%BeginExpansion
%\centering
%\includegraphics[
%height=2.0003in,
%width=3in
%]%
%{HJC6YV00.eps)
%\\
%%$m\in\qz^{+}-\gz^{+}$%
%%EndExpansion
\end{enumerate}
\end{enumerate}
