


\begin{proof}
Se define $v:\Sigma^{\prime}\left(  \lambda\right)  \longrightarrow\mathbb{R}$
por $v:=u\left(  x^{\lambda}\right)  $ y se obtiene que
\begin{equation}
\Delta v\left(  x\right)  -b\left(  x^{\lambda}\right)  v_{1}\left(  x\right)
+g\left(  v\left(  x\right)  \right)  =0\text{ \ \ }\forall x\in\Sigma
^{\prime}\left(  \lambda\right)  \text{.}\tag{14}%
\end{equation}
$w:=v-u$ en $\Sigma^{\prime}\left(  \lambda\right)  $ satisface $w\leq0$ y
$w\neq0$. Luego de $\left(  13\right)  $, $\left(  14\right)  $ y el principio
del m\'{a}ximo fuerte se sigue que
\[
u\left(  x\right)  <u\left(  x^{\lambda}\right)  \text{ \ \ para todo \ }%
x\in\Sigma\left(  \lambda\right)  .
\]
Como $w=0$ sobre $T_{\lambda}$, del lema de frontera de Hopf se concluye
\[
u_{1}\left(  x\right)  <0\text{ \ \ para todo \ }x\in T_{\lambda}\text{.}%
\]

\end{proof}
