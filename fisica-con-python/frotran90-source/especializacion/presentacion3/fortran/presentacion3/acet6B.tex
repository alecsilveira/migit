




Ahora las funci\'{o}nes $u\in C^{2}\left(  \Omega\right)  \cap C\left(
\overline{\Omega}\right)  $ y el operador $L$ es
\begin{equation}
Lu:=\sum_{i,j=1}^{n}a_{ij}\left(  x\right)  D_{ij}^{2}u+\sum_{i=1}^{n}%
b_{i}\left(  x\right)  D_{i}u+c\left(  x\right)  u,\tag{9}%
\end{equation}
donde los coeficientes $a_{ij},b_{i}$ y $c$ est\`{a}n definidos en
un abierto, no vacio, $\Omega$ en $\mathbb{R}^{n}$, $n\geq2$, y
son acotados. Adem\'{a}s $A=\left[  a_{ij}\left(  x\right) \right]
$ sim\'{e}trica $\forall x\in\Omega$ y $L$ es estrictamente
el\'{\i}ptico, $\left(  2\right)  $.

\begin{theorem}
(Principio del m\'{a}ximo d\'{e}bil para $c\leq0$) si $Lu\geq0$ $\left(
Lu\leq0\right)  $ en $\Omega$ y $c\leq0$, entonces
\[
\max_{\overline{\Omega}}u\leq\max_{\partial\Omega}u^{+}\text{ \ }\left(
\min_{\overline{\Omega}}u\geq\min_{\partial\Omega}u^{-}\right)  .
\]

\end{theorem}
