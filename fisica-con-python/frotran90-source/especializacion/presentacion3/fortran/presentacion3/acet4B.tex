

%\section{ Problema de Dirichlet generalizado}
Con el principio del m\'{a}ximo generalizado (colorario) se sigue
que existe soluci\'{o}n d\'{e}bil \'{u}nica para el problema de
Direchlet generalizado. Adicionalmente en la prueba se utilizan
los siguientes teoremas: la alternativa Fredholm, Lax-Milgram, y
Rellich-Kondrackov el cual establece que
\[
W_{0}^{1,2}\left(  \Omega\right)  \subset\subset L^{2}\left(  \Omega\right)
\text{ \ si }n>2,
\]
y $\Omega$ es abierto acotado en $\mathbb{R}^{n}$ con $\partial\Omega$ de
clase $C^{1}$.
\begin{definition}
Sean $g$, $f\in L^{2}\left(  \Omega\right)  $ y $L$ el operador diferencial
definido en $\left(  1\right)  $, $u\in W^{1,2}\left(  \Omega\right)  $ es una
\textbf{soluci\'{o}n d\'{e}bil} de
\[
\left\{
\begin{array}
[c]{c}%
Lu=f\text{ \ en }\Omega,\\
u=g\text{ sobre }\partial\Omega,
\end{array}
\right.
\]
si $u-g\in W_{0}^{1,2}\left(  \Omega\right)  $ y adem\'{a}s
\begin{equation}
\mathcal{L}\left(  u,v\right)  =F\left(  v\right)  =-\int_{\Omega}fvdx\text{,
}\forall v\in C_{0}^{1}\left(  \Omega\right)  ,\tag{8}%
\end{equation}
donde $\mathcal{L}$ viene dado por $\left(  6\right)  .$
\end{definition}
\begin{lemma}
Si $I:W_{0}^{1,2}\left(  \Omega\right)  \longrightarrow\left(  W_{0}%
^{1,2}\left(  \Omega\right)  \right)  ^{\ast}$, $u\longrightarrow Iu$, donde
\[
Iu\left(  v\right)  :=\int_{\Omega}uvdx\text{, }v\in W_{0}^{1,2}\left(
\Omega\right)  .
\]
Entonces $I$ es una inmersi\'{o}n compacta.
\end{lemma}
