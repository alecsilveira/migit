\documentclass[12pt]{article}
\usepackage{addmath}
\usepackage[spanish]{babel}

\begin{document}

\section{Integraci\'on num\'erica b\'asica}

\hfill{\tiny Versi\'on final 1.0-19 agosto 2002\footnote{Esta secci\'on
    est\'a basada en el  cap{\'\i}tulo 10 del libro: {\em Numerical
      Methods for Physics, second edition} de Alejandro L. Garcia,
    editorial {\sc Prentice Hall}.}}

En esta secci\'on veremos algunos m\'etodos b\'asicos para
realizar cuadraturas (evaluar integrales num\'ericamente).

\subsection{Regla trapezoidal}

Consideremos la integral
\begin{equation}
I = \int_a^b f(x)\, dx \ .  
\end{equation}
Nuestra estrategia para estimar $I$ es evaluar $f(x)$ en unos cuantos
puntos, y ajustar una curva simple a trav\'es de estos
puntos. Primero, subdividamos el intervalo $[a,b]$ en $N-1$
subintervalos. Definamos los puntos $x_i$ como
$$ x_1=a, x_N=b, x_1<x_2<\cdots<x_{N-1}<x_N \ . $$
La funci\'on $f(x)$ s\'olo se eval\'ua en estos puntos, de modo que
usamos la notaci\'on $f_i\equiv f(x_i)$.

El m\'etodo m\'as simple de cuadratura es la {\em regla
  trapezoidal}. Conectamos los puntos $f_i$ con l\'{\i}neas rectas, y
la funci\'on lineal formada por la sucesi\'on de l\'{\i}neas rectas es
nuestra curva aproximante. La integral de esta funci\'on aproximante
es f\'acil de calcular, pues es la suma de las \'areas de
trapezoides. El \'area de un trapezoide es
$$ T_i = \frac 12 (x_{i+1}-x_i)(f_{i+1}+f_i) \ . $$
La integral es entonces calculada como la suma de las \'areas de los
trapezoides:
$$ I \simeq I_T = T_1 + T_2 + \cdots + T_{N-1} \ . $$

Las expresiones son m\'as simples si tomamos puntos equidistantes. El
espaciado es $h=\frac{b-a}{N-1}$, de modo que $x_i = a+h(i-1)
$. El \'area de un trapezoide es entonces
$$ T_i = \frac 12 h (f_{i+1}+f_i) \ , $$
y la regla trapezoidal para puntos equiespaciados queda
\begin{align}
  I_T(h) &= \frac 12 h f_1 + h f_2 + h f_3 + \cdots + h f_{N-1} +
  \frac 12 h f_N \nonumber \\
&= \frac 12 h(f_1+f_N) + h \sum_{i=2}^{N-1} f_i 
\end{align}

Como ejemplo, consideremos la funci\'on error
\begin{equation}
  \erf(x) = \frac 2{\sqrt{\pi}} \int_0^x e^{-y^2} \, dy \ . 
\end{equation}
Para $x=1$, $\erf(1) \simeq 0.842701$. La regla trapezoidal con $N=5$
da $0.83837$, que es correcto a dos lugares decimales. Por supuesto,
el integrando en este ejemplo es muy suave y bien comportado.

El error de truncamiento para la regla trapezoidal se puede escribir
como
\begin{equation}
 I - I_T(h) = - \frac 1{12} (b-a) h^2 f''(\zeta) \ , 
\end{equation}
para alg\'un $\zeta$ en $[a,b]$, o bien como
\begin{equation}
\label{error2}
  I-I_T(h) = -\frac 1{12} h^2[f'(b)-f'(a)]+O(h^4) \ . 
\end{equation}
Vemos que el error es proporcional a $h^2$, y que la regla trapezoidal
tendr\'a problemas cuando la derivada diverge en los extremos del
intervalo. Por ejemplo, la integral $\int_0^b \sqrt{x}\, dx$ es
problem\'atica. 

\subsection{Integraci\'on de Romberg}

Una pregunta usual es  cu\'antas subdivisiones del intervalo
realizar. Un modo de decidir es repetir el c\'alculo con un intervalo
m\'as peque\~no. Si la respuesta no cambia apreciablemente, la
aceptamos como correcta (esto no evita que podamos ser enga\~nados por
funciones patol\'ogicas o escenarios inusuales, sin embargo). Con la
regla trapezoidal, si el n\'umero de paneles es una potencia de dos,
podemos dividir el tama\~no del intervalo por dos sin tener que
recalcular todos los puntos.

Definamos la secuencia de tama\~nos de intervalo,
\begin{equation}
  h_1 = (b-a) \ , h_2 = \frac 12(b-a) \ , \ldots \ , h_n = \frac
  1{2^{n-1}} (b-a) \ . 
\end{equation}
Para $n=1$ s\'olo hay un panel, luego
\begin{equation*}
  I_T(h_1) = \frac 12(b-a)[f(a)+f(b)] = \frac 12 h_1[f(a)+f(b)] \ . 
\end{equation*}
Para $n=2$ se a\~nade un punto interior, luego
\begin{align*}
  I_T(h_2) &=\frac 12 h_2[f(a)+f(b)]+h_2f(a+h_2) \\
& = \frac 12 I_T(h_1) + h_2f(a+h_2) \ .
\end{align*}
Hay una f\'ormula recursiva para calcular $I_T(h_n)$ usando
$I_T(h_{n-1})$:
\begin{equation}
  I_T(h_n) = \frac 12I_T(h_{n-1}) + h_n \sum_{i=1}^{2^{n-2}}
  f[a+(2i-1)h_n] \ . 
\end{equation}
El segundo t\'ermino del lado derecho da la contribuci\'on de los
puntos interiores que se han agregado cuando el tama\~no del intervalo
es reducido a la mitad. 

Usando el m\'etodo recursivo descrito, podemos agregar paneles hasta
que la respuesta parezca converger. Sin embargo, podemos mejorar
notablemente este proceso usando un m\'etodo llamado {\em
  integraci\'on de Romberg}. Primero veamos la mec\'anica, y despu\'es
explicaremos por qu\'e funciona. El m\'etodo calcula los elementos
de una  matriz triangular:
\begin{equation}
  R = 
\begin{matrix}
R_{1,1} & \text{--} & \text{--} & \text{--} \\
R_{2,1} & R_{2,2} & \text{--} & \text{--} \\
R_{3,1} & R_{3,2} & R_{3,3} & \text{--} \\
\vdots & \vdots & \vdots & \ddots
\end{matrix}
\end{equation}

La primera columna es simplemente la regla trapezoidal recursiva:
\begin{equation}
  R_{i,1} = I_T(h_i) \ . 
\end{equation}
Las sucesivas columnas a la  derecha se calculan usando la f\'ormula
de extrapolaci\'on de Richardson:
\begin{equation}
  R_{i+1,j+1} = R_{i+1,j} + \frac 1{4^j-1} [ R_{i+1,j} - R_{i,j} ] \
.
\end{equation}
La estimaci\'on m\'as precisa para la integral es el elemento
$R_{N,N}$. 


El programa \verb+romberg.cc+ calcula la integral $\erf(1)$ usando el
m\'etodo de Romberg. Con $N=3$, se obtiene la tabla
\begin{verbatim}
0.771743        0        0 
0.825263 0.843103        0 
0.838368 0.842736 0.842712 
\end{verbatim}
$R_{3,3}$ da el resultado exacto con 4 decimales, usando los mismos 4
paneles que antes usamos con la regla trapezoidal (y que ahora
reobtenemos en el elemento $R_{3,1}$). 

Es \'util que el programa entregue toda la tabla y no s\'olo el
\'ultimo t\'ermino, para tener una estimaci\'on del error. Como en
otros contextos, usar una tabla demasiado grande puede no ser
conveniente pues errores de redondeo pueden comenzar a degradar la
respuesta. 

\vspace{.3cm}
Para entender por qu\'e el esquema de Romberg funciona, consideremos el
error para la regla trapezoidal, $E_T(h_n) = I-I_T(h_n)$. Usando
\eqref{error2}, 
\begin{equation*}
  E_T(h_n) = -\frac 1{12} h^2_n [f'(b) - f'(a)] + O(h_n^4) \ . 
\end{equation*}
Como $h_{n+1} = h_n/2$,
\begin{equation*}
  E_T(h_{n+1}) = -\frac 1{48} h^2_n [f'(b)-f'(a)]+O(h_n^4) \ . 
\end{equation*}
Consideremos ahora la segunda columna de la tabla de Romberg. El error
de truncamiento para $R_{n+1,2}$ es:
\begin{align*}
  I-R_{n+1,2} &= I - \left\{
    I_T(h_{n+1})+\frac13[I_T(h_{n+1})-I_T(h_n)]\right\} \\
&=E_T(h_{n+1})+\frac 13[E_T(h_{n+1})-E_T(h_n)]\\
&=-\left[\frac 1{48} + \frac 13 \left(\frac 1{48} - \frac 1{12}\right)
  \right] h_n^2 [f'(b)-f'(a)]+O(h_n^4) \\
&=O(h_n^4)
\end{align*}
Notemos c\'omo el t\'ermino $h_n^2$ se cancela, dejando un error de
truncamiento de orden $h_n^4$. La siguiente columna (la tercera) de la
tabla de Romberg cancela este t\'ermino, y as\'{\i} sucesivamente. 

\section{Listados del programa.}

\subsection{\tt trapecio.cc}
\begin{verbatim}
#include "NumMeth.h"

double integrando(double);

int main(){
  double a=0, b=1,x;
  int N;
  double h;
  double I, suma=0;
  cout << "Regla trapezoidal para erf(1)" << endl;
  cout << "Numero de puntos: " ;
  cin >> N;
  
  h = (b-a)/(N-1);
  for (int i=2;i<N;i++){
    x = a+(i-1)*h;
    suma+=integrando(x);
  }
  I = h*((integrando(a)+integrando(b))/2.0 + suma);
  cout << "Valor aproximado: erf(1) ~= " << I << endl;
  cout << "Valor exacto:     erf(1)  = 0.842701" << endl;
}

double integrando(double x){
  return (2.0/sqrt(M_PI))*exp(-x*x);
}
  
\end{verbatim}

\subsection{\tt romberg.cc}

\begin{verbatim}
#include "NumMeth.h"
#include <iomanip.h>

double integrando(double);

int main(){
  double a=0, b=1,x;
  int N,np;
  double h;
  double I, suma=0;
  cout << "Integracion de Romberg para erf(1)" << endl;
  cout << "Dimension de tabla: " ;
  cin >> N;
  double ** R = new double * [N];
  for (int i=0;i<N;i++){
    R[i] = new double[N];
  }

  for (int i=0;i<N;i++){
    for (int j=0;j<N;j++){
      R[i][j] = 0 ;
    }
  }

//   for (int i=0;i<N;i++){
//     for (int j=0;j<N;j++){
//       cout << R[i][j] << " ";
//     }
//     cout << endl;
//   }

  h = (b-a); // Ancho del intervalo
  np = 1;    // Numero de paneles
  for (int i=0;i<N;i++,h/=2,np*=2){
 
    if (i==0){
      R[0][0] = h*(integrando(a)+integrando(b))/2.0;
    }
    else {
      suma = 0;
      for (int j=1;j<=np-1;j+=2){
        x = a+j*h;
        suma += integrando(x);
      }
      R[i][0] = R[i-1][0]/2.0 + h*suma;
    }
  }
  
  int m = 1;
  for (int j=1;j<N;j++){
    m *= 4;
    for (int i=j;i<N;i++){
      R[i][j] = R[i][j-1] + (R[i][j-1]-R[i-1][j-1])/(m-1);
    }
  }
  
  cout << "Tabla R" << endl << endl;
  for (int i=0;i<N;i++){
    for (int j=0;j<N;j++){
      cout << /*setprecision(6) << */ setw(8) <<  R[i][j] << " ";
    }
    cout << endl;
  }

  cout << endl;
  cout << "Valor aproximado: erf(1) ~= " << R[N-1][N-1] << endl;
  cout << "Valor exacto:     erf(1)  = 0.842701" << endl;

  for (int i=0;i<N;i++){
    delete [] R[i];
  }
  delete [] R;
}

double integrando(double x){
  return (2.0/sqrt(M_PI))*exp(-x*x);
}
  

\end{verbatim}

\end{document}
