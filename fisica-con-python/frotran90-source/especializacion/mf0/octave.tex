\documentclass[12pt]{article}
\usepackage{addmath,agrandar}
\usepackage[spanish]{babel}

\begin{document}

\section{Visualizaci\'on de datos con Octave}

\hfill{\tiny Versi\'on final 1.0-19 agosto 2002}

Octave es un poderoso software de c\'alculo num\'erico. 
En este documento explicamos un subconjunto realmente peque\~no de
sus comandos, centrados en la visualizaci\'on de datos. Para un
estudio m\'as detallado de Octave, recomendamos 
los apuntes de {\it ``Introducci\'on a los M\'etodos de la F\'{\i}sica
  Matem\'atica''}, cap\'{\i}tulo 4
 (``Una breve introducci\'on a Octave/Matlab''). 

\subsection{Comandos b\'asicos}

Digamos que tenemos un programa en C++, \verb+interpolacion.cc+, que
toma tres puntos (pares de coordenadas $(x^0_i,y^0_i)$, $x^0_1<x^0_2<x^0_3$), 
y construye una curva
de interpolaci\'on cuadr\'atica en el intervalo $[x^0_1,x^0_3]$ con
ellos. La salida del programa es un archivo \verb+curva.dat+, que
contiene una lista de pares ordenados:
\begin{verbatim}
x_1    y_1
x_2    y_2
x_3    y_3
etc.
\end{verbatim}
Deseamos ver esta curva de interpolaci\'on. Un modo de hacerlo es
utilizar Octave. Para ello, primero abrimos Octave, luego cargamos el
archivo de datos, y por \'ultimo graficamos los datos.

Existen tres modos de usar Octave: en modo interactivo o por medio de
scripts, y \'estos a su vez pueden ser aut\'onomos o no. Cada uno
tiene sus propias virtudes y desventajas, y depender\'a del usuario
cu\'al es el modo m\'as ventajoso en cada circunstancia particular.

\subsubsection{Modo interactivo}

Para abrir octave, basta escribir \verb+octave+ en la l\'{\i}nea de comandos:
\begin{verbatim}
vmunoz@llacolen:~/cursos/computacion$ octave
GNU Octave, version 2.0.16 (i386-pc-linux-gnu).
Copyright (C) 1996, 1997, 1998, 1999, 2000 John W. Eaton.
This is free software with ABSOLUTELY NO WARRANTY.
For details, type `warranty'.

octave:1> 
\end{verbatim}

Despu\'es de la presentaci\'on aparece un prompt, y podemos comenzar a
introducir comandos uno por uno. Esto es lo que se llama {\it modo
  interactivo}. Para ver la curva dada por los datos en
\verb+curva.dat+, basta escribir los siguientes comandos uno tras otro
en el prompt de Octave:
\begin{verbatim}
octave:1>load curva.dat;
octave:2>plot(curva(:,1),curva(:,2));
\end{verbatim}

La primera l\'{\i}nea carga los contenidos de \verb+curva.dat+,
ley\'endolos como si fueran una matriz llamada \verb+curva+.
As\'{\i}, dada la estructura del archivo de datos, \verb+curva+
resulta ser una matriz de $N\times2$, de modo que las  abscisas
de la curva est\'an en la primera columna, y las ordenadas en la
segunda. El archivo de datos podr\'{\i}a llamarse de cualquier manera,
y tener cualquier extensi\'on; Octave asignar\'a los datos contenidos
en \'el a una matriz que se llama igual que el archivo, sin
extensi\'on. La \'unica restricci\'on es que los datos deben
efectivamente estar distribuidos formando una matriz, de $N$ filas por
$M$ columnas, con $N,M\geq1$. 

En la segunda l\'{\i}nea, el comando \verb+plot+ dibuja las ordenadas,
contenidas en la segunda columna (\verb+curva(:,2)+ significa algo
como ``tome los elementos de \verb+curva+ contenidos en todas las
filas, para la columna \verb+2+''), versus las abscisas, contenidas en
la primera columna (\verb+curva(:,1)+). En general, si \verb+x+,
\verb+y+ son vectores, \verb+plot(x,y)+ grafica \verb+y+ vs.\
\verb+x+. 

Un n\'umero arbitrario de l\'{\i}neas de comandos de Octave se pueden
colapsar en una sola usando \verb+;+ como separador. El ejemplo
anterior es equivalente a:
\begin{verbatim}
octave:1>load curva.dat;plot(curva(:,1),curva(:,2));
\end{verbatim}


Para salir de Octave escribimos \verb+quit+ en el prompt.

\vspace{.3cm}

Digamos ahora que queremos, adem\'as de la curva, graficar los tres
puntos que la generaron. Para ello podemos hacer que nuestro programa
\verb+interpolacion.cc+ cree un segundo archivo, \verb+puntos.dat+,
que tiene los pares ordenados de dichos puntos, en el mismo formato
que \verb+curva.dat+ (abscisas en la primera columna, ordenadas en la
segunda). Entonces damos las siguientes instrucciones en el prompt:
\begin{verbatim}
octave:1>load curva.dat;
octave:2>plot(curva(:,1),curva(:,2));
octave:3>hold on;
octave:4>load puntos.dat;
octave:5>plot(puntos(:,1),puntos(:,2),'ob');
octave:6>hold off;
\end{verbatim}

La instrucci\'on \verb+hold on+ permite ``congelar'' el
gr\'afico actual, permitiendo que todos los gr\'aficos sucesivos se
superpongan al primero, en vez de borrarlo que es el comportamiento
default. \verb+hold off+ ``descongela'' el gr\'afico, y el pr\'oximo
gr\'afico reemplazar\'a al actual.

Observamos tambi\'en que el segundo  \verb+plot+ tiene un argumento
adicional. \'Esta es una cadena de caracteres que especifica el formato
para graficar la familia de puntos especificados. Un car\'acter indica
el tipo de punto con que se graficar\'a: \verb+'o'+ para c\'{\i}rculos, '.'
para puntos, \verb+'x'+ para cruces, \verb+'-'+ para unir los puntos con
l\'{\i}neas. El otro car\'acter indica el color: \verb+'b'+ es azul ({\it
  blue\/}), \verb+'r'+ es rojo ({\it red\/}), \verb+'k'+
 es negro ({\it black\/}),
etc\@. Ambas especificaciones pueden ir en cualquier orden:
\verb+'ob'+, \verb+'bo'+, \verb+'k-'+, etc\@. Si una o ambas
especificaciones no se indican, se adopta el valor default, que es
\verb+'-'+ para el tipo de l\'{\i}nea y \verb+'r'+ para el
color. As\'{\i}, \verb+plot(x,y)+ dibujar\'a los puntos unidos por una
l\'{\i}nea roja; \verb+plot(x,y,'b')+ har\'a una l\'{\i}nea azul,
\verb+plot(x,y,'o')+ har\'a puntos rojos, etc. 

Es muy posible que en una misma sesi\'on de Octave queramos cargar dos
veces el mismo archivo. Por ejemplo, si corremos el programa en un
terminal, y tenemos Octave abierto en otro para graficar los datos
resultantes, podr\'{\i}amos querer correr el programa de nuevo en el
primer terminal, y rehacer el gr\'afico con los datos actualizados. 
\verb+load curva.dat+ dar\'a un error en ese caso, porque la matriz
\verb+curva+ ya existe y no puede ser reescrita con nuevos datos de un
archivo. Para forzar a Octave a reemplazar la matriz con los nuevos
datos se debe usar \verb+load -force curva.dat+. 


\subsubsection{Scripts}

Existe otro modo de usar Octave, y es a trav\'es de {\it scripts}, que
no son sino archivos con extensi\'on \verb+.m+, en los cuales cada
l\'{\i}nea contiene una instrucci\'on o instrucciones para Octave, que
se ejecutan igual que si fueran dadas en el prompt en modo
interactivo.

As\'{\i}, podemos rehacer el ejemplo de la curva de interpolaci\'on
creando un archivo de texto, llamado por ejemplo
\verb+interpolacion.m+ (la extensi\'on \verb+.m+ es obligatoria), con
los siguientes contenidos:
\begin{verbatim}
% interpolacion.m
%    Grafica los datos generados por interpolacion.cc

load curva.dat;
plot(curva(:,1),curva(:,2));
hold on;
load puntos.dat;
plot(puntos(:,1),puntos(:,2),'ob');
hold off;
\end{verbatim}

Las primeras dos l\'{\i}neas son comentarios. Octave ignora cualquier
texto desde un \verb+%+ hasta el siguiente cambio de
l\'{\i}nea. \verb+#+ tambi\'en sirve para introducir comentarios.
La tercera l\'{\i}nea es una l\'{\i}nea en blanco. Podemos poner
l\'{\i}neas en blanco a nuestro arbitrio dentro del archivo para 
hacerlo m\'as claro. Son ignoradas por Octave. 

En ocasiones necesitamos escribir una instrucci\'on que es demasiado 
grande para que quepa en una sola l\'{\i}nea. Octave acepta que una
misma l\'{\i}nea de instrucciones sea separada en dos, siempre que
sean conectadas por puntos suspensivos (\verb+...+). Un ejemplo un
poco in\'util es el siguiente:
\begin{verbatim}
load ...
curva.dat;
\end{verbatim}
y
\begin{verbatim}
load ...
    curva.dat;
\end{verbatim}
son equivalentes a \verb+load curva.dat;+

Supongamos entonces que  hemos escrito nuestro archivo
\verb+interpolacion.m+,
y que est\'a en el directorio
\verb+/home/vmunoz/cursos/computacion/+.
Invocamos a Octave desde este directorio, y simplemente escribimos en
el prompt \verb+interpolacion+ (sin extensi\'on). El efecto de esto es
que Octave busca si existe una funci\'on interna con ese nombre; como
no la encuentra, busca si existe en el directorio actual (o en el {\it
  path\/} predefinido\footnote{Ojo: No es la misma variable
  {\tt PATH} del shell.}), un archivo \verb+interpolacion.m+, y lo
ejecuta. 

\subsubsection{Scripts aut\'onomos}

Una tercera manera de ver nuestros datos es usando scripts
aut\'onomos. Para ello, modificamos nuestro script, agreg\'andole al
comienzo la l\'{\i}nea:
\begin{verbatim}
#!/usr/bin/octave -q
\end{verbatim}
que indica d\'onde est\'a el ejecutable de Octave (podemos preguntarle
al sistema operativo d\'onde est\'a en la m\'aquina que usamos dando
en el prompt del shell la instrucci\'on \verb+which octave+). La
opci\'on \verb+-q+ es simplemente para evitar que aparezca en pantalla
la presentaci\'on de Octave (autor, versi\'on, etc.). A continuaci\'on
damos permiso de ejecuci\'on a nuestro archivo:
\begin{verbatim}
vmunoz@llacolen:~/cursos/computacion$chmod u+x interpolacion.m 
\end{verbatim}
Ahora el script se puede ejecutar como si fuera un comando m\'as del
shell, en la forma
\begin{verbatim}
vmunoz@llacolen:~/cursos/computacion$./interpolacion.m
\end{verbatim}

M\'as a\'un, ahora puede ser invocado como un comando del sistema
desde el propio programa \verb+interpolacion.cc+, sin necesidad de
abrir un terminal aparte para visualizar los datos. Esto se hace con
el comando \verb+system+, disponible en \verb+stdlib.h+. As\'{\i}, 
si \verb+interpolacion.cc+ tiene la forma:
\begin{verbatim}
#include <fstream.h>
#include <stdlib.h>

int main(){

   ofstream archivo_curva("curva.dat");
   ofstream archivo_puntos("puntos.dat");

// Comandos de calculo de los datos
//   y escritura de archivos de salida
//   curva.dat y puntos.dat

   archivo_curva.close();
   archivo_puntos.close();

   system("./interpolacion.m");
}
\end{verbatim}
ser\'a nuestro propio programa el que se encargue de llamar al script
y mostrar la visualizaci\'on de los datos. 

El procedimiento anterior tiene la desventaja de que la ventana
gr\'afica se cierra tan pronto como el programa
\verb+interpolacion.cc+ termina. Por ello, si se llama a un script
aut\'onomo, es conveniente poner una pausa para que tengamos tiempo de
ver algo. Poniendo al final de \verb+interpolacion.m+ el comando
\verb+pause(s); +, con \verb+s+ un n\'umero real o entero, la
ejecuci\'on de Octave se detendr\'a durante \verb+s+ segundos, y de
ese modo alcanzaremos a ver el resultado mientras nuestro programa en
C++ se ejecuta.

\end{document}
